\documentclass{vldb}
%\documentclass[letterpaper,twocolumn,10pt]{article}
%\usepackage{times}
\usepackage{epsfig,endnotes}

%========================
%  Packages
%========================

\usepackage{graphicx,url,color}
\usepackage{amsmath}
\usepackage{amssymb}
%\usepackage{amsthm}  %<---- for a different "look" in theorems (theorem word in bold, etc.) %ACM conflict with proof definition?
%\usepackage{subfigure}
%\usepackage[tight,footnotesize]{subfigure}
\usepackage{algorithm}
\usepackage[noend]{algpseudocode}
\usepackage{times}
\usepackage[outercaption]{sidecap}    


%\usepackage{todonotes}
%\usepackage[normalem]{ulem} %strikethrough: \sout{Hello World}
%\usepackage{lastpage} %for number of pages
%\usepackage{xspace}
%\usepackage{multirow}
%\usepackage{balance}

\usepackage[colorlinks=true,allcolors=blue,breaklinks]{hyperref}   % hyperlinks, including DOIs and URLs in bibliography
\usepackage{subcaption}
%\usepackage{caption}
\usepackage{float}
%\usepackage{floatrow}

\usepackage{balance}  % for  \balance command ON LAST PAGE  (only there!)

% Include information below and uncomment for camera ready
\vldbTitle{Taking Omid to the Clouds: Fast, Scalable Transactions for Real-Time Cloud Analytics}
\vldbAuthors{Shacham et al}
\vldbDOI{https://doi.org/TBD}


%========================
%  Macros
%========================

\newcommand{\code}[1]{\textsf{\fontsize{9}{11}\selectfont #1}}

\newcommand{\inred}[1]{{\color{red}{#1}}}
\newcommand{\remove}[1]{}
\newcommand{\Idit}[1]{[[\inred{Idit: #1}]]}
\newcommand{\Yoni}[1]{[\inred{Yoni: #1}]}
\newcommand{\tb}{\hspace{10mm}}

\newcommand{\sys}{Omid~FP}
\newcommand{\sysll}{Omid~LL}
\newcommand{\syspc}{Omid~2PC}

\newcommand{\speedup}[1]{#1$\times$}
\newcommand{\tuple}[1]{\ensuremath{\langle \mbox{#1} \rangle}}

\newcommand{\mypara}[1]{ \vspace{8pt} \noindent{\bf #1}\hspace{6pt}}

%========================
\begin{document}

\date{}

\title{Taking Omid to the Clouds: \\ 
Fast,  Scalable Transactions for Real-Time Cloud Analytics }

\numberofauthors{6} 
\author{
\alignauthor
Ohad Shacham\\
       \affaddr{Yahoo Research}
       \email{}
\alignauthor
Yonatan Gottesman\\ %\footnotemark[1]
       \affaddr{Yahoo Research}
       \email{}
\alignauthor
Aran Bergman\\
       \affaddr{Technion}
       \email{}
\and
\alignauthor
Edward Bortnikov\\
       \affaddr{Yahoo Research}
       \email{}
\alignauthor
Eshcar Hillel\\
       \affaddr{Yahoo Research}
       \email{}
\alignauthor
Idit Keidar\\
       \affaddr{Technion and Yahoo Research}
       \email{}
} % end author


\maketitle

%=========================================================================
%  Abstract
%=========================================================================

\begin{abstract}

We describe how we evolve Omid, a transaction processing system for Apache HBase, 
to power Apache Phoenix, a cloud-grade real-time SQL analytics engine.  

Omid was originally designed for data processing pipelines at Yahoo, which are, by and large, 
%huge-scale 
throughput-oriented monolithic NoSQL applications. 
%Evolving Omid into 
Providing a platform 
to support 
%a variety of 
converged real-time transaction processing and analytics applications --
dubbed {\em translytics} --  
introduces new functional and performance requirements. For example, SQL 
support is key for developer productivity, multi-tenancy is essential for cloud deployment, 
and latency is cardinal for just-in-time data ingestion and analytics insights.
% in a variety of applications. 

We discuss our efforts to adapt Omid to these new domains, 
as part of 
%which emerged in 
the process of integrating it into Phoenix as the transaction processing backend. A central piece
of our work is latency reduction in Omid's protocol, which also improves  scalability.  
Under light load, the new protocol's latency is 4x to 5x smaller than the legacy Omid's, whereas 
under increased loads it is an order of magnitude faster. We further describe a \emph{fast path} f
protocol or single-key transactions, which enables processing them almost as fast 
as native HBase operations.

\end{abstract}

%========================

\section{Introduction} \label{sec:intro}


Popular production \emph{transaction processing systems (TPSs)} 
such as Percolator, Omid, Tephra, and CockroachDB 
support transactions over data that resides in distributed NOSQL key-value stores such 
as BigTable, HBase, and RocksDB. 
While such NOSQL stores are typically widely distributed, the aim of TPSs is to 
support  \emph{global transactions}, i.e., ones
that span multiple \emph{regions} (sometimes called nodes, shards, domains, or partitions).


\idit{Reword below: it's not about centralized TM, it's about adding a transaction tier above the data tier.}


%using centralized timestamp oracles, also called \emph{transaction managers} (TMs) . 
TMs provide a \emph{global
version clock} (GVC) [?] and sometimes additional functionalities such as conflict detection and
tracking of committed transactions[?]. 
Hence, central coordination is an inherent part of their design. This is in
contrast with federated systems, where transactions are by default local, i.e.,
access a single region, and protocols like two phase commit are used to ensure
atomicity of global transactions.

Using a centralized TM introduces a tradeoff: while transactions spanning
multiple regions are greatly facilitated and expedited using this design, local
transactions incur a performance penalty. This penalty is particularly
significant for short transactions, where the overhead of accessing the TM once
(to read the GVC) or twice (also in order to commit) per transaction is not
amortized across many operations. Typical Web workloads include both long
multi-region transactions and short local transactions, including ones
consisting of a single object read, write or read-modify-write.

In this work we mitigate the above tradeoff by offering a fast path for short
local transactions, without a significant impact on multi-region ones while
maintaining correctness. We consider a single-TM system that supports global
transactions, and enhance it with optimized support for specific types of local
transactions. Our main goal is to expedite short single-object transactions,
which are popular in production Web workloads [?].

We focus here on systems supporting \emph{snapshot isolation} (SI) [?], which is
popular in real-world systems [?] and amenable to scalable implementations [?].
The transaction semantics for local transactions are the same as for global
ones, and in particular, a local transaction can be performed using the usual
transaction mechanism, (which accesses the TM), with no change in semantics.

We implement our solution in Omid [?], an open source transaction processing
engine for key-value stores. Omid is database-neutral, while its open-sourced
implementation uses HBase. Our solution consists of two main parts - extensions
to the underlying key-value store, which we implement in HBase, and client-side
extensions to the transaction library, which we implement in Omid�s client
library. In addition, our implementation slightly modifies Omid�s TM to be
compatible with the changes to HBase.
  

\section{Background} \label{sec:api}
A Transaction Processing System runs atop an underlying 
key-value store and allows users to bundle multiple data operations 
into a single atomic transaction. Section~\ref{ssec:data-model} describes the NoSQL data 
model and the data access API, and  Section~\ref{ssec:transactions} defines transaction semantics
provided by the TPS. Section~\ref{ssec:schema} provides  background on the modus operandi 
of existing TPSs that support SI,  including Omid. Finally, Section~\ref{ssec:bigdata} overviews HBase and Phoenix. 

\subsection{Data store API}
\label{ssec:data-model}

The  data store holds  \emph{objects} (often referred to as \emph{rows} or {\em records}) identified by unique \emph{keys}.
Each row can consist of multiple \emph{fields}, representing different \emph{columns}. 
We consider multi-versioned objects, where object values are associated with \emph{version numbers}, and
multiple versions associated with the same key may co-exist. {\inred{The store may garbage-collect 
obsolete data versions.}}
\remove{
Thus, at any given time, an object holds a tuple \tuple{key,\tuple{version,value}+}, where value
can be structured to consist of multiple columns.
}
We further assume that a write operation can specify the version number it writes to.
%monotonically increasing ??
%\paragraph{API} 
The  data store provides the following API:
\begin{description}
\item [{get(key, version)}] --  returns the requested version of key.
%if no version is provided, returns the latest; 
%if no field is provided returns the entire record. 
%\item 
The API further allows traversing (reading) earlier versions of the same key.
% in descending order.
{\inred{
\item [{scan (fromKey, toKey, version)}] -- a range query (extension of get). 
Returns an iterator that supports retrieval of multiple records. 
}}
\inred{\item [{put(key, version, fields, values)}] -- 
creates or updates an object, setting the specified fields to the specified values. 
If the version already exists, its value is updated; otherwise, a new version is added. 
}
%Typically, its semantics are non-atomic for the sake of implementation efficiency. 
%\item [{update(key, version, fields, values)}] -- 
%creates or updates an object, setting the specified fields to the specified values. 
%If the version already exists, its value is updated; otherwise, a new version is added. 
\remove{The data store may buffer the write in memory until an ensuing flush. }
\item [{remove(key, version)}] -- removes an object with the given key and version.
\item [{check\&mutate(key, version, field, old, new)}] -- checks the record associated with key and version. 
\inred{If field holds old, replace it with new and return old; otherwise return field value.}
\inred{If old is nil, create field with new only if field is missing and return value accordingly.}
\remove{ \item [\code{flush}] -- persists all previous updates to disk.}
%data stores often provide means to atomically read and update a single object, e.g., HBase exports check\&mutate operations, which are 
%internally implemented using a per-row RW lock.
%, whereas BigTable supports row transactions. 
%We will extend this capability below in order to implement certain atomic operations at the data store level.
\end{description}

\noindent
{\inred{Most NoSQL data store implementations guarantee atomicity (all-of-nothing semantics) 
of all the above API's, except scan.}

\subsection{Transaction semantics} \label{ssec:transactions}

TPSs provide \emph{begin} and \emph{commit} APIs for delineating transactions: 
a \emph{transaction} is a sequence of \emph{read} and \emph{write} operations on different objects 
that occur between begin and commit. {\inred{(For simplicity, we only specify single-key operations 
here; the semantics of multi-key range queries (scans) are identical to reads.)}}
Note that transactional reads and writes are implemented using the 
datastore's get and put operations.
Two transactions are said to be \emph{concurrent} if 
their executions overlap, i.e., one of them begins between the begin time and commit time of the other;
otherwise, we say that they are \emph{non-overlapping}.

A TPS  ensures the ACID properties for transactions:
\emph{atomicity} (all-or-nothing), \emph{consistency} (preserving each object's semantics), 
\emph{isolation} (in that concurrent transactions do not see each other's partial updates), and 
\emph{durability} (whereby updates survive crashes).

Different isolation levels can be considered for the third property. We consider a variant of 
\emph{snapshot isolation (SI)}~\cite{DBLP:conf/sigmod/BerensonBGMOO95} that, similarly to \emph{generalized snapshot isolation}~\cite{DBLP:conf/srds/ElniketyZP05}, relaxes  the real-time order requirement. 
Nevertheless, our implementation only relaxes the ordering of fast path  transactions (described in Section~\ref{sec:alg}) 
relative to regular ones (that do not use the fast path); regular transactions continue to satisfy SI amongst themselves. 
Moreover, \sysll, without the fast path, satisfies SI as Omid does.

Our relaxed correctness condition satisfies the key ``snapshot'' property of SI, which ensures that a transaction reading from the  database
does not see a mix old and new values. For example, if a transaction updates the values of two stocks, 
then no other transaction may observe the old value of one of these stocks and the new value of the other.
However, it relaxes the real-time order guarantee of SI by allowing (fast-path) transactions to take effect `in the past'.  
 \remove{ % Idit: removed the example and figure to save space
Similarly, a regular transaction overlapping
two fast path ones may observe an update of the second and miss an update by the first,  as illustrated in 

Figure~\ref{fig:ltx-rt}, shows an example where fast path transaction FP2 is ordered `in the past'.
%Yet we do enforce real-time order on regular transactions as well as on all updates of the same key.

\begin{figure}[ht]
\includegraphics[width=\columnwidth]{figs/FP-semantics}
\caption{Possible violation of real-time order among fast path transactions. Regular transaction $T_1$
reads $x$ before it is updated by fast path transaction $FP_1$ and reads $y$ after it is updated by fast path transaction $FP_2$ even 
though $FP_2$ occurs after $FP_1$. 
%$T1$'s global version is $10$, and its skips the local version clocks of the regions holding $x$ and $y$ to $10$ when reading from them.
}
\label{fig:ltx-rt}
\end{figure}
} % remove
Specifically, %our correctness condition stipulates that
the system enforces a total order ${\cal T}$ on all committed transactions, so that
\begin{enumerate}
    \setlength{\itemsep}{0pt}
    \setlength{\parskip}{0pt}
    \setlength{\parsep}{2pt}  
%\item
%regular transactions (though not FP ones) are ordered in ${\cal T}$  according to their commit times;
\item
non-overlapping transactions 
%(regular and FP) 
that update the same key occur in ${\cal T}$  in order of their commit times;
\item
each  transaction's read operations see a consistent snapshot of the database reflecting 
a prefix of  ${\cal T}$; 
%and  
%that includes at least all regular transactions committed prior to its start time; and 
\item
 a transaction commits only if none of the items it updates is modified by a transaction ordered in ${\cal T}$ after
 its snapshot time and before its commit time.
 \end{enumerate}

\remove{ % OLD SI definition - no relaxation of RTO
More precisely, 
SI enforces a total order on committed transactions according to their commit times so that 
\begin{enumerate}
    \setlength{\itemsep}{0pt}
    \setlength{\parskip}{0pt}
    \setlength{\parsep}{2pt}  
\item
each transaction's read operations see a consistent snapshot of the database reflecting write operations by
 exactly those transactions that committed prior to the transaction's start time; and 
\item
 a transaction commits only if none of the items it updates has been modified since that snapshot.
 \end{enumerate}
 } %Remove
 
Note that as with SI, two concurrent transactions conflict only if they both \emph{update} the same item.  
In contrast, under serializability, a transaction that updates an item also conflicts with transactions that \emph{read} that item. 
Snapshot isolation is thus amenable to implementations (using multi-versioning) that 
allow more concurrency than serializable ones, and hence scale better.
It is provided by popular database technologies such as Oracle, PostgreSQL, and SQL Server,
and TPSs such as Percolator, Omid, and  CockroachDB.

Following a commit call, the transaction may successfully \emph{commit}, whereby all of its operations take effect, 
or 
%in case of conflicts, (i.e., when two concurrent transactions attempt to update the same item), the transaction may
\emph{abort}, in which case none of its changes take effect. 



%An abort may also be initiated by the programmer, e.g., 
%on encountering an error. Applications typically retry a transaction upon  abort. 


%%%The data is \emph{partitioned} (or sharded), and each object belongs to one region. 
%%%%Global transactions may span multiple regions, and atomically commit or abort on all. 
%%%\emph{Local transactions} are ones that access a single region.



\subsection{TPS operation schema}
\label{ssec:schema}

\begin{figure}
\centerline{
\includegraphics[width=0.45\textwidth]{FragolaComponents.jpg}
}
\caption{Transaction processing architecture: A client library exposes an  API for  executing transactions of data store operations. 
A centralized Transaction Manager (TM) handles transaction begin and commit requests, while data is written directly to the 
underlying data store. The TM has a backup for high availability.}
\label{fig:components}
\end{figure}

Figure~\ref{fig:components} depicts, at a high level, the primary components of the TPS architecture,  
their API's, and interaction with the data store. 

In many TPSs, transaction processing follows the following general schema, 
outlined in Algorithm~\ref{alg:schema}, while systems vary in their implementations of each of the steps.

\remove{
For example, whereas most systems rely on a centralized service for timestamp allocation~\cite{OmidICDE2014,Omid2017,tephra,Percolator2010}, this is not essential~\cite{cockroach}; similarly, validation (conflict detection) can use a centralized service~\cite{OmidICDE2014,Omid2017,tephra}, per-transaction entries in a global table~\cite{cockroach}, or distributed locking and validation~\cite{Percolator2010}. 
Different ways to implement this schema are  discussed in Section~\ref{sec:context}.
We now overview the phases a transaction goes through, focusing on Omid's approach.
}

\begin{algorithm}[tb]
\begin{algorithmic}[1]
%\small
\Procedure{begin}{}
\State obtain read timestamp $ts_r$ 
\EndProcedure
%\Statex

\Procedure{write}{$ts_r$, key, \inred{fields, values}} \hspace{-0.2cm} \Comment transactional write
\State optionally check for conflicts and abort if found 
\State indicate write intent for key with \inred{values} and $ts_r$
\State add key to local write-set
\EndProcedure
%\Statex

\Procedure{read}{$ts_r$, key} \Comment transactional read
\If{key has write intent}
	\State resolve, possibly abort writing transaction \label{l:resolve}
\EndIf
\State return highest version   $\le ts_r$ of key
\EndProcedure

%\Statex

\Procedure{commit}{$ts_r$, write-set}
\Statex \Comment check for write-write conflicts  \label{l:validate}
\State obtain commit timestamp $ts_c$
\If{validate(write-set, $ts_r$)}  
%	\Statex \Comment commit all write intents with version $ts_c$
	\State write commit  version $ts_c$ to persistent commit entry \label{l:commit}
\Else
	\State abort	
\EndIf
\State post-commit: update meta-data
\EndProcedure

\end{algorithmic}
\caption{TPS operation schema.} 
\label{alg:schema}
\end{algorithm} 

Most of the systems employ a centralized \emph{transaction manager (TM)\/} service~\cite{Percolator2010,OmidICDE2014,Omid2017,tephra},
 sometimes called timestamp oracle, for timestamp allocation and other functionalities. 
 Because a centralized service can become a single point of failure, the TM is sometimes implemented
 as a primary-backup server pair to ensure its continued availability following failures.

\mypara{Begin.} 
  When a transaction begins, it obtains a read timestamp (version) $ts_r$ for reading its consistent snapshot.
  %, and unique transaction id.   The two can be combined (i.e., $ts_r$ can serve as the  transaction id, provided that it is unique).
 In most cases, this is done using the centralized TM~\cite{Percolator2010,OmidICDE2014,Omid2017,tephra}. 
%  In CockroachDB, the timestamp is based on a local clock that is ``close to'' real-time and preserves causality 
 % across regions, and unique transaction ids are used to break ties in case timestamps (from different regions) are identical. 

\mypara{Transactional writes.} 
 During a transaction, a write operation indicates its \emph{intent} to write to a single object a certain new value with a certain version number.
%a dedicated \emph{commit} column in the object indicates that the write is tentative.
In Omid, the version is the transaction's $ts_r$, which exceeds all versions written by transactions that committed before the
current transaction began. Note that the version order among concurrent transactions that  attempt to update the same key is immaterial, 
since all but one of these transactions are doomed to abort. 

It is possible to buffer write intents locally (at the client) in the course of the transaction, and add the write intents to the data store at commit time~\cite{Percolator2010}.

In some solutions writes check for conflicts before declaring their intents~\cite{cockroach}, whereas in others, 
all conflict detection is deferred to commit time~\cite{Percolator2010,OmidICDE2014,Omid2017,tephra}. 

\mypara{Transactional reads.} 
The reads of a given transaction obtain a consistent snapshot of the data store at logical time (i.e., version) $ts_r$.
Each read operation retrieves the value of a single object associated with the highest timestamp that is 
smaller or equal to the transaction's $ts_r$. 

On encountering a write intent, read cannot proceed without determining whether the tentative write should be included in its snapshot,
for which it must know the writing transaction's commit status. 
To this end, TPSs keep per-transaction \emph{commit entries}, which are the source of truth regarding the transaction status 
(pending, committed, or aborted). 
This entry is updated in line~\ref{l:commit} of Algorithm~\ref{alg:schema} as we explain below, 
and is checked in order to resolve write intents in line~\ref{l:resolve}.
In some cases~\cite{Percolator2010,cockroach}, when the status of the writing transaction  is undetermined, the read forcefully aborts
it by updating the commit entry accordingly, as explained below.

%Similarly, the solution we implement in \sys\ forces the transaction with the pending write intent to abort. 

  \mypara{Commit.} 
  Commit occurs in four steps:
  \begin{enumerate}
    \setlength{\itemsep}{0pt}
    \setlength{\parskip}{0pt}
    \setlength{\parsep}{2pt}  
  \item
  Obtain a commit timestamp, $ts_c$. 
  In most cases, e.g.,~\cite{Percolator2010,OmidICDE2014,Omid2017,tephra}, 
  this is the value of some global clock maintained by a centralized entity. 
  \item \emph{Validate} that the transaction does not conflict with any concurrent transaction that has committed since it 
had begun.  For SI, we need to check for write-write conflicts only. 
If write intent indications are buffered, they are added at this point~\cite{Percolator2010}.
Validation can be centralized~\cite{OmidICDE2014,Omid2017,tephra} or distributed~\cite{Percolator2010,cockroach}. 


\item \emph{Commit} or abort in one  irrevocable atomic step by persistently writing to the \emph{commit entry}, 
  which can reside in a global table~\cite{Omid2017,cockroach} or alongside the first  key written by 
  the transaction~\cite{Percolator2010}.  
  
 \item \emph{Post-commit}: 
  Finally, a transaction changes its write intents to
  persistent writes in case of commit, and removes them in case of abort. This
  step is not essential for correctness, but reduces the overhead of future transactions. It
  occurs after the transaction is persistently committed or aborted via the commit entry, 
  and can be done asynchronously.
  %in
  %order to reduce the overhead of future transactions (by sparing them the need to check the commit entry)
  %and to garbage collect   obsolete information. 
 \end{enumerate}
 
  \remove{Note that whenever a transaction encounters a write
  indication in the collect phase it must access the commit entry in order to
  check the transaction's commit status. Once the post-commit phase is over, future
  transactions no longer incur this overhead for keys updated by the terminated
  transaction.}

\subsection{Big data platforms}
\label{ssec:bigdata}

{\inred{

Apache HBase is one of the most scalable key-value storage technologies available today. 
Like many state-of-the-art data stores, it scales through horizontal \emph{sharding} 
(partitioning) of data across \emph{regions}. An HBase instance is deployed 
on multiple nodes (\emph{region servers}), each of which typically serves hundreds 
of regions. Production HBase clusters of 1K nodes and above are becoming common. 
For example, Yahoo Japan leverages an HBase cluster of $3{,}800$ nodes 
that collectively store $37$PB of data~\cite{yahoojapanhbase}. 

Phoenix complements the HBase storage tier with a query processing (compute)
tier. The latter scales independently (the current scalability goal is $10{,}000$ query servers). 
Phoenix compiles every SQL statement into a plan, and executes it on one or more servers. 
Its query processing code invokes the underlying HBase for low-level data access, and a 
TPS (Omid or Tephra) for transaction management, through client libraries. 

Wherever possible, Phoenix strives to push computation close to data (e.g., for filtering
and aggregation), in order to minimize cross-tier communication. For this, it makes an
extensive use of server-side stored procedures, which in HBase are supported by the 
non-intrusive {\em coprocessor\/} mechanism. Omid uses HBase coprocessors too, either for 
performance improvements or for specific services (e.g., garbage collection of redundant 
data). 
}}

\section{Low-Latency Transactions} \label{sec:ll}


We now describe \sysll, a scalable low-latency TPS algorithm 
% on top of HBase. 
%\sys\ is an evolution of the open-source Omid TPS, redesigned to reduce latency.
%For clarity of the presentation, in this section we describe \sys\ without the fast-path; 
%This protocol 
that satisfies standard (unrelaxed) SI semantics and is amenable to multi-tenancy.
We begin in Section~\ref{ssec:schema} with  background on the modus operandi of existing TPSs that support SI, including Omid. 
We will see that, while many TPSs follow a similar schema,  they make different design choices when implementing this schema. 
We discuss our design choices in Section~\ref{ssec:ll-txns}. 
We then proceed to give a detailed description of the protocol
in Section~\ref{ssec:ll}.

\subsection{Background: Schema of TPS operation}
\label{ssec:schema}


In many TPSs, transaction processing follows the following general schema, outlined in Algorithm~\ref{alg:schema}, 
while systems vary in their implementations of each of the steps.

\remove{
For example, whereas most systems rely on a centralized service for timestamp allocation~\cite{OmidICDE2014,Omid2017,tephra,Percolator2010}, this is not essential~\cite{cockroach}; similarly, validation (conflict detection) can use a centralized service~\cite{OmidICDE2014,Omid2017,tephra}, per-transaction entries in a global table~\cite{cockroach}, or distributed locking and validation~\cite{Percolator2010}. 
Different ways to implement this schema are  discussed in Section~\ref{sec:context}.
We now overview the phases a transaction goes through, focusing on Omid's approach.
}

\begin{algorithm}[tb]
\begin{algorithmic}[1]
%\small
\Procedure{begin}{}
\State obtain read timestamp $ts_r$ 
\EndProcedure
%\Statex

\Procedure{write}{$ts_r$, key, value} \Comment transactional write
\State optionally check for conflicts and abort if found 
\State indicate write intent for key with value and $ts_r$
\State add key to local write-set
\EndProcedure
%\Statex

\Procedure{read}{$ts_r$, key} \Comment transactional read
\If{key has write intent}
	\State resolve, possibly abort writing transaction \label{l:resolve}
\EndIf
\State return highest version   $\le ts_r$ of key
\EndProcedure

%\Statex

\Procedure{commit}{$ts_r$, write-set}
\Statex \Comment check for write-write conflicts  \label{l:validate}
\State obtain commit timestamp $ts_c$
\If{validate(write-set, $ts_r$)}  
%	\Statex \Comment commit all write intents with version $ts_c$
	\State write commit  version $ts_c$ to persistent commit entry \label{l:commit}
\Else
	\State abort	
\EndIf
\State post-commit: update meta-data
\EndProcedure

\end{algorithmic}
\caption{TPS operation schema.} 
\label{alg:schema}
\end{algorithm} 

Most of the systems employ a centralized \emph{transaction manager (TM)}~\cite{Percolator2010,OmidICDE2014,Omid2017,tephra},
 sometimes called timestamp oracle, for timestamp allocation and other functionalities. 
 Because a centralized service can become a single point of failure, the TM is sometimes implemented
 as a primary-backup server pair to ensure its continued availability following failures.

\mypara{Begin.} 
  When a transaction begins, it obtains a read timestamp (version) $ts_r$ for reading its consistent snapshot.
  %, and unique transaction id.   The two can be combined (i.e., $ts_r$ can serve as the  transaction id, provided that it is unique).
 In most cases, this is done using the centralized TM~\cite{Percolator2010,OmidICDE2014,Omid2017,tephra}. 
%  In CockroachDB, the timestamp is based on a local clock that is ``close to'' real-time and preserves causality 
 % across regions, and unique transaction ids are used to break ties in case timestamps (from different regions) are identical. 

\mypara{Transactional writes.} 
 During a transaction, a write operation indicates its \emph{intent} to write to a single object a certain new value with a certain version number.
%a dedicated \emph{commit} column in the object indicates that the write is tentative.
In Omid, the version is the transaction's $ts_r$, which exceeds all versions written by transactions that committed before the
current transaction began. Note that the version order among concurrent transactions that  attempt to update the same key is immaterial, 
since all but one of these transactions are doomed to abort. 

It is possible to buffer write intents locally (at the client) in the course of the transaction, and add the write intents to the data store at commit time~\cite{Percolator2010}.

In some solutions writes check for conflicts before declaring their intents~\cite{cockroach}, whereas in others, 
all conflict detection is deferred to commit time~\cite{Percolator2010,OmidICDE2014,Omid2017,tephra}. 

\mypara{Transactional reads.} 
The reads of a given transaction obtain a consistent snapshot of the data store at logical time (i.e., version) $ts_r$.
Each read operation retrieves the value of a single object associated with the highest timestamp that is 
smaller or equal to the transaction's $ts_r$. 

On encountering a write intent, read cannot proceed without determining whether the tentative write should be included in its snapshot,
for which it must know the writing transaction's commit status. 
To this end, TPSs keep per-transaction \emph{commit entries}, which are the source of truth regarding the transaction status 
(pending, committed, or aborted). 
This entry is updated in line~\ref{l:commit} of Algorithm~\ref{alg:schema} as we explain below, 
and is checked in order to resolve write intents in line~\ref{l:resolve}.
In some cases~\cite{Percolator2010,cockroach}, when the status of the writing transaction  is undetermined, the read forcefully aborts
it by updating the commit entry accordingly, as explained below.

%Similarly, the solution we implement in \sys\ forces the transaction with the pending write intent to abort. 

  \mypara{Commit.} 
  Commit occurs in four steps:
  \begin{enumerate}
    \setlength{\itemsep}{0pt}
    \setlength{\parskip}{0pt}
    \setlength{\parsep}{2pt}  
  \item
  Obtain a commit timestamp, $ts_c$. 
  In most cases, e.g.,~\cite{Percolator2010,OmidICDE2014,Omid2017,tephra}, 
  this is the value of some global clock maintained by a centralized entity. 
  \item \emph{Validate} that the transaction does not conflict with any concurrent transaction that has committed since it 
had begun.  For SI, we need to check for write-write conflicts only. 
If write intent indications are buffered, they are added at this point~\cite{Percolator2010}.
Validation can be centralized~\cite{OmidICDE2014,Omid2017,tephra} or distributed~\cite{Percolator2010,cockroach}. 


\item \emph{Commit} or abort in one  irrevocable atomic step by persistently writing to the \emph{commit entry}, 
  which can reside in a global table~\cite{Omid2017,cockroach} or alongside the first  key written by 
  the transaction~\cite{Percolator2010}.  
  
 \item \emph{Post-commit}: 
  Finally, a transaction changes its write intents to
  persistent writes in case of commit, and removes them in case of abort. This
  step is not essential for correctness, but reduces the overhead of future transactions. It
  occurs after the transaction is persistently committed or aborted via the commit entry, 
  and can be done asynchronously.
  %in
  %order to reduce the overhead of future transactions (by sparing them the need to check the commit entry)
  %and to garbage collect   obsolete information. 
 \end{enumerate}
 
  \remove{Note that whenever a transaction encounters a write
  indication in the collect phase it must access the commit entry in order to
  check the transaction's commit status. Once the post-commit phase is over, future
  transactions no longer incur this overhead for keys updated by the terminated
  transaction.}



\subsection{\sysll\ design choices}
\label{ssec:ll-txns}

We now discuss our design choices, which are geared towards high performance without sacrificing cloud deployability.
They are compared with the choices made in other TPSs in  Table~\ref{table:design-space}. 
We are not familiar with another TPS that makes the same design choices as \sysll. 


\begin{table*}[htb]
\small
\centerline{
\begin{tabular}{|l|ccccc|}
\hline
TPS & validation & data store modification & commit  entry updates	&  multi-tenancy &
 read may cause abort  \\
\hline
Percolator, \syspc & {\bf D}  & yes & {\bf D} & no & yes \\
CockroachDB	& {\bf D}     & yes & {\bf D} & yes & yes \\
Omid1, Tephra 	& {\bf C}  & no & {\bf R} & yes & no \\
Omid  		& {\bf C}  & no & {\bf C} & yes & no \\
{\bf \sysll }		& {\bf C}  & {\bf no} & {\bf D} & {\bf yes} & {\bf yes} \\
% \sys			& {\bf C} & {\bf D} & {\bf D} & yes& yes\\
\hline
\end{tabular}
}
\caption{Design choices  in TPSs. {\bf C} -- centralized,  {\bf D} -- distributed, {\bf R} -- replicated.}
\label{table:design-space}
\end{table*}



\mypara{Centralized validation.}
\sysll\ adopts Omid's centralized conflict detection mechanism, which eliminates the need for locking objects
in the data store, and is extremely scalable~\cite{Omid2017}. 

Other TPSs (like Percolator and  CockroachDB~\cite{cockroach}) instead use a distributed 2PC-like protocol that locks all written objects during validation (either at commit time or during the write). To this end, they use atomic check\&mutate operations on the underlying data store. This slows down either commits (in case of commit-time validation) or transactional writes (in case of write-time validation), which takes a toll on long transactions, where validation time is substantial. 
\remove{
%that can abort either the current transaction or a conflicting one. 
Moreover, it may hold locks for a long period (in case of a long transaction),
during which conflicts lead to aborts.  
}

To allow us to compare the two approaches, we also implement a 2PC-based version of Omid, \syspc.  
%Furthermore, 
%this approach does not entail any changes to the underlying data store (HBase in our implementation),
%which can facilitate production deployment.
\inred{
We note that unlike our approach, this variant \emph{does} necessitate changes in the 
underlying data store to allow transaction validation (conflict detection) using check\&mutate for each written key. 
}
%Using check\&mutate for all updates makes data store writes slower,

\mypara{Distributed commit entry updates with  multi-tenancy.}
The early generation of Omid~\cite{OmidICDE2014} (referred to as Omid1) and Tephra replicate commit entries 
of pending transactions among all active clients, which consumes high bandwidth and does not scale. Omid 
%and CockroachDB 
instead uses a dedicated table, and 
%CockroachDB updates the table in a distributed manner, while Omid 
has the centralized TM persist all commits to this table. 
Our experiments show that the centralized access to commit entries is Omid's main scalability bottleneck, 
and while this bottleneck is mitigated via batching commit table updates, this also increases latency.
Omid chose this  option as it was designed for high throughput. 

Here, on the other hand, we target  low latency. 
We therefore distribute the commit entry updates, and allow commit entries of different transactions to be 
updated in parallel by independent clients. We will see below that this modification reduces latency
by up to an order of magnitude on small transactions.

\inred{
We note that commit table updates are distributed also in CockroachDB and Percolator. 
Percolator takes this approach one step further, and distributes not only the commit table updates 
but also the actual commit entries. There, commit entries reside in user data tables, where the first table 
accessed in a given transaction holds the commit entry for that transaction.
The problem with this approach is that it assumes that all clients have permissions to access all 
data tables -- a transaction attempting to read from  table $A$ may encounter a write intent 
produced by a transaction that accessed table $B$ before table $A$, and will need to refer to 
that transaction's commit entry in table $B$ in order to determine its status. 
This did not pose a problem in Percolator, which was designed for use inside Google's 
data centers, but is usually unacceptable in multi-tenant data stores.

Unlike data tables, which are owned by individual applications that manage their permissions,  
the dedicated commit table is owned by the TPS. It is not accessed direclty by application code,
but rather from transactional code that is part of the TPS.
}

\mypara{Write intent resolution.}
As in other TPSs, reads resolve write intents via the commit entry.
If the transaction status is committed, the commit time is checked, and, if smaller than or equal to $ts_r$, it is taken into account;
if the transaction is aborted, the value is ignored.
In case the transaction status is pending, \sys, like Percolator and CockroachDB, has the reader
force the writing transaction to abort. This is done using an atomic check\&mutate operation to set the status in the
writing transaction's commit entry to aborted.

Omid and Tephra, on the other hand, do not need to force such aborts\footnote{Omid's high availability mechanism  
may force such aborts in rare cases of TM failover.}, because they ensure that  if the read sees a write intent
by an uncommitted transaction, the latter will not commit with an earlier timestamp than the read.
This is achieved  
%a single TM that allocates timestamps, performs validation, and writes the commit entry.
either by delaying begin and commit requests until all transactions that concurrently attempt to commit complete (as in Omid),
or sending information about all these transactions to the beginning client (as in Omid1 and Tephra).
\sysll\ avoids such costly mechanisms by allowing reads to force aborts. 



\remove{

The centralized service, called Transactional Status Oracle (TSO), maintains an in-memory hash table mapping keys to 
commit timestamps ($ts_c$) of transactions that last wrote them. A conflict arises whenever a key in a transaction's write-set has been written 
with a timestamp higher than its $ts_r$. 
In case the conflict detection service crashes, all pending transactions are aborted, and it can be immediately restarted with an empty table, because only conflicts with concurrent transactions need to be checked. 


%
The possibility of reads aborting pending transactions means that commit attempts (line~\ref{l:commit}) have to  
check whether the transaction is aborted atomically with writing the commit status.


\subsection{Local transactions in \sys}
\label{ssec:fp-impl}
To support local transactions, we implemented the algorithms and APIs described in Section~\ref{sec:alg}. Each Hbase region server maintains an LVC which is updated whenever a transaction accesses it. Omid's API was extended to support FP transactions, and the write stage in regular transactions was modified to use checkAndMutate instead of put. 


 }
 
\subsection{\sysll\ algorithm} 
\label{ssec:ll}

\begin{algorithm}[htb]
\begin{algorithmic}[1]
\small
\Procedure{begin}{}
\State checkRenew() \Comment check and renew lease for HA support
\State return Clock.fetchAndIncrement()
\EndProcedure
%\Statex
\Procedure{commit}{$txid$, write-set}
\State checkRenew() \Comment check and renew lease for HA support
\State $ts_c \leftarrow$ Clock.fetchAndIncrement()
\If{conflictDetect($txid$, write-set, $ts_c$) } 
 \State return $ts_c$
 \Else
 \State return {\sc abort}
\EndIf
\EndProcedure
%\Statex
%
\remove{
\Procedure{checkRenew}{} \Comment HA support;  $\delta$ is the lease time
\If{lease $<$ now + \inred{0.2}$\delta$} \label{l:lease-start} \Comment extend lease
 \State renew lease for $\delta$ time \Comment atomic operation
 \If{failed} halt \EndIf 
\EndIf  \label{l:lease-end}
\If{Clock $=$ epoch}  \label{l:epoch-start} \Comment extend epoch
\State epoch $\leftarrow$ Clock +  range
 
\If{$\neg$CAS(maxTS, Clock, epoch) } halt \EndIf 
\EndIf  \label{l:epoch-end}
\EndProcedure
}
\end{algorithmic}
\caption{\sysll's TM algorithm.}
\label{alg:ha}
\end{algorithm}



When a transaction begins, it obtains a read timestamp (version) $ts_r$ for reading its consistent snapshot.
To commit, it obtains  a commit timestamp, $ts_c$ and persists a commit entry in the CT that indicates that
the transaction with read timestamp $ts_r$ has committed with timestamp $ts_c$. 

Whereas Omid's CT is updated by the TM, \sysll\ distributes the CT updates amongst the clients.
Its TM is thus a simplified version of Omid's TM, and appears  in Algorithm~\ref{alg:ha}. It has two roles:
First, it allocates read and commit timestamps by fetching-and-incrementing a monotonically increasing global clock.
Second, upon commit, it calls the \emph{conflictDetect} function, which 
performs validation  using an in-memory hash table
by checking, for each key in the write-set, that its commit timestamp in the hash-table is smaller than the 
committing transaction's $ts_r$. If there are no conflicts, it 
updates the hash table with the write-set of the new transaction and its $ts_c$. 
%(For more details see~\cite{Omid2017}).
The  \emph{checkRenew} procedure supports the TM's  high availability (HA), and its description is beyond the scope of this paper; see~\cite{Omid2017,OmidVLDB} for details. 
%explained at the end of this section.



\noindent
Client operations proceed as follows (cf.~Algorithm~\ref{fig:get-pseudocode}):


\begin{algorithm}[htb]
 \small
  \begin{algorithmic}
 \Procedure{begin}{}
       \State  return TM.begin
\EndProcedure
%\Statex
  \Procedure{write}{$ts_r$, key, {fields, values}}
       \State  track key in write-set 
       \State {add commit to fields and nil to values}
       \State    {return ds.put(key, $ts_r$, fields, {values})}
\EndProcedure
%     
%\Statex
\Procedure{read}{$ts_r$, key} 
      \For{rec $\leftarrow$ ds.get(key, versions down from $ts_r$)}
% 			\State $ts_c \leftarrow$ {\sc checkStatus}(rec, $ts_r$)     			       
			\State \Comment  set $ts_c$ to the commit timestamp associated with rec
			\If{rec.commit $\not =$nil}  \Comment commit cell exists
     				\State $ts_c \leftarrow$  rec.commit 
			\Else \Comment commit cell is empty, check CT
      				\State $ts_c  \leftarrow$ CT.get(rec.ts)  
      				\If{$ts_c =$ nil}  \Comment no CT entry, set status to  $\bot$
      					\State {$ts_c\leftarrow$CT.check\&mutate(rec.ts, commit, nil, $\bot$)}  
      					\If{$ts_c$=nil}  \Comment forced abort successful 
      						\State $ts_c  \leftarrow \bot$ 
      					\EndIf
				\EndIf
      				
 				 \If{$ts_c = \bot$}  
 				 	\State \Comment check for race: commit before $\bot$ entry  created 
 				 	\State re-read rec from ds
 				 	\If{rec.commit $\not =$nil}
 				 		 \State $ts_c \leftarrow$  rec.commit 
 				 		 \State CT.remove(rec.ts)
 				 	\Else\  
 				 		 continue	\Comment writing transaction aborted	
 				 	\EndIf
				\EndIf
			\EndIf

       			 \If{$ts_c <  ts_r$}  
%     			 \If{$ts_c \not=$ abort  $\wedge\ ts_c <  ts_r$}  
 			 	\State return  rec.value 
			\EndIf
      \EndFor
      \State  return nil \Comment no returnable version found in loop
      \EndProcedure



%      		\Procedure{checkLeader}{\tuple{k, ts}} 
%     			\State leader $\leftarrow$ ds.get(\tuple{k, ts})
%    			 \If{leader $=$ nil} return abort \Comment{leader removed} \EndIf 
%		     	 \If{ leader.commit $\not=$nil} \Comment transaction is complete
%		     	 	\State return leader.commit 
%		     	 \EndIf
%		     	 \Comment try to abort pending transaction 
%			\State ok $\leftarrow$ ds.check\&mutate(leader.commit, nil, abort)
%			\If{$\neg$ok} \Comment leader status has changed -- recheck  
%				\State return  {\sc checkLeader}({leaderPtr}) \EndIf
%			\State return abort
%	   	\EndProcedure


\Procedure{commit}{$ts_r$, write-set}
      	\State $ts_c \leftarrow$ TM.commit($ts_r$, write-set) \Comment may return abort
      	\If{$ts_c \not=$ abort}
	      	\If {CT.check\&mutate($ts_r$, commit, nil, $ts_c$) $=\bot$} 
   		 	\State $ts_c \leftarrow$ abort  %\Comment transaction forcefully aborted
   		 \EndIf
	\EndIf

	\ForAll{keys k $\in$ write-set}
			\If{$ts_c =$ abort} ds.remove(k, $ts_r$)  	
%			\Else\ update  commit field of \tuple{k, $ts_r$} to $ts_c$  in ds%  $\leftarrow ts_c$ 
			{\Else\ ds.put(k, $ts_r$, commit, $ts_c$)}
			\EndIf
	\EndFor
	\State CT.remove($ts_r$) \Comment garbage-collect CT entry
\EndProcedure
      
  \end{algorithmic}
  \caption{\sysll's client-side operations.} 
  %for transaction with read timestamp $ts_r$.}
  \label{fig:get-pseudocode}
\end{algorithm} 

%\clearpage



\mypara{Begin.}
The client sends a begin request to the TM, which assigns it a read timestamp $ts_r$.

\mypara{Write.}
 During a transaction, a write operation   \emph{tentatively} writes a new value to an object with a certain version number.
The version is the transaction's $ts_r$, which exceeds all versions written by transactions that committed before it  began. 
A dedicated \emph{commit} column is initialized to nil to indicate that the write is still tentative (not committed); 
after the transaction commits successfully, the value of the commit field is set to the writing transaction's commit timestamp.
The write further tracks key in its local write-set.

\mypara{Read.}
The algorithm traverses data  records (using the data store's ds.get API) pertaining
to the requested key with a version that does not exceed $ts_r$, latest to earliest, and returns the first value that is committed
with a  version smaller than or equal to $ts_r$. To this end, it needs to discover the commit timestamp, $ts_c$, associated with
each data record it considers. 

If the record is tentative (commit=nil), then we do not know whether the transaction that wrote it is still pending. 
%or simply not committed. 
Since read cannot return without determining its final commit timestamp, 
it must forcefully abort it in case it is still pending, and otherwise discover its outcome.
To this end, read first refers to the CT.
If there is no CT entry associated with the transaction ($ts_c$=nil), 
read attempts to create an entry with $\bot$ as the commit timestamp. 
This is done using an atomic {check\&mutate} operation, due to a possible 
race with a commit operation. %; if a competing commit attempt finds the   $\bot$ entry, it  aborts. 
%Note that it is possible  to wait a configurable amount of time before attempting to create a $\bot$ entry, in
%order to allow the transaction complete without aborting it.

There is  another subtle race to consider in case the commit record is set to $\bot$ by a read operation. 
Consider a slow read operation that reads the data record rec when its writing transaction is still pending, and then stalls
for a while. In the interim, the writing transaction successfully commits, updates the data record, 
and finally garbage-collects its  CT entry.  At this point the slow reader wakes up 
and does not find the commit entry in the CT. It then creates a $\bot$ entry even though the transaction had successfully committed.
To discover this race, we have a read operation that either creates or finds a $\bot$ entry in the CT re-read the data record.
%, and  if it does contain a commit entry, use it. 

\mypara{Commit.}
The client first 
\remove{flushes all its writes to the data store and then} 
sends a commit request to the TM. 
If the TM detects conflicts then the transaction aborts, and otherwise the TM provides the transaction with its commit  timestamp $ts_c$. 
\remove{Flushing before calling the TM ensures that transactions with read timestamps exceeding $ts_c$ 
will find the committing transaction's records in the data store.}
The client then proceeds to commit the transaction, provided that no read had forcefully aborted  it. To ensure the latter, the client uses 
an atomic {check\&mutate} to create the commit entry. 

To avoid an extra read of the CT on every transactional read, once a transaction is committed, it writes 
the transaction's $ts_c$ to the commit columns of all keys in  the transaction's write-set. 
In case of an abort, it removes the pending records.
Finally, it removes the transaction's CT entry. 




\subsection{Performance evaluation}
\label{ssec:ll-graphs} 

We compare \sysll's performance to that of Omid. 
Our experiment testbed consists of nine 12-core Intel Xeon 5 machines with 46GB RAM and 4TB 
SSD storage, interconnected by 10G Ethernet. We allocate three of these to HBase nodes, 
one to the TM, one to emulate the client whose performance we measure, and four more to simulate 
background traffic.  
Our test cluster stores approximately 23M keys ($\sim\!\!7$M keys per node). 
The values are 2K big, yielding roughly 46GB of actual data, replicated three-way in HDFS. The keys are hash-partitioned
across the servers. The data accesses are 50\% reads and 
50\% writes. The key access frequencies follow a Zipf distribution
%, generated following 
%the description in~\cite{Gray:1994:QGB:191839.191886}, with $\theta=0.8$, which yields the abort rate reported
%in the production deployment of Omid~\cite{Omid2017}. In this distribution, the first key is accessed 0.7\% of the  time.
%Transaction sizes (number of reads and writes) follow a Zipf distribution with $\theta=0.99$, with a cutoff at $10$. 
%With these parameters, $63\%$ of the transactions access three keys or less, and only $3\%$ access $10$ keys. 
%To make the comparison meaningful, we study transactions of different lengths  separately. 

Figure~\ref{fig:throughput-latency} presents the average latency of single-key and ten-operation transactions as a function of {system} throughput.
Additional experiments are reported in~\cite{OmidVLDB}.  


\begin{figure*}[htb]

\centering
\begin{tabular}{cc}

    \begin{subfigure}[t]{0.3\textwidth}
      	\includegraphics[width=\textwidth]{figs/throughputlatency1.pdf}
	    \caption[]{\small Transaction size = 1}
        \label{fig:tl-1}      
   \end{subfigure}  
& 
    \begin{subfigure}[t]{0.3\textwidth}
	\includegraphics[width=\textwidth]{figs/throughputlatency10.pdf}
	\caption[]{\small Transaction size=10}
    \label{fig:tl-10}
  \end{subfigure}  
 \end{tabular} 
 
  \caption{\small Latency vs.\ throughput  in Omid and \sysll. \inred{Yoni: please create the graphs with only Omid and Omid-LL, thanks!} }
  \vspace{-0.3cm}
  \label{fig:throughput-latency}
\end{figure*}





\section{Fast-Path Transactions}
\label{sec:alg}

The goal of our fast path is to forgo the overhead associated with 
communicating with 
the TM to begin and commit transactions. This is particularly important for short transactions, where the begin and commit overhead is not amortized
across many operations.
We therefore focus on single-key transactions.

To this end, we introduce in Section~\ref{ssec:fast-api} a streamlined  \emph{fast path (FP)}
API that jointly executes multiple API calls of the original TPS.
We proceed, in Section~\ref{ssec:fast-algorithm}, to explain a high-level general fast path algorithm 
{for any system that follows the generic schema of  Algorithm~\ref{alg:schema} above}. 
Finally, in Section~\ref{ssec:fast-impl}, we describe our implementation of the fast path in \sys, and 
important practical optimizations we applied in this context.
 
\subsection{API}
\label{ssec:fast-api}

%\paragraph{API.}
For brevity, we refer to the TPS's API calls  begin, read, write, and commit as \code{b, r, w}, and \code{c} respectively, and 
we combine them to allow fast processing.
The basic FP transactions are singletons, i.e., transactions that perform a single
read or write. These are supported by the APIs: 
\begin{description}
\item[brc(key)] -- begins an FP transaction, reads key within it, and commits.
\item[bwc(key,val)] -- begins an FP transaction,  writes val into a new version of key that exceeds all existing ones, and commits.
\end{description}

We further support a fast path transaction consisting of a read and a dependent write, via a pair of API calls:
\begin{description}
\item[br(key)] -- begins an FP transaction and  reads the latest version of key. 
Returns the read value along with a version ver.
\item[wc(ver, key,val)] -- 	validates that key has not been written since the  \code{br} call that returned ver, writes val into a new version of key, and commits.
\end{description}

Read-only transactions never abort, but \code{bwc} and \code{wc} may abort. 
If an FP transaction aborts, it can be retried either via the fast path again, or as a regular transaction.

\remove{
We note that though the FP API richness can potentially complicate development, this drawback may be abstracted 
away by high-level access semantics. For example, a SQL database (e.g., Phoenix~\cite{Phoenix})
can select the most appropriate API in its query optimizer, transparently to the user.  
}

\remove{
A more elaborate example is a read-modify-write API: 
\begin{description}
\item[brwc(key,f)] -- begins an FP transaction,  reads the latest version of key, applies $f$ to it (on the server side), 
	writes the result into a new version of key that exceeds all existing ones, and commits.
\end{description}

To use the above API, the programmer has to encapsulate the transaction logic in a function for server-side processing. 
Alternatively, we allow FP transactions to unfold dynamically much like regular  transactions do.
A dynamic FP transaction may instead begin with a \code{br} call, perform client-side processing, and then call the following
function to update either the same or a different key: 

\begin{description}
\item[wc(key,val)] -- writes val into a new version of key that exceeds all existing ones, and commits.
\end{description}

Moreover, we do not restrict FP transactions to perform a single read -- any number of \code{r}'s may be called between the \code{br} 
and \code{wc}. The supported types of FP transactions are summarized in Table~\ref{table:fp-types}.
Note, however, that all calls must be directed at the same region, else the transaction is not local.
In case an FP transaction dynamically discovers that it needs to access additional regions, it is aborted and should be restarted as a regular transaction. 

\begin{table}[htb]
%\def\arraystretch{1.5}%  1 is the default, change whatever you need
\centerline{
\begin{tabular}{l  @{\hspace{2em}} l}
Call sequence & Transaction type\\
\hline
\code{br} & single read\\
\code{bwc} & single write\\
\code{br, r*} &  multi-read\\
\code{br, r*, wc} & multi-read, single write\\
\code{brwc} & server-side single-read-write\\
%\hline
\end{tabular}
}
\caption{Supported FP transaction types.}
\label{table:fp-types}
\end{table}

In principle, it would have been possible to also allow \code{w} calls in the span of an FP transaction, 
but in this case, it is not possible to forgo the two-phase execution. 
That is, the \code{w} calls would need to indicate write intents, and  to be atomically committed (or aborted) during the final \code{wc} 
(or \code{c}) call. 
Given the limited benefit and extra complexity of allowing many writes in FP transactions, we do not support this option in our solution.

}


\subsection{Generic fast path algorithm}
\label{ssec:fast-algorithm}



\begin{algorithm}[htb]
%\small
%\hspace{10mm}\\

\underline{Client-side logic:}
\begin{algorithmic}[1]
%\small
\Procedure{brc}{key} \label{l:brc}
\State rec  $\leftarrow$ ds.get(last committed record of key)  \Comment no preGet
\State  return rec.value
\EndProcedure
% \Statex
\Procedure{bwc}{key, value} 
%	\State old $\leftarrow$ ds.get(last version of key)
%	\If{old.commit = nil}  \Comment  tentative value -- abort 
%		\State return abort 
%	\EndIf
	\State return {ds.putVersion}($\infty$, key, value)
\EndProcedure
% \Statex
\Procedure{br}{key} 
\State rec  $\leftarrow$ ds.get(last committed record of key) \Comment no preGet
\State  return \tuple{rec.version, rec.value}
\EndProcedure
% \Statex
\Procedure{wc}{ver, key, value} 
\State return {ds.putVersion}(ver, key, value)
\EndProcedure
\Statex
\algstore{fp}
\end{algorithmic}

\underline{Data store (stored procedure) logic:}
\begin{algorithmic}[1]
\algrestore{fp}
%\small
\Procedure{putVersion}{old, key, value} 
\Statex \Comment used by FP writes (\code{bwc} and \code{wc})
%\Atomic 
	\If{key has no tentative version $\wedge$\\
		\hspace{7mm} last committed version of key $\le$ old}
		\State ver $\leftarrow$ F\&I(key's maxVersion) $+1$   \label{l:fi}
		\State ds.put(key, value, ver, ver) 	\Comment no prePut	%  \tuple{key, ver} with commit = ver  
		\State return commit
	\Else 
		\State return abort 
	\EndIf
%\EndAtomic
\EndProcedure

\Statex
\Procedure{preGet}{$ts_r$, key}   \label{l:txr}
\Statex \Comment executed atomically with ds.get call, once per key 
\State bump(key's maxVersion, $ts_r$)
%\State return get(key)
\EndProcedure
% \Statex
\Procedure{prePut}{key, val, $ts_r$, nil}   \label{l:txw}
 \Statex \Comment checked atomically before put (by transactional write)
 \If{key has a committed version that exceeds $ts_r$} 
 	\State abort 
 \EndIf
\EndProcedure

\Procedure{preUpdate}{key, \inred{$ts_r$}, commit, $ts_c$}   \label{l:txu}
\Statex \Comment executed atomically with ds.update call (by  post-commit) 
\State bump(key's maxVersion, $ts_c$)
\EndProcedure
\end{algorithmic}
\caption{Generic support for FP transactions; each data store operation is executed atomically.}
\label{alg:fp}
\end{algorithm}

The generic fast  path algorithm, which we present in Algorithm~\ref{alg:fp}, consists of two parts: 
client side-logic, and logic running within the underlying data store. 
The latter is implemented as a stored procedure, %%%%%%%%. 
Specifically, it supports a new flavor of put, \emph{putVersion}, which is used by singleton writes, 
and it extends the put, get, and update APIs used by regular transactions with additional logic.  
The new logic is presented in the \emph{preGet}, \emph{prePut}, and \emph{preUpdate} procedures, 
which are  executed before, and atomically with, the corresponding data store calls.

Singleton reads (line~\ref{l:brc}) simply return the value associated with the  latest committed version of the requested key they encounter.  
They ignore tentative versions, which may lead to missing the latest commit in case its post-commit did not complete, 
but is allowed by our semantics. 
FP reads can forgo the begin call since they do not need to obtain a snapshot time a priori. 
They can also forgo the commit call, since they perform a single read, and hence their `snapshot' is trivially valid.

In case  \code{bwc} encounters a tentative version, it does not try to resolve it, but rather simply aborts.
This may cause false aborts in case the transaction that wrote the tentative version has committed and did not 
complete post-commit, as well as in the case that it will eventually abort. 
In general, this mechanism prioritizes regular transactions over FP ones. We choose this approach since
the goal is to complete FP transactions quickly, and if an FP transaction cannot complete quickly, it might as well be 
retried as a regular one.
   
Such a singleton write has two additional concerns: (1) it needs to  produce a new version number that exceeds all committed ones and
is smaller than any commit timestamp that will be assigned to a regular transaction in the future.
(2)  It needs to make sure that conflicts with regular transactions are detected. 

To handle these concerns,  
%in a way similar to Mediator~\cite{mediator}. Namely, 
we maintain the timestamps as two-component structures, consisting of a global version and a locally advancing sequence number.
In practice, we implement the two components in one long integer, with some number $\ell$ least significant bits
reserved for sequence numbers assigned by FP writes (in our implementation, $\ell=20$).
The most significant bits represent the global version set by the TM. The latter increases 
the global clock by $2^\ell$ upon every begin and commit request.

To support (1) a {\code{bwc}} transaction reads the object's latest committed version and increments it. 
%This is different from~\cite{mediator} which aggregates multiple timestamps in a local clock and more like~\cite{cockroach} which reads and manipulates versions at the object's level.
The increment is done provided that it does not overflow the sequence number. 
In the rare case when the lower $\ell$ bits are all ones, the global clock must be incremented, and so the FP transaction aborts and is retried as a regular one. 

It is important to note that the singleton write needs to \emph{atomically} find the latest version and produce a new one that exceeds it, 
to make sure that no other transaction creates a newer version in the interim. This is done by a new \emph{putVersion} function implemented in  code that resides at the data store level. In Section~\ref{ssec:fast-impl}
below, we explain how we implement such atomic conditional updates in an HBase {coprocessor} as part of \sys. 
The first parameter to putVersion is an upper bound on the key's current committed version; since a singleton write
imposes no constraints on the object's current version, its upper bound is $\infty$.

Next, we address (2) -- conflicts between FP and regular transactions.
%(Note that the atomic conditional write in \code{bwc} takes care of conflicts among singletons.)
In case an ongoing regular transaction writes to a key before \code{bwc} accesses it, 
\code{bwc} finds the tentative write and aborts. 

It therefore remains to consider the case that
a regular transaction $T_1$ writes to some key after FP transaction $FP_1$, but $T_1$ must abort because
%the key is updated between its snapshot time and update time. 
it reads the old version of the key before $FP_1$'s update. This scenario is illustrated in Figure~\ref{fig:why-bump}. 
Note that in this scenario it is not possible to move $FP_1$ to `the past' because of the read.

\begin{figure}[htb]
\includegraphics[width=\columnwidth]{figs/FP-why-bump}
\caption{Conflict between FP transaction $FP_1$ and regular transaction $T_1$.}
\label{fig:why-bump}
\end{figure}

In order for $T_1$ to detect this 
conflict, the version written by $FP_1$ has to exceed $T_1$'s snapshot time, i.e., $ts_r$.
To this end, we maintain a new field \emph{maxVersion} for each key, which is at least as 
high as the key's latest committed version. 
The data store needs to support two atomic operations for updating {maxVersion}.
The first is \emph{fetch-and-increment, F\&I}, which increments {maxVersion} and returns its
old value; F\&I throws an abort exception in case of wrap-around of the 
sequence number part  of  the version. 
The second operation, \emph{bump}, takes a new version  as a parameter and
sets  {maxVersion} to the maximum between this parameter and its old value.

Singleton writes  increment the version using F\&I  (line~\ref{l:fi}), and  
the post-commit of transactional writes  (line~\ref{l:txu}) bumps it 
to reflect the highest committed version.
Every transactional read bumps the key's {maxVersion}
to the reading transaction's $ts_r$  (line~\ref{l:txr}); 
transactional writes (line~\ref{l:txw}) are modified to check for conflicts, namely, 
new committed versions exceeding their $ts_r$.

In the example of Figure~\ref{fig:why-bump}, $T_1$'s read bumps $x$'s {maxVersion} to its $ts_r$, 
and so $FP_1$, which increments $x$'s maxVersion, writes $1$ with a version that exceeds $ts_r$.
Thus, $T_1$'s write  detects the conflict on $x$. 

 Note that this modification of transactional writes incurs an extra cost on
 regular (non-FP) transactions, which we quantify empirically in Section~\ref{sec:eval}.  

The \code{br} and \code{wc} operations are  similar to \code{brc} and \code{bwc}, 
respectively, except that \code{wc} uses the version read by \code{br} as its upper bound
in order to detect conflicting writes that occur between its two calls.

\subsection{Implementation and optimization}
\label{ssec:fast-impl}

Associating a maxVersion field with each key is wasteful,
both in terms of space, and in terms of the number of updates this field undergoes.
Instead, when implementing support for \sys's fast path in HBase, we aggregate the maxVersions of many keys in a single variable, 
which we call the \emph{Local Version Clock (LVC)}.
%; (similar local clocks were previously used in Mediator~\cite{mediator} and CockroachDB~\cite{cockroach}).

Our implementation \inred{uses} one LVC in each region server. 
Using a shared LVC reduces the number of updates:  
a transactional read modifies the LVC only if its $ts_r$ exceeds it. In particular, 
a transaction with multiple reads in the same region server needs to bump it only once. 

We implement the two required atomic methods - F\&I and bump on the LVC using atomic hardware operations (F\&I and CAS, respectively). 
The HBase coprocessor mechanism enforces atomic execution of the stored code blocks by holding  a lock on the affected key for the duration of the operation.
Thus, putVersion executes as an atomic block, and calls the LVC's F\&I method inside this block.
Similarly, the calls to bump from preGet and preUpdate execute inside an atomic block with the ensuing get an update, respectively. 

Note that although the coprocessor only holds a lock on the affected key, the joint update of the key and the LVC is consistent
because it uses a form of two-phase locking: when the stored procedure begins, its locks the key, then the atomic access to the LVC
effectively locks the LVC during its update; this is followed by an update of the key's record and the key's lock being released.

The LVC is kept in memory, and is not persisted. Migration of region control across region servers, which HBase performs 
to balance load and handle server crashes, must be handled by the LVC management. 
In both cases, we need to ensure that the monotonicity of the LVC associated with the %(migrated or recovered) 
region is preserved. To this end, when a region is started in a new server (following migration or recovery), 
we force the first operation accessing it
in the new server to accesses the TM to increment the global clock, and then 
bumps the local clock to the value of the global clock.
Since the LVC value can never exceed the global clock's value, this bootstrapping procedure maintains its monotonicity.





\section{Evaluation} \label{sec:eval}


We describe our methodology and experiment setup in Section~\ref{ssec:methodology}, and present our results 
in Section~\ref{ssec:results}.

\subsection{Methodology}
\label{ssec:methodology}

\paragraph{Evaluated systems.}

We compare \sys\  to  state-of-the art TPSs and separately evaluate the effectiveness of its FP mechanism.
To this end, we compare four TPSs, all of which use HBase as the underlying data store:
\begin{description}
\item[Vanilla \sys] -- the algorithm described in Section~\ref{sec:ll}, without support for FP transactions.
\item[\sys] -- single-key transactions use the FP API, 
whereas longer transactions use the standard API, 
and their transactional reads and writes are modified to access the LVC along with the data as described in Section~\ref{sec:alg}.
\item[Omid] -- the Apache Incubator version of Omid, on which we base our implementation of \sys. 
\item[2PC] -- an implementation (in the \sys\ framework) of Percolator's concurrency control mechanism:
Begins and commits use the TM only for timestamp allocation and not for conflict detection; conflicts are detected via 
\emph{Two Phase Commit (2PC)}.
\end{description}

Direct comparison to CockroachDB is not feasible because CockroachDB supports SQL transactions over a multi-tier
architecture using various components that are incompatible with HBase. 
We do no compare \sys\ to Omid1 (and the similarly-designed Tephra), since Omid significantly
outperforms Omid1~\cite{Omid2017}.

To reduce latency, we configure the TPSs to perform the post-commit phase asynchronously, 
by a separate thread, after the transaction completes.

\remove{
In Omid and Vanilla \sys, all transactions use the standard API, namely 
delineating transactions with begin and commit calls.
In FP \sys, transactions that can use the FP API (specifically, singleton reads, singleton writes, 
and single-key read-writes) do so, whereas other transactions use the standard API.
}

\paragraph{Experiment setup.}

Our experiment testbed consists of nine 12-core Intel Xeon 5 machines with 46GB RAM and 4TB 
SSD storage, interconnected by 10G Ethernet. We allocate three of these to HBase nodes, 
one to the TM, one to emulate the client whose performance we measure, and four more to simulate 
background traffic as explained below. Each HBase node runs both an HBase region server and 
the underlying Hadoop File System (HDFS) server within 8GB JVM containers. 

Note that the only scalability bottleneck in the tested systems is the centralized TM.
HBase, on the other hand, scales horizontally across thousands of nodes, each of which processes a small fraction of the total load. 
Since each node typically serves millions of  keys, data access rates remain load-balanced across nodes 
even when access to individual keys is highly skewed.  And since read/write requests are processed independently by each HBase node, 
their performance remains constant as the workload and  system size are scaled by the same factor. 

Thus, to understand the system's performance at scale, 
we can run transactions over a small HBase cluster with an appropriately-scaled load, 
but need to stress the TM as a bigger deployment would. 
We do this at a ratio of 3:1000; that is, we run transactions on a $3$-node HBase cluster and 
load the TM with a begin/commit request rate that would arise in a $1000$-node HBase cluster with the same per-node load.
For example, to test the client's latency at $100$K tps, we have the TM handle $100$K tps, and have  an HBase deployment of
three nodes handle read/write requests of $0.3$K tps. As explained above, 
the HBase latency at $100$K tps with $1000$ servers would be roughly the same as in this deployment.

We use four machines to generate the background load on the TM using a custom tool~\cite{Omid2017} 
that asynchronously posts begin and commit requests on the wire and collects the TM's responses. 
We note that although in this experiment the TM maintains a small number of client connections (serving many requests per connection), 
the number in a true $1000$-node system still falls well within the OS limit, hence no real bottleneck is ignored. 

We measure the end-to-end client-incurred latency on a single client node that generates transactions over the $3$-server HBase cluster. 
Note that the client also generates begin/commit requests, which account for \mbox{$\sim0.3\%$} of the TM's load.
The client runs the popular YCSB benchmark~\cite{Cooper:2010:BCS:1807128.1807152}, 
exercising the synchronous transaction processing API in a varying number of threads. 

\paragraph{Workload.}

Our test cluster stores approximately 23M keys ($\sim\!\!7$M keys per node). 
The values are 2K big, yielding roughly 46GB of actual data, replicated three-way in HDFS. The keys are hash-partitioned
across the servers. The data accesses are 50\% reads and 
50\% writes. The key access frequencies follow a Zipf distribution, generated following 
the description in~\cite{Gray:1994:QGB:191839.191886}, with $\theta=0.8$, which we derive from production 
workloads. \inred{In this distribution, the first key is accessed XXX\% of the  time.}
%in Yahoo's deployment of Omid~\cite{Omid2017}. 
%Note that under this access distribution, data requests are load-balanced across nodes.

\noindent
We test the system with two transaction mixes:
\begin{description}
\item[Random mix] -- 
transaction sizes (number of reads and writes) follow a Zipf distribution with $\theta=0.99$, with a cutoff at $10$. 
With these parameters, $63\%$ of the transactions access three keys or less, and only $3\%$ access $10$ keys. 
We vary the system load from 30K to 500K transactions per second (tps). 
\item[BRWC] -- $80\%$ of the transactions are drawn from the random mix distribution, and $20\%$ perform a read
and then a write to the same key. 
\end{description}

We add the BRWC workload since single-key read+write transactions are common in production, but are highly unlikely to 
occur in our random mix, which uses random key selection with billions of keys.

\begin{figure}[htb]
	\centering
      	\includegraphics[width=0.48\textwidth]{figs/throughputlatency1.pdf}
	    \caption{Throughput vs.\ latency, transaction size = 1.}
        \label{fig:tl-1}      
\end{figure}


\begin{figure*}[hbt]
\centering
\begin{tabular}{ccc}
      \begin{subfigure}[t]{0.48\textwidth}
         \includegraphics[width=\textwidth]{figs/latency_allPUTGET.pdf}
        \caption[]{Low load (100K tps).}
        \label{fig:stack-brc}

      \end{subfigure} 
    
& 
      \begin{subfigure}[t]{0.48\textwidth}
      	\includegraphics[width=\textwidth]{figs/latencyHighThrough_PUTGETRMW.pdf}
	\caption{High load (500K tps).}
	\label{fig:hightx}
      \end{subfigure}  & 

\end{tabular}
       \caption{Latency breakdown  for single-key transactions under  random mix workload. }
\end{figure*}


\subsection{Results}
\label{ssec:results}

\paragraph{Throughput and latency of short transactions.} 

Recall that \sys\ is motivated by the prevalence of short transactions in production, and is designed with
the goal of speeding such transactions up.
Its advantage is less pronounced for long transactions, where the cost of begin and commit is amortized
across many reads and writes.
To make the comparison meaningful, we classify transactions by their lengths and whether they
access a single key, and study each transaction class separately. 

We begin with short transactions. 
Figure~\ref{fig:tl-1} presents the average latency of single-key transactions run as part of the random mix,
as a function of {system} throughput.
Figure~\ref{fig:stack-brc}  then zooms in on the latency of such transactions under 
a throughput of 100K tps, and breaks up the different factors contributing to it. 
Figure~\ref{fig:hightx}  presents a similar breakdown under a high load of 500K tps; Omid 
is not included in the latter since it does not sustain such high throughput.

As we can see, under light load, \sys\ improves the latency of Omid by 4x to 5x, even without the fast path.
This is because in Omid, both begin and commit wait for preceding transactions to complete the writes of 
their commit entries; this stems from Omid's design choice to avoid the need for resolving pending write intents
by aborting transactions; see last (rightmost) column in Table~\ref{table:design-space}. 
Single-key writes suffer from both the begin and commit latencies, whereas single-key reads  
suffer only from begins (Figure~\ref{fig:stack-brc}). 

As load increases, Omid suffers from a well-pronounced bottleneck, and its latency at 250K tps is doubled, where \sys's 
latency is unaffected. The extra delay in Omid is due to batching of commit record updates, 
which its TM applies to handle congestion~\cite{Omid2017}. 
\sys, in turn, begins to experience a bottleneck (due to a heavy request load on the TM)  at 400K tps.
Such congestion arises due to Omid's centralized 
commit entry management (penultimate column in Table~\ref{table:design-space}).

The FP API delivers better performance for this traffic. For instance, under low load (Figure~\ref{fig:stack-brc}),
single writes take an average of 2.4ms using 
the {\code bwc} API versus 5.65ms using the regular transaction API (consisting of begin, write, and commit). 
For comparison, a native HBase write takes roughly 2ms under this load.
A single read executed using {\code brc} takes 1.5ms, which is the average latency of a native HBase read,
versus 2.5ms as a regular transaction. 
For transactions that read and write a single key as part of the BRWC workload, 
%Their latency breakdown under a 100K tps throughput is shown in Figure~\ref{fig:rmw}.
the fast path implementation (consisting of \code{br} and \code{wc} calls) completes within 3.9ms,
versus 6.5ms for Vanilla \sys\ and 37.9ms for Omid. 
%
Under high load (Figure~\ref{fig:hightx}), the fast path is more beneficial: it reduces the latency of write
by a factor of 2.7, from 6.9ms  to 2.5ms, and the read latency by a factor of 2.2, from  3.65ms to 1.6ms.
%, read, and brwc from 11.24ms, 8.24ms, and 11.85ms to 2.4ms, 1.29ms, and 4.19ms respectively. 

As for 2PC, the performance is same as \sys\ for throughput up to $500K$ tps. Above $500K$ tps the centrlized conflict analysis becomes a bottlneck and 2PC has a minor speedup of 1.1x over \sys.

\paragraph{Long transactions.} 
We now examine longer transactions run as part of the random mix workload.
Figure~\ref{fig:throughput-latency} shows the results for transactions of lengths $5$ and $10$.
We see that the latency gap of \sys\ over Omid remains similar, but is amortized 
by other operations. Omid's control requests (begin and commit) continue to 
dominate the delay, and comprise $61\%$ of the latency of 10-access transactions.
In contrast, \sys's transaction latency is dominated by data access. For example, in 10-operation transactions only 
$17\%$ of the time is spent on the control path, which leads to much faster completion. 



\begin{figure*}[t]
\centering{
\begin{tabular}{cc}

    \begin{subfigure}[t]{0.48\textwidth}
	\includegraphics[width=\textwidth]{figs/throughputlatency10.pdf}
	\caption[]{Throughput versus latency, transaction size = 10}
    \label{fig:tl-10}
  \end{subfigure} & 


  \begin{subfigure}[t]{0.48\textwidth}
	\includegraphics[width=\textwidth]{figs/latency_5_10.pdf}
	\caption[]{Latency breakdown, transaction size = 5, 10}
    \label{fig:stack-tx10}
  \end{subfigure} 
\end{tabular}  	
}		
  \caption{Latency vs.\ throughput  and latency breakdown  for long transactions in random mix workload. }
  \label{fig:throughput-latency}
\end{figure*}

Nevertheless,
the FP mechanism takes its toll on the data path, which resorts to atomic check\&mutate operations 
instead of simple writes. This is exacerbated for long transactions. 
For example, a 10-access transaction takes 24.7ms with FP \sys, 
versus 21.6ms with Vanilla \sys. The performace of 2PC with long transactions is the same as FP \sys\ because it also uses check\&mutate operations during the voting phase of a commit.



\begin{figure}[h!]
\centering
\begin{subfigure}[t]{0.48\textwidth}
\centerline{
\includegraphics[width=.9\textwidth]{figs/low_speedup.pdf}
}
\caption{Low load (100 tps)} 
\label{fig:slowdown-low}
\end{subfigure} 
\begin{subfigure}[t]{0.48\textwidth}
\centerline{
\includegraphics[width=.9\textwidth]{figs/high_speedup.pdf}
}
\caption{High load (500 tps)} 
\label{fig:slowdown-high}
\end{subfigure} 
\caption{Latency speedup with  fast path API in {\sys}.}
\label{fig:fp-tradeoff}
\end{figure}



Figure~\ref{fig:fp-tradeoff} summarizes the tradeoffs entailed by the fast path API
(relative to Vanilla \sys) for the different transaction classes. 
We see that under low load (Figure~\ref{fig:slowdown-low}),
the speedup for single-write transactions is 2.3x, whereas the worst slowdown is $13\%$. 
In systems geared towards real-time processing, this is a reasonable tradeoff, since long transactions 
are infrequent and less sensitive to extra delay.
Under high load (Figure~\ref{fig:slowdown-high}), the fast path is clearly advantageous, most notably, the speedup for writes exceeds $170\%$.
%In systems geared towards real-time processing, this is a reasonable tradeoff, since long transactions 
%are infrequent and less sensitive to extra delay. Under the studied workload, e.g.,   FP \sys\/ is more desirable. 



\paragraph{Abort rates.}

We note that \sys\/ yields slightly higher rates of transaction aborts compared to Omid 
(recall that Vanilla \sys\/ aborts tentative writes in favor of concurrent reads, whereas FP \sys\/ also aborts
singleton writes in presence of concurrent tentative writes). However, the abort rates exhibited by all  
the systems are minor. Here, under the highest contention, FP \sys\/ aborts approximately $0.1\%$ 
of the transactions vs Vanilla \sys's $0.08\%$ and Omid's $0.07\%$ 
(the latter  is in line with~\cite{Omid2017}).  
\remove{
}


\section{SQL Oriented Features} \label{sec:sql}
{\inred{

Comprehensive SQL support in Phoenix required multiple extensions to Omid. 

{\bf Indexes. \/} A secondary index in SQL is an auxiliary table that provides fast access to data in a table 
by a key that is different from the table's primary key. This need often emerges in analytics scenarios, in
which data is accessed by multiple dimensions. Typically, the secondary key serves as the index's primary;
it is associated with the unique reference into the base table (e.g., primary key + timestamp). SQL query 
optimizers exploit secondary indexes in order to produce efficient query execution plans. Query speed 
is therefore at odds with update speed since every write to a table triggers writes to all its indexes. 

The SQL standard allows creating indexes on demand. When the user issues the {\sc {Create Index}} 
command, the database (1) populates the new index table with historic data from the base table, and
(2) installs a trigger to augment every new write to the base table with a write to the index table. 
It is desirable to allow temporal overlap between the two, in order to avoid stalling the writes while 
the index is being populated. 

Transaction managers that provide SI consistency offer a simple mechanism for doing so. Index 
population is a transaction which scans a snapshot of the base table and streams the data into the  
index table. This way, historic data is captured without blocking the concurrent puts. Once this 
process completes, the index can become available to queries. The index update trigger, which is 
created is parallel with the bulk population, is also a transaction, which guarantees the atomicity of 
all updates. 

In order to guarantee the new index's consistency with the base table, the snapshot creation 
and the update trigger setup must be atomic. In other words, all writes beyond the snapshot   
timestamp must be handled by the trigger. Omid achieve this through a new {\em fence\/} API
implemented by the TM. Namely, the trigger installation is associated with a fence timestamp 
allocated upon the call to fence. The TM aborts every transaction that begins before the fence 
timestamp and tries to commit after it. Note that fence timestamps do not burden the TM state 
significantly: every fence is retired after being installed beyond the maximal transaction lifetime. 

Since the TM detects all write conflicts at the base table level, there is no need to embed the affected 
secondary index keys in the commit request. Hence, the load on the TM does not increase. Omid enables 
this optimization by allowing selective inclusion of keys in the working set. 

Note that the above mechanism is not unique for indexes. It applies to all types of derived data, 
e.g., materialized views. 

{\bf Multiple read points. \/}

%{\sc Insert into T (id, grade, date) \\  Select (id, grade, ?today) \\ From T}

{\bf Scan performance. \/} In many cases, Phoenix pushes computation close to data, in order
to speed up query evaluation.

}

\section{Related Work} \label{sec:related}


The past few years have seen a growing interest in distributed 
transaction management~\cite{PattersonENAA12,Cowling2012,Aguilera2015,Balakrishnan2013,Thomson2012,eyal2013ordering,Warp}.
Recently, many efforts have been dedicated to improving performance using advanced 
hardware trends like RDMA and HTM~\cite{Wei2015,Dragojevic2014,Dragojevic2015}.  
These efforts are, by and large, orthogonal to ours.

Our work follows the line of industrial production systems, such as 
Google's Spanner~\cite{Spanner2012}, Megastore~\cite{Megastore}, and Percolator~\cite{Percolator2010}, 
Yahoo's Omid1~\cite{OmidICDE2014} and Omid~\cite{Omid2017}, 
Cask's Tephra\cite{tephra}, and more~\cite{cockroach}.
These systems develop transaction processing engines on top of existing persistent 
highly-available data stores; for example, Megastore is layered on top of
Bigtable~\cite{Chang2008}, Warp~\cite{Warp} uses HyperDex~\cite{Escriva2012}, 
and CockroachDB~\cite{cockroach} uses RocksDB~\cite{rocksdb}.
Like  Omid~\cite{Omid2017}, we layer \sys\ atop Apache HBase~\cite{hbase}.

As discussed in Section~\ref{sec:ll}, a number of these systems follow a common paradigm
with different design choices, and \sys\ chooses a new operation point in 
this design space. In particular, \sys\ eliminates the bottleneck of Omid by
distributing the commit entry, makes commits and begins faster than Omid's by 
allowing reads to abort pending transactions, but unlike Percolator and CockroadDB, 
uses centralized conflict detection. This eliminates the need to detect client 
failures as in Percolator or have transactional writes perform conflict detection 
as in CockroachDB. 

As a separate contribution we developed a fast path for single-key transactions,
which is applicable to any of the aforementioned systems. A similar mechanism 
was previously developed in Mediator~\cite{mediator}. The latter focused on reconciling 
transactions with native atomic operations, rather than on a unified FP API, and suggested
a less efficient implementation.  

\remove{

Omid most closely resembles Tephra [6] and
Omid1 [25], which also run on top of a distributed keyvalue
store and leverage a centralized TM (sometimes
called oracle) for timestamp allocation and conflict resolution.
However, Omid1 and Tephra store all the information
about committed and aborted transactions in the
TM�s RAM, and proactively duplicate it to every client
that begins a transaction (in order to allow the client
to determine locally which non-committed data should
be excluded from its reads). This approach is not scalable,
as the information sent to clients can consist of
many megabytes. Omid avoids such bandwidth overhead
by storing pertinent information in a metadata table
that clients can access as needed. Our performance
measurements in Section 7 below show that Omid significantly
out-performs Omid1, whose design is very close
to Tephra�s. For high availability, Tephra and Omid1
use a write-ahead log, which entails long recovery times
for replaying the log; Omid, instead, reuses the inherent
availability of the underlying key-value store, and hence
recovers quickly from failures.
Percolator also uses a centralized �oracle� for timestamp
allocation but resolves conflicts via two-phase commit,
whereby clients lock database records rendering
them inaccessible to other transactions; the Percolator
paper does not discuss high availability. Other systems
like Spanner and CockroachDB allot globally increasing
timestamps using a (somewhat) synchronized clock
service. Spanner also uses two-phase commit whereas
CockroachDB uses distributed conflict resolution where
read-only transactions can cause concurrent update transactions
to abort. In contrast, Omid never locks (or prevents
access to) a database record, and never aborts due
to conflicts with read-only transactions.
The use cases production systems serve allow them
to provide SI [31, 25, 6, 5], at least for read-only transactions
[17]. It is nevertheless straightforward to extend
Omid to provide serializability, similarly to a serializable
extension of Omid1 [35] and Spanner [17]; it is merely
a matter of extending the conflict analysis to cover readsets
[24, 14], which may degrade performance.
A number of other recent efforts avoid the complexity
of two-phase commit [26] by serializing transactions using
a global serialization service such as highly-available
log [11, 23, 13] or totally-ordered multicast [15]. Omid
is unique in utilizing a single transaction manager to resolve
conflicts in a scalable way.

}

\section{Conclusion} \label{sec:conclusions}

As transaction processing services begin to be used in 
new application domains, low transaction latency and rich SQL semantics
become important considerations. In addition, public cloud deployments 
necessitate solutions compatible with multi-tenancy.
Motivated by such use cases we evolved Omid for cloud use.

As part of this evolution, we
improved Omid's protocol to reduce latency (by 4x to 5x under light load
and an order of magnitude under heavy load) and improve throughput scalability.
%
We further designed a generic \emph{fast path} for single-key transactions, 
which executes them  almost as fast as native HBase operations (in terms of 
both throughput and latency), while preserving
transactional semantics relative to longer transactions.
Our fast path algorithm is not Omid-specific, and can be similarly supported in 
other transaction management systems.
Our implementation of the fast path in \sys\ can process single-key
transactions  at a virtually unbounded rate (thanks to HBase's horizontal 
scalability), and improves the latency of short transactions
by another 3x--5x under high load.

We have extended Omid with functionalities required in SQL engines, namely 
secondary index construction and multi-snapshot semantics. 
We have successfully integrated it into the Apache Phoenix translytics engine. 

\subsection*{Acknowledgments}
We thank James Taylor for fruitful discussions. 

\bibliographystyle{acm}

\bibliography{refs}

\end{document}
