\subsection{\sysll\ algorithm} 
\label{ssec:ll}

\begin{figure}[!t]
  \centering
  
  \begin{subfigure}[tb]{\columnwidth}
      \centering\small
      
      
      \begin{tabular}{|c|c|c|c| c| c | c|}
\multicolumn{4}{c}{Data Table}&  \multicolumn{3}{r}{\hspace{1mm} Commit Table}\\
\cline{1-4} \cline{6-7}
key & value & version & commit & & \inred{key} & ֿ\inred{commit} \\
\cline{1-4} \cline{6-7}
$k_1$ & a & 3 & nil & &  &  \\
\cline{6-7} \cline{1-4}
\end{tabular}
    
	\caption[]{Pending transaction}
    \label{fig:model:tentative}
  \end{subfigure}
  
  \begin{subfigure}[t]{\columnwidth}
    \centering\small
    
          \begin{tabular}{|c|c|c|c| c| c | c|}
\multicolumn{4}{c}{Data Table}&  \multicolumn{3}{r}{\hspace{1mm} Commit Table}\\
\cline{1-4} \cline{6-7}
key & value & version & commit & & \inred{key} & ֿ\inred{commit} \\
\cline{1-4} \cline{6-7}
$k_1$ & a & 3 & nil & & 3 & 7 \\
\cline{6-7} \cline{1-4}
\end{tabular}
    	\caption[]{Committed transaction}
    \label{fig:model:committed}
  \end{subfigure}


  \begin{subfigure}[tb]{\columnwidth}
    \centering\small
          \begin{tabular}{|c|c|c|c| c| c | c|}
\multicolumn{4}{c}{Data Table}&  \multicolumn{3}{r}{\hspace{1mm} Commit Table}\\
\cline{1-4} \cline{6-7}
key & value & version & commit & & \inred{key} & ֿ\inred{commit} \\
\cline{1-4} \cline{6-7}
$k_1$ & a & 3 & 7& & 3 & 7 \\
\cline{6-7}
%$k_2$ & b & 3 & nil \\
\cline{1-4}
\end{tabular}
	\caption[]{Post-committed transaction}
    \label{fig:model:postcommit}
  \end{subfigure}

  
  \caption{Evolution of \sysll\ metadata during a transaction. The transaction receives $ts_r=3$ in the begin stage 
  and uses this as the version number for tentative writes. It receives a commit timestamp $ts_c=7$ when committing.}
  \label{fig:model}
\end{figure}



\begin{algorithm}[htb]
\begin{algorithmic}[1]
%\small
\Procedure{begin}{}
\State checkRenew()
\State return Clock.fetchAndIncrement()
\EndProcedure
\Statex
\Procedure{commit}{$txid$, write-set}
\State checkRenew()
\State $ts_c \leftarrow$ Clock.fetchAndIncrement()
\If{conflictDetect($txid$, write-set, $ts_c$) } 
 \State return $ts_c$
 \Else
 \State return {\sc abort}
\EndIf
\EndProcedure
\Statex
%
\Procedure{checkRenew}{} \Comment HA support;  $\delta$ is the lease time
\If{lease $<$ now + \inred{0.2}$\delta$} \label{l:lease-start} \Comment extend lease
 \State renew lease for $\delta$ time \Comment atomic operation
 \If{failed} halt \EndIf 
\EndIf  \label{l:lease-end}
\If{Clock $=$ epoch}  \label{l:epoch-start} \Comment extend epoch
\State epoch $\leftarrow$ Clock +  range
 
\If{$\neg$CAS(maxTS, Clock, epoch) } halt \EndIf 
\EndIf  \label{l:epoch-end}
\EndProcedure

\end{algorithmic}
\caption{\sysll's TM algorithm with HA support.}
\label{alg:ha}
\end{algorithm}

\begin{algorithm}[t]
  \begin{algorithmic}
 %   \small
 \Procedure{begin}{}
       \State  return TM.begin
\EndProcedure
\Statex
  \Procedure{write}{$ts_r$, key, \inred{fields, values}}
       \State  track key in write-set 
       \State \inred{add commit to fields and nil to values}
       \State    \inred{ return ds.put(key, $ts_r$, fields, \inred{values})}
\EndProcedure
%     
\Statex
\Procedure{read}{$ts_r$, key} 
      \For{rec $\leftarrow$ ds.get(key, versions down from $ts_r$)}
% 			\State $ts_c \leftarrow$ {\sc checkStatus}(rec, $ts_r$)     			       
			\State \Comment  set $ts_c$ to the commit timestamp associated with rec
			\If{rec.commit $\not =$nil}  \Comment commit cell exists
     				\State $ts_c \leftarrow$  rec.commit 
			\Else \Comment commit cell is empty, check CT
      				\State $ts_c  \leftarrow$ CT.get(rec.ts)  
      				\If{$ts_c =$ nil}  \Comment no CT entry, set status to  $\bot$
      					\State \inred{$ts_c\leftarrow$CT.check\&mutate(rec.ts,nil,commit,nil,$\bot$)}  
      					\inred{\Comment aborts the transaction if it was not commited}
      					\If{$ts_c$=nil}  %\Comment check\&mutate succeeded
      						\State $ts_c  \leftarrow \bot$ 
      					\EndIf
				\EndIf
      				
 				 \If{$ts_c = \bot$}  
 				 	\State \Comment check for race: commit before $\bot$ entry  created 
 				 	\State re-read rec from ds
 				 	\If{rec.commit $\not =$nil} 
 				 		 \State $ts_c \leftarrow$  rec.commit 
 				 		 \State CT.remove(rec.ts)
 				 	\Else\  
 				 		 continue	\Comment writing transaction aborted	
 				 	\EndIf
				\EndIf
			\EndIf

       			 \If{$ts_c <  ts_r$}  
%     			 \If{$ts_c \not=$ abort  $\wedge\ ts_c <  ts_r$}  
 			 	\State return  rec.value 
			\EndIf
      \EndFor
      \State  return nil \Comment no returnable version found in loop
      \EndProcedure

\Statex

%      		\Procedure{checkLeader}{\tuple{k, ts}} 
%     			\State leader $\leftarrow$ ds.get(\tuple{k, ts})
%    			 \If{leader $=$ nil} return abort \Comment{leader removed} \EndIf 
%		     	 \If{ leader.commit $\not=$nil} \Comment transaction is complete
%		     	 	\State return leader.commit 
%		     	 \EndIf
%		     	 \Comment try to abort pending transaction 
%			\State ok $\leftarrow$ ds.check\&mutate(leader.commit, nil, abort)
%			\If{$\neg$ok} \Comment leader status has changed -- recheck  
%				\State return  {\sc checkLeader}({leaderPtr}) \EndIf
%			\State return abort
%	   	\EndProcedure
\remove{
      		\Procedure{checkStatus}{rec, ts} 
	%	     	 \While{true} \Comment at most two iterations 
			\If{rec.commit $\not =$nil}  
     			 \State return rec.commit \EndIf
      				\State $ts_c  \leftarrow$ CT.get(ts) 
      				\If{$ts_c = nil$}  \Comment no CT entry -- set transaction status to $\bot$
      					\State $ts_c \leftarrow$ CT.putIfAbsent(ts, $\bot$) 
      					 \Comment returns new value
				\EndIf
      				
 				 \If{$ts_c = \bot$  }  
 				 	\State \Comment check for race: committed before $\bot$ entry was created
 				 	\If{rec.commit $\not =$nil}  return rec.commit \EndIf
 				 	\State return abort 				
				\EndIf
%
 %    				 \If{$ts_c =$ nil}  \Comment{abort transaction}  
     				% 	\State ok $\leftarrow$ CT.putIfAbsent(ts, abort)
%					\If{CT.putIfAbsent(ts, abort)}
%						\State return abort	
%					\Else \Comment entry no longer nil, retry
%						\State return {\sc checkStatus}(ts)
%	     				 \EndIf 			
%				\EndIf
%				return abort
%				\Else
%					\State return rec.commit  %\Comment transaction is complete 
%					\State return leader.commit \Comment transaction is complete
%				\EndIf 		     	 	
%		     	 \EndWhile
	   	\EndProcedure
}


\Procedure{commit}{$ts_r$, write-set}
      	\remove{\State ds.flush}
      	\State $ts_c \leftarrow$ TM.commit($ts_r$, write-set) \Comment may return abort
      	\If{$ts_c \not=$ abort}
	      	\inred{\If {CT.check\&mutate($ts_r$, nil, commit, nil, $ts_c$) $=\bot$}} % \Comment Set field only if field is missing and return $\bot$ if transaction $ts_r$ was aborted}
%   		\If{$\neg$ok} 
   		 	\State $ts_c \leftarrow$ abort 
   		 \EndIf
	\EndIf
	\State \Comment post-commit
	\ForAll{keys k $\in$ write-set}
			\If{$ts_c =$ abort} ds.remove(k, $ts_r$)  	
%			\Else\ update  commit field of \tuple{k, $ts_r$} to $ts_c$  in ds%  $\leftarrow ts_c$ 
			\inred{\Else\ ds.put(k, $ts_r$, commit, $ts_c$)}
			\EndIf
	\EndFor
	\State CT.remove($ts_r$) \Comment garbage-collect CT entry
\EndProcedure
      
  \end{algorithmic}
  \caption{\sysll's client-side operations.} 
  %for transaction with read timestamp $ts_r$.}
  \label{fig:get-pseudocode}
\end{algorithm} 

Like Omid, \sysll\ uses a dedicated \emph{commit table (CT)} for storing commit entries.
A transaction is atomically committed by adding to the CT a commit entry mapping its $ts_r$ to its $ts_c$.
The post-commit phase then copies this information to 
a dedicated \emph{commit} column in the data table, in order to spare future reading transactions
 the overhead of checking the commit table. 
 Figure~\ref{fig:model} shows the metadata of a transaction during its stages of execution. 

Whereas Omid's CT is updated by the TM, \sysll\ distributes the CT updates amongst the clients.
Its TM is thus a simplified version of Omid's TM, and appears  in Algorithm~\ref{alg:ha}. It has two roles:
First, it allocates begin and commit timestamps by fetching-and-incrementing a monotonically increasing global clock.
Second, upon commit, it calls the \emph{conflictDetect} function, which 
performs validation (i.e., conflict detection) using an in-memory hash table
by checking, for each key in the write-set, that its commit timestamp in the hash-table is smaller than the 
committing transaction's $ts_r$. If there are no conflicts, it 
updates the hash table with the write-set of the new transaction and its $ts_c$. 
%(For more details see~\cite{Omid2017}).
The  \emph{checkRenew} procedure supports the TM's  high availability, and is explained at the end of this section.



\noindent
Client operations proceed as follows (cf.~Algorithm~\ref{fig:get-pseudocode}):

\mypara{Begin.}
The client sends a begin request to the TM, which assigns it a read timestamp $ts_r$.

\mypara{Write.}
The client uses the data store's update API to add a tentative record to the data store, with the written key and value, 
version number $ts_r$, and nil in the commit column.
It also tracks key in its local write-set.

\mypara{Read.}
The algorithm traverses data  records (using the data store's ds.get API) pertaining
to the requested key with a version that does not exceed $ts_r$, latest to earliest, and returns the first value that is committed
with a  version smaller than or equal to $ts_r$. To this end, it needs to discover the commit timestamp, $ts_c$, associated with
each data record it considers. 

If the commit cell of the record is empty (commit=nil), then we do not know whether the transaction that wrote it is still pending 
or simply not post-committed. Since read cannot return without determining its final commit timestamp, 
it must forcefully abort it in case it is still pending, and otherwise discover its outcome.
To this end, read first refers to the CT.
If there is no CT entry associated with the transaction ($ts_c$=nil), 
read attempts to create an entry with $\bot$ as the commit timestamp. 
This is done using an atomic \inred{check\&mutate} operation, due to a possible 
race with a commit operation; if a competing commit attempt finds the   $\bot$ entry, it  aborts. 
Note that it is possible  to wait a configurable amount of time before attempting to create a $\bot$ entry, in
order to allow the transaction complete without aborting it.

There is, however, a subtle race to consider in case the commit record is set to $\bot$ by a read operation. 
Consider a slow read operation that reads the data record rec when its writing transaction is still pending, and then stalls
for a while. In the interim, the writing transaction successfully commits, updates the data record during post-commit, 
and finally garbage-collects its  CT entry.  At this point the slow reader wakes up 
and does not find the commit entry in the CT. It then creates a $\bot$ entry even though the transaction had successfully committed.
To discover this race, we have a read operation that either creates or finds a $\bot$ entry in the CT re-read the data record, and 
if it does contain a commit entry, use it. 

\mypara{Commit.}
The client first 
\remove{flushes all its writes to the data store and then} 
sends a commit request to the TM. 
If the TM detects conflicts then the transaction aborts, and otherwise the TM provides the transaction with its commit  timestamp $ts_c$. 
\remove{Flushing before calling the TM ensures that transactions with read timestamps exceeding $ts_c$ 
will find the committing transaction's records in the data store.}
The client then proceeds to commit the transaction, provided that no read had forcefully aborted  it. To ensure the latter, the client uses 
an atomic \inred{check\&mutate} to create the commit entry. 

To avoid an extra read of the CT on every transactional read, once a transaction is committed, the post-commit stage writes 
the transaction's $ts_c$ to the commit columns of all keys in  the transaction's write-set. 
In case of an abort, it removes the pending records.
Finally, it garbage-collects the transaction's CT entry. 

\subsubsection{High availability} The TM is implemented as a primary-backup process pair to ensure its high availability. 
The backup detects the primary's failure using timeouts. When it detects a failure, it immediately begins serving new
transactions, without any recovery procedure. Because the backup may falsely suspect the primary because of
unexpected network or processing delays (e.g., garbage collection stalls), we take precautions to avoid correctness
violations in (rare) cases when both primaries are active at the same time.

To this end, we maintain two shared objects, managed in Apache Zookeeper and accessed infrequently.
The first is \emph{maxTS}, which is the maximum timestamp an active TM is allowed to return to its clients.
An active (primary) TM periodically allocates a new \emph{epoch} of timestamps that it may return to clients by atomically
increasing maxTS using an atomic compare-and-swap (CAS) operation.  
Following failover, the new primary  allocates a new epoch for itself, and thus all the timestamps it returns to clients
exceed those returned by previous primary. 

The second shared object is a locally-checkable \emph{lease}, which is essentially a  limited-time lock.
The lease is renewed for some parameter $\delta$  time once 80\% of this time elapses.
As with locks, at most one TM may hold the lease at a given time (this requires the TMs' clocks to advance roughly at the same rate). 
This ensures that no client will be able to commit a transaction in an old epoch after the new TM has started 
using a new one.



