
NOSQL key-value stores such as BigTable, HBase, and RocksDB have increased in popularity over the past few years, and are now very widely deployed. Yet in many installations, the single-key access offered by these data stores is insufficient; developers  increasingly require 
 \emph{ACID} (atomic, consistent, isolated, and durable) \emph{transactions} that access multiple keys.
 (Transaction API and semantics are discussed in Section~\ref{sec:api}).
This need is addressed by a number of popular  \emph{transaction processing systems (TPSs)} 
such as  Percolator~\cite{Percolator2010}, Omid~\cite{OmidICDE2014,omid-blog}, Tephra~\cite{tephra}, and CockroachDB~\cite{cockroach}, 
which offer transactions on top of  BigTable~\cite{bigtable-osdi06}, HBase~\cite{hbase}, HBase, and RocksDB~\cite{rocksdb}, resp. 
The underlying NOSQL stores are typically distributed  over multiple \emph{regions} (sometimes called nodes, shards, domains, or partitions).
The main purpose of a TPS is to support \emph{global transactions}, i.e., ones that span multiple regions.

The need to atomically perform multiple  updates  entails a two-phase design, whereby transactions first indicate their \emph{intent} to write certain values, and then atomically \emph{commit} (or \emph{abort}) all their write intents at the same time. In addition, 
all the values written by a single transaction are associated with a common  \emph{timestamp} (or \emph{version number}). 
Reads, in turn, rely on these timestamps in order to read a consistent snapshot of the data store. They further  
require some mechanism (either centralized or distributed) to determine what to do on encountering a write intent
%, which may be implemented in various ways, either centralized or distributed.

Transaction processing thus follows the general schema outlined in Algorithm~\ref{alg:schema}, 
while systems vary in their implementations of each of the steps. 
For example, whereas most systems rely on a centralized service (requiring remote access) for timestamp allocation~\cite{OmidICDE2014,omid-blog,tephra,Percolator2010}, this is not essential~\cite{cockroach}; similarly, validation (conflict detection) can use a centralized service~\cite{OmidICDE2014,omid-blog,tephra}, per-transaction records in a global table~\cite{cockroach}, or two-phase commit~\cite{Percolator2010}. Different ways to implement this schema are  discussed in Section~\ref{sec:context}.
 
\begin{algorithm}[tb]
\begin{algorithmic}
\Procedure{begin}{}
\State obtain tentative (start) timestamp $ts_1$ 
%and transaction id $txid$
\EndProcedure
\Statex

\Procedure{read}{key} \Comment transactional read
\If{hasWriteIntent(key)}
	\State resolve intents, possibly by aborting transactions
\EndIf
\State return latest version of key that does not exceed $ts_1$
\EndProcedure
\Statex

\Procedure{write}{key, value} \Comment transactional write
\State optionally check for conflicts and abort if found 
\State indicate write intent for key with value and $ts_1$
\State add key to local write-set
\EndProcedure
\Statex

\Procedure{commit}{write-set}
\State obtain commit timestamp $ts_2$
\If{validate(write-set, $ts_2$)}  \Comment check for conflicts
	\Statex \Comment commit all write intents with version $ts_2$
	\State atomically and persistently indicate commit 
\Else
	\State abort	
\EndIf
\State clean-up write intents and other meta-data
\EndProcedure
\end{algorithmic}
\caption{TPS operation schema.} 
\label{alg:schema}
\end{algorithm} 

The two-phase transaction processing schema induces high overhead, especially on short transactions. For example, a transaction writing a single key (and reading none) goes through the following stages:  
(1) \emph{begin} to obtain a tentative timestamp, in most implementations from a centralized service;  
(2) add \emph{write intent} to the data store; 
(3) \emph{get commit timestamp}, typically using a centralized service, (though in some cases, the tentative timestamp is used);
(4) \emph{validate} (conflict detection), possibly combined with the above (using a centralized service) or using two-phase locking; 
(5) \emph{indicate commit}, sometimes at the data store and sometimes at a global table; and 
(6) \emph{clean-up} -- replace write intent with actual write in the data store. 
The overhead is exacberated by 
the fact that some of these steps involve access to centralized services, which, in a large deployment, are remote most of the time.

In this work we expedite   \emph{local transactions}, namely,  transactions that span a single region, in particular, short ones. We
introduce a \emph{fast path} for such transactions, which mitigates this overhead
without a significant impact on global (multi-region) transactions.
The concept is generic and is applicable to a family of TPSs that use the schema of Section~\ref{sec:context}, 
including Percolator~\cite{Percolator2010}, Omid (versions 1 and 2)~\cite{OmidICDE2014,omid-blog}, Tephra~\cite{tephra}, 
and CockroachDB~\cite{cockroach}. 
In Section~\ref{sec:new-api} we define API extensions for fast local transactions, 
and in Section~\ref{sec:alg}, we consider  a generic TPS that supports global transactions, 
and enhance it with optimized support for specific types of local transactions. 
Our main goal is to expedite short single-object transactions,
which are popular in production web workloads~\cite{omid-blog}.

A second contribution of this paper is building  a new low-latency transaction processing system, \sys, presented in Section~\ref{sec:impl},
where we embody the fast path concept.  
\sys\ is based on the open-source version of Omid 2~\cite{omid-blog}, a transaction processing engine for HBase, 
but replaces its centralized commit table access with a lower-latency distributed approach to indicating commits.  
\sys\ supports local transactions via (1) distributed timestamp allocation; (2) extensions to the underlying key-value store, 
which we implement in HBase; and (3) an extended client-side transaction library. 
\Idit{Discuss evaluation results given in Section~\ref{sec:eval}.}

\remove{
TPS design is typically optimized for global transactions, and they typically employ some centralized component to facilitate their coordination. 
For example, Omid and Tephra use a centralized \emph{Transaction Manager (TM)} 
(also called \emph{Transaction Status Oracle (TSO)}) for (1) allocating global timestamps; (2) conflict detection; 
and (3) tracking committed transactions in a global table, 
whereas  Percolator uses a centralized timestamp oracle for the first 
and CockroachDB uses a global \emph{Transaction Table (TT)} for the last. 
Common to all these systems is the use of globally-meaningful timestamps, either provided by the centralized TM (Omid, Tephra) or oracle (Percolator), or maintained using a causality-enforcing clock synchronization protocol (in CockroachDB).
Note that these are different from federated systems, where transactions are by default local, i.e.,
access a single region, and lock-based (blocking) protocols like two phase commit are used to ensure atomicity of global transactions.

Optimizing for global transactions introduces a tradeoff: while transactions spanning
multiple regions are greatly facilitated and expedited using this design, local
transactions incur a performance penalty. This penalty is particularly
significant for short transactions, where the overhead of accessing 
a centralized entity for obtaining a timestamp, logging the transaction in a common table, or detecting conflicts 
is not amortized across many operations. Typical web workloads include both long
multi-region transactions and short local transactions, including ones
consisting of a single object read, write, or read-modify-write.

In this work we mitigate the above tradeoff by offering a \emph{fast path} for short
local transactions, without a significant impact on multi-region ones, while
maintaining correctness. We consider a TPS that supports global
transactions, and enhance it with optimized support for specific types of local
transactions. Our main goal is to expedite short single-object transactions,
which are popular in production web workloads~\cite{sieve}.
%
We focus here on systems supporting \emph{snapshot isolation} (SI)~\cite{DBLP:conf/sigmod/BerensonBGMOO95}, which is
popular in real-world systems  and amenable to scalable implementations.
Our local transactions can be performed using the usual
transaction mechanism, with no change in semantics.

 }
  