

We now explain our support for local transactions with SI semantics in the
context of a system as described in Section~\ref{sec:context} above. 
Our solution consists of three parts: First, in Section~\ref{ssec:region-clock}, we enhance the
underlying data store with support for per-region \emph{Local Version Clocks (LVCs)}. 
This aspect is already implemented in CockroachDB, which uses 
per-region Hybrid Logical Clocks~\cite{Kulkarni2014LogicalPC} in order to allow for distributed timestamp allocation. 
In the systems that maintain a GVC, (e.g.,~\cite{Percolator2010,tephra,OmidICDE2014,omid-blog}), our 
addition of  LVCs 
entails a minor modification to management of global timestamps (in the transaction manager or oracle). 
Second, in Section~\ref{ssec:lvc-access}
we extend the underlying data store's API to allow manipulating 
a region's LVC jointly with objects stored at that region. 
Finally, we add client-side support for the fast path API, as explained in Section~\ref{ssec:lc-client}.

\subsection{Local version clock} \label{ssec:region-clock}

\remove{
We add some functionality to a region, to be used exclusively by our client-side
extensions given in Section~\ref{ssec:lc-client} below. The key mechanism used to support local
transactions is a local version clock (LVC) per region. 
}

Like the GVC, LVCs are also monotonically increasing. Each region has its
own LVC, which is loosely synchronized with the GVC. The idea is to use the
region's LVC for ordering FP transactions in any given region, and allow
FP transactions to progress independently in different regions. Note that it
is safe to do so because no global order needs to be enforced among FP transactions
that access disjoint sets of objects.

Regular (multi-region) transactions, in turn, continue to obtain their versions from the
GVC (except in CockroachDB, where local clocks are used for all transactions and 
a conservative conflict detection policy enforces aborts in case there is uncertainty regarding
the global order of events). 
Therefore, whenever a regular transaction $T$ accesses a given region,
we synchronize that region's LVC with the GVC in order to ensure that
the timestamp obtained by $T$ exceeds those obtained by earlier completed
transactions within the region, and that later transactions within that region
will obtain higher versions than $T$'s. (A similar approach was used for enforcing 
causality in Hybrid Logical Clocks~\cite{Kulkarni2014LogicalPC} as used by CockroachDB, 
and for supporting local transactions in Mediator~\cite{mediator}.

To support such loose synchronization, the GVC now advances at a coarse
granularity of \emph{epochs}. This can be implemented, for example, by choosing some
epoch size $2^\ell$, and keeping the $\ell$ least significant bits of the GVC padded
with zeros. In other words, every increment of the GVC increases its value by
$2^\ell$.
The LVC obtains the epoch ($n-\ell$ most-significant bits for an $n$-bit GVC) from the
GVC, and proceeds to assign timestamps within the designated epoch (by
incrementing the least significant bits).

The LVC has a single component, LVC.current, and it supports the following API:
\begin{description}
\item[\code{LVC.skip(epoch)}] -- atomically set LVC.current to max(epoch, LVC.current).

\item[\code{LVC.fetchAndIncrement()}] -- atomically increment LVC.current and return its new
value.
For simplicity, we assume that the LVC is used so it does not overrun the epoch
and does not wrap around within the epoch. That is, fetchAndIncrement is called
less than $2^\ell$ times in each epoch.

\item[\code{LVC.get()}] -- return LVC.current.
\end{description}

By incrementing the GVC, a multi-region transaction essentially initiates a new
epoch, and obtains a timestamp exceeding all those of older local transactions.
In addition, multi-region transactions enforce the synchronization of the LVC
with respect to the GVC using the skip operation. Specifically, whenever a
transaction in a new epoch accesses (for either read or write intention
indication) an object in a region whose LVC is still in an older epoch, it
invokes that region's LVC.skip so it will not lag behind the transaction's
tentative timestamp ($ts_1$) obtained from the GVC. Thus, new local transactions that will
begin later in the region will have higher timestamps, as needed.
Note that transaction commits do not alter the LVC; the LVC only reflects
transactions' tentative timestamps obtained when they  begin.

Note also that as long as a running transaction does not access a given region,
further updates can occur by local transactions in that region, and these
updates can be reflected in the transaction's snapshot in case it later reads
these objects. Thus, unlike with regular transactions, which satisfy real-time
order, a transaction's snapshot may reflect changes that occur after it
commences, as in the example of Figure~\ref{fig:ltx-rt}.

Here, x is written with LVC=10 in one region, and later y is written in another region with its LVC=0. The
concurrent transaction's snapshot time is 10, which includes the update of x and
not that of y. The same scenario does not occur with two local transactions
accessing the same region, since once the LVC is incremented, no further
updates in the same region can occur with older LVC values.



\subsection{Accessing data and the LVC together} \label{ssec:lvc-access}

In order to allow FP transactions to execute with a single data store access,
we extend the data store to support functions that access data objects and the
LVC together. Thus, an FP transaction can increment LVC, obtain a timestamp, 
write with this timestamp, and commit, all in one round-trip to the local data store.

We further enforce atomic access to the data and the LVC. This is important in
order to avoid races between obtaining a version from the LVC and updating the
data. For example, if the updates of the LVC and the data were separate, the
following scenario could have arisen (x and y are in the same region):
\Yoni{Find shorter example?}
\begin{itemize}
  \item FP transaction $FP_1$ plans to update object x and obtains LVC value $1$.
  \item Regular transaction $T_1$ obtains tentative timestamp $10$.
  \Yoni{\item $T_1$ reads y, causes LVC to skip to $10$.}
  \item FP transaction $FP_2$ obtains LVC value $11$ and updates object y.
  \item Regular transaction $T_2$ obtains tentative timestamp $20$.
  \item $T_2$ reads the old version of x  (since it had not yet been written by $FP_1$)
  and the new version of  y.
  \item $FP_1$ writes x with version $1$ and completes.
  \item $T_2$ reads the new version of x and the old version of  y.
\end{itemize}
Here, SI is violated.

In addition, the new functions must take care not to breach the  atomicity of concurrent transactions. 
To this end, they rely on  write intents. 
One option is to abort an FP transaction whenever a write intent exists for one of its written or read keys.\Yoni{read keys?} 
Starvation can be avoided by retrying the transaction as a regular one.
Alternatively, an FP transaction may resolve write intents similarly to regular ones.
Note that in case the resolution accesses a global table, this makes the FP transaction non-local.
Nevertheless, such remote access is needed only in case of contention between a regular transaction 
and a concurrent FP one on the same key.

We extend the data store API with the following calls, whose 
pseudocode is given in Algorithm~\ref{alg:lvc-and-data}. 

\begin{description}
   \item [\code{getWithTs(key)}]  returns the current LVC read $ts$, plus the highest committed version-value pair pertaining to key with version up to ts.
   Write intents are either resolved or lead to abort.
   \item[\code{mutate(key, ts)}] atomically increments the region's LVC
  and creates a new version for key associated with the new LVC value. The
  operation fails (returns false) if the latest version of the object has a write intent indication. 
   \item [\code{validateAndMutate(key, ts, value)}]
atomically validates that the  highest version of key in the data store does not exceed the provided timestamp
and mutates the  object associated with key and the LVC as \code{mutate} does. Returns true if the
validation is successful, otherwise returns false and does not perform the mutation. 
\end{description}

\begin{algorithm}[htb]
\begin{algorithmic}
\Procedure{getWithTs}{key}
\State ts $\leftarrow$ \code{LVC.get()}
\State $\langle$ version, value $\rangle \leftarrow$ highest version  $\leq$ ts of key
\If{\code{hasWriteIntent(key, version)}} 
	\State resolve write intent or abort
\EndIf
\State return $\langle$ ts, value $\rangle$ 
\EndProcedure
\Statex
%
\Procedure{mutate}{key, value} 
\If{ \code{hasWriteIntent(key)} }
\State return false
\EndIf
\State  version $\leftarrow$ \code{LVC.fetchAndIncrement()}
\State add $\langle$ version, value $\rangle$ to key
\State return true
\EndProcedure
\Statex
%
\Procedure{validateAndMutate}{key, ts, value} 
%%%%$\langle$ \rkey, \rversion $\rangle$*}
\If{ \code{hasWriteIntent(key)}}
	\State return false
\EndIf
\If{latest version of key $>$ ts}
	\State return false
\EndIf
\remove{
\ForAll{$\langle$ \rkey, \rversion $\rangle$}
 	\If{latest version of \rkey $\neq$ \rversion} 
    		\State return false
    	\EndIf
 \EndFor
 }
\State  version $\leftarrow$ \code{LVC.fetchAndIncrement()}
\State add $\langle$ version, value $\rangle$ to key
\State return true
\EndProcedure
%
\end{algorithmic}
\caption{Atomic access of LVC and data in a single region data store. Each method executes atomically.}
\label{alg:lvc-and-data}
\end{algorithm}


\subsection{Client-side support for FP transactions} \label{ssec:lc-client}

There are two aspects to supporting FP transactions. First, we need to implement the client API defined in Section~\ref{sec:new-api} above.
We do this in Section~\ref{ssec:local-client-alg}. Second, we need to modify the implementation of regular transactions to be aware of 
FP ones; this is addressed in Section~\ref{ssec:regular-client-alg}.

\subsubsection{Implementing FP APIs} \label{ssec:local-client-alg}

Pseudocode for the client-side implementation of the FP transaction API is given in  Algorithm~\ref{alg:ltx-client}. 
As for regular transactions, the ongoing transaction's tentative timestamp is stored in a local variable,
which we denote by \code{ts$_1$}. 

The 
\code{br} and \code{bwc} APIs are implemented directly using the new functions \code{getWithTs} and \code{checkAndMutate}, resp.
In addition, \code{br} stores its obtained timestamp in  \code{ts$_1$},  for potential subsequent \code{r} calls.
An ensuing \code{wc} call validates that the written key has not been modified since the preceding \code{br} call.


 Another streamlined type of FP transaction is a single-key read-and-write,  \code{brwc}. It is
parametrized by some compute function $f$ that generates the new value from the old one.
Because \code{brwc} does not perform the update to the data store atomically
with the read, it needs to validate the \code{mutate} operation to ensure that the object
has not been updated since its latest version was read at the beginning of the
operation. If the validation fails, the transaction can be retried, either the
same way or using a regular transaction.


\begin{algorithm}[htb]
\begin{algorithmic}
\State local variable \code{ts$_1$} \Comment tentative ts of ongoing transaction
\Statex
\Procedure{br}{key} 
\State $\langle$ \code{ts$_1$}, value  $\rangle \leftarrow$ getWithTs(key)
\State  return value
\EndProcedure
%
\Statex
\Procedure{bwc}{key, value} 
\State  return mutate(key, value)
\EndProcedure
%
 \Statex
\Procedure{wc}{key, value} 
\State return \code{validateAndMutate(key, \code{ts$_1$}, value)}
\EndProcedure
%
 \Statex
\Procedure{brwc}{key, f} 
\State $\langle$ \code{ts$_1$}, value $\rangle \leftarrow$ getWithTs(key)
\State  return \code{validateAndMutate(key, \code{ts$_1$}, f(value))}
\EndProcedure
\end{algorithmic}
\caption{Client-side code for FP transactions.}
\label{alg:ltx-client}
\end{algorithm}

\subsubsection{Ensuring monotonicity of written versions} \label{ssec:regular-client-alg}

We saw that FP transactions become aware of pending regular transactions via write intents. This ensures correctness
in case a write intent is written for a key before the FP transaction accesses it. However, it may be the
case that a regular transaction with a smaller timestamp than the FP transaction will write its intent (with 
the smaller timestamp) \emph{after} the FP transaction accesses the key as in the following scenario:
\begin{enumerate}
\item Transaction $T_1$ begins with tentative timestamp $10$.  
\item FP transaction $FP_1$ writes key and commits with timestamp $11$.  
\item Transaction $T_1$ writes key with timestamp $10$.  
\end{enumerate}
In this case, $T_1$ must detect the write-write conflict with $FP_1$ and abort.
To this end, we slightly modify the write operation of regular transactions so as to validate that no 
later versions exist for the written key. In other words, a write operation only creates a new version if it exceeds all
existing ones, and otherwise aborts.
We note that  some data stores and TPSs already employ this modus operandi of writing only monotonically increasing versions.
This feature is also beneficial for regular transactions, as it leads to early conflict detection.




