\subsection{\sysll\ algorithm} 
\label{ssec:ll}

\begin{algorithm}[htb]
\begin{algorithmic}[1]
\small
\Procedure{begin}{}
\State checkRenew() \Comment check and renew lease for HA support
\State return Clock.fetchAndIncrement()
\EndProcedure
%\Statex
\Procedure{commit}{$txid$, write-set}
\State checkRenew() \Comment check and renew lease for HA support
\State $ts_c \leftarrow$ Clock.fetchAndIncrement()
\If{conflictDetect($txid$, write-set, $ts_c$) } 
 \State return $ts_c$
 \Else
 \State return {\sc abort}
\EndIf
\EndProcedure
%\Statex
%
\remove{
\Procedure{checkRenew}{} \Comment HA support;  $\delta$ is the lease time
\If{lease $<$ now + \inred{0.2}$\delta$} \label{l:lease-start} \Comment extend lease
 \State renew lease for $\delta$ time \Comment atomic operation
 \If{failed} halt \EndIf 
\EndIf  \label{l:lease-end}
\If{Clock $=$ epoch}  \label{l:epoch-start} \Comment extend epoch
\State epoch $\leftarrow$ Clock +  range
 
\If{$\neg$CAS(maxTS, Clock, epoch) } halt \EndIf 
\EndIf  \label{l:epoch-end}
\EndProcedure
}
\end{algorithmic}
\caption{\sysll's TM algorithm.}
\label{alg:ha}
\end{algorithm}



When a transaction begins, it obtains a read timestamp (version) $ts_r$ for reading its consistent snapshot.
To commit, it obtains  a commit timestamp, $ts_c$ and persists a commit entry in the CT that indicates that
the transaction with read timestamp $ts_r$ has committed with timestamp $ts_c$. 

Whereas Omid's CT is updated by the TM, \sysll\ distributes the CT updates amongst the clients.
Its TM is thus a simplified version of Omid's TM, and appears  in Algorithm~\ref{alg:ha}. It has two roles:
First, it allocates read and commit timestamps by fetching-and-incrementing a monotonically increasing global clock.
Second, upon commit, it calls the \emph{conflictDetect} function, which 
performs validation  using an in-memory hash table
by checking, for each key in the write-set, that its commit timestamp in the hash-table is smaller than the 
committing transaction's $ts_r$. If there are no conflicts, it 
updates the hash table with the write-set of the new transaction and its $ts_c$. 
%(For more details see~\cite{Omid2017}).
The  \emph{checkRenew} procedure supports the TM's  high availability (HA), and its description is beyond the scope of this paper; see~\cite{Omid2017,OmidVLDB} for details. 
%explained at the end of this section.



\noindent
Client operations proceed as follows (cf.~Algorithm~\ref{fig:get-pseudocode}):


\begin{algorithm}[htb]
 \small
  \begin{algorithmic}
 \Procedure{begin}{}
       \State  return TM.begin
\EndProcedure
%\Statex
  \Procedure{write}{$ts_r$, key, {fields, values}}
       \State  track key in write-set 
       \State {add commit to fields and nil to values}
       \State    {return ds.put(key, $ts_r$, fields, {values})}
\EndProcedure
%     
%\Statex
\Procedure{read}{$ts_r$, key} 
      \For{rec $\leftarrow$ ds.get(key, versions down from $ts_r$)}
% 			\State $ts_c \leftarrow$ {\sc checkStatus}(rec, $ts_r$)     			       
			\State \Comment  set $ts_c$ to the commit timestamp associated with rec
			\If{rec.commit $\not =$nil}  \Comment commit cell exists
     				\State $ts_c \leftarrow$  rec.commit 
			\Else \Comment commit cell is empty, check CT
      				\State $ts_c  \leftarrow$ CT.get(rec.ts)  
      				\If{$ts_c =$ nil}  \Comment no CT entry, set status to  $\bot$
      					\State {$ts_c\leftarrow$CT.check\&mutate(rec.ts, commit, nil, $\bot$)}  
      					\If{$ts_c$=nil}  \Comment forced abort successful 
      						\State $ts_c  \leftarrow \bot$ 
      					\EndIf
				\EndIf
      				
 				 \If{$ts_c = \bot$}  
 				 	\State \Comment check for race: commit before $\bot$ entry  created 
 				 	\State re-read rec from ds
 				 	\If{rec.commit $\not =$nil}
 				 		 \State $ts_c \leftarrow$  rec.commit 
 				 		 \State CT.remove(rec.ts)
 				 	\Else\  
 				 		 continue	\Comment writing transaction aborted	
 				 	\EndIf
				\EndIf
			\EndIf

       			 \If{$ts_c <  ts_r$}  
%     			 \If{$ts_c \not=$ abort  $\wedge\ ts_c <  ts_r$}  
 			 	\State return  rec.value 
			\EndIf
      \EndFor
      \State  return nil \Comment no returnable version found in loop
      \EndProcedure



%      		\Procedure{checkLeader}{\tuple{k, ts}} 
%     			\State leader $\leftarrow$ ds.get(\tuple{k, ts})
%    			 \If{leader $=$ nil} return abort \Comment{leader removed} \EndIf 
%		     	 \If{ leader.commit $\not=$nil} \Comment transaction is complete
%		     	 	\State return leader.commit 
%		     	 \EndIf
%		     	 \Comment try to abort pending transaction 
%			\State ok $\leftarrow$ ds.check\&mutate(leader.commit, nil, abort)
%			\If{$\neg$ok} \Comment leader status has changed -- recheck  
%				\State return  {\sc checkLeader}({leaderPtr}) \EndIf
%			\State return abort
%	   	\EndProcedure


\Procedure{commit}{$ts_r$, write-set}
      	\State $ts_c \leftarrow$ TM.commit($ts_r$, write-set) \Comment may return abort
      	\If{$ts_c \not=$ abort}
	      	\If {CT.check\&mutate($ts_r$, commit, nil, $ts_c$) $=\bot$} 
   		 	\State $ts_c \leftarrow$ abort  %\Comment transaction forcefully aborted
   		 \EndIf
	\EndIf

	\ForAll{keys k $\in$ write-set}
			\If{$ts_c =$ abort} ds.remove(k, $ts_r$)  	
%			\Else\ update  commit field of \tuple{k, $ts_r$} to $ts_c$  in ds%  $\leftarrow ts_c$ 
			{\Else\ ds.put(k, $ts_r$, commit, $ts_c$)}
			\EndIf
	\EndFor
	\State CT.remove($ts_r$) \Comment garbage-collect CT entry
\EndProcedure
      
  \end{algorithmic}
  \caption{\sysll's client-side operations.} 
  %for transaction with read timestamp $ts_r$.}
  \label{fig:get-pseudocode}
\end{algorithm} 

%\clearpage



\mypara{Begin.}
The client sends a begin request to the TM, which assigns it a read timestamp $ts_r$.

\mypara{Write.}
 During a transaction, a write operation   \emph{tentatively} writes a new value to an object with a certain version number.
The version is the transaction's $ts_r$, which exceeds all versions written by transactions that committed before it  began. 
A dedicated \emph{commit} column is initialized to nil to indicate that the write is still tentative (not committed); 
after the transaction commits successfully, the value of the commit field is set to the writing transaction's commit timestamp.
The write further tracks key in its local write-set.

\mypara{Read.}
The algorithm traverses data  records (using the data store's ds.get API) pertaining
to the requested key with a version that does not exceed $ts_r$, latest to earliest, and returns the first value that is committed
with a  version smaller than or equal to $ts_r$. To this end, it needs to discover the commit timestamp, $ts_c$, associated with
each data record it considers. 

If the record is tentative (commit=nil), then we do not know whether the transaction that wrote it is still pending. 
%or simply not committed. 
Since read cannot return without determining its final commit timestamp, 
it must forcefully abort it in case it is still pending, and otherwise discover its outcome.
To this end, read first refers to the CT.
If there is no CT entry associated with the transaction ($ts_c$=nil), 
read attempts to create an entry with $\bot$ as the commit timestamp. 
This is done using an atomic {check\&mutate} operation, due to a possible 
race with a commit operation. %; if a competing commit attempt finds the   $\bot$ entry, it  aborts. 
%Note that it is possible  to wait a configurable amount of time before attempting to create a $\bot$ entry, in
%order to allow the transaction complete without aborting it.

There is  another subtle race to consider in case the commit record is set to $\bot$ by a read operation. 
Consider a slow read operation that reads the data record rec when its writing transaction is still pending, and then stalls
for a while. In the interim, the writing transaction successfully commits, updates the data record, 
and finally garbage-collects its  CT entry.  At this point the slow reader wakes up 
and does not find the commit entry in the CT. It then creates a $\bot$ entry even though the transaction had successfully committed.
To discover this race, we have a read operation that either creates or finds a $\bot$ entry in the CT re-read the data record.
%, and  if it does contain a commit entry, use it. 

\mypara{Commit.}
The client first 
\remove{flushes all its writes to the data store and then} 
sends a commit request to the TM. 
If the TM detects conflicts then the transaction aborts, and otherwise the TM provides the transaction with its commit  timestamp $ts_c$. 
\remove{Flushing before calling the TM ensures that transactions with read timestamps exceeding $ts_c$ 
will find the committing transaction's records in the data store.}
The client then proceeds to commit the transaction, provided that no read had forcefully aborted  it. To ensure the latter, the client uses 
an atomic {check\&mutate} to create the commit entry. 

To avoid an extra read of the CT on every transactional read, once a transaction is committed, it writes 
the transaction's $ts_c$ to the commit columns of all keys in  the transaction's write-set. 
In case of an abort, it removes the pending records.
Finally, it removes the transaction's CT entry. 




\subsection{Performance evaluation}
\label{ssec:ll-graphs} 

We compare \sysll's performance to that of Omid. 
Our experiment testbed consists of nine 12-core Intel Xeon 5 machines with 46GB RAM and 4TB 
SSD storage, interconnected by 10G Ethernet. We allocate three of these to HBase nodes, 
one to the TM, one to emulate the client whose performance we measure, and four more to simulate 
background traffic.  
Our test cluster stores approximately 23M keys ($\sim\!\!7$M keys per node). 
The values are 2K big, yielding roughly 46GB of actual data, replicated three-way in HDFS. The keys are hash-partitioned
across the servers. The data accesses are 50\% reads and 
50\% writes. The key access frequencies follow a Zipf distribution
%, generated following 
%the description in~\cite{Gray:1994:QGB:191839.191886}, with $\theta=0.8$, which yields the abort rate reported
%in the production deployment of Omid~\cite{Omid2017}. In this distribution, the first key is accessed 0.7\% of the  time.
%Transaction sizes (number of reads and writes) follow a Zipf distribution with $\theta=0.99$, with a cutoff at $10$. 
%With these parameters, $63\%$ of the transactions access three keys or less, and only $3\%$ access $10$ keys. 
%To make the comparison meaningful, we study transactions of different lengths  separately. 

Figure~\ref{fig:throughput-latency} presents the average latency of single-key and ten-operation transactions as a function of {system} throughput.
Additional experiments are reported in~\cite{OmidVLDB}.  


\begin{figure*}[htb]

\centering
\begin{tabular}{cc}

    \begin{subfigure}[t]{0.3\textwidth}
      	\includegraphics[width=\textwidth]{figs/throughputlatency1.pdf}
	    \caption[]{\small Transaction size = 1}
        \label{fig:tl-1}      
   \end{subfigure}  
& 
    \begin{subfigure}[t]{0.3\textwidth}
	\includegraphics[width=\textwidth]{figs/throughputlatency10.pdf}
	\caption[]{\small Transaction size=10}
    \label{fig:tl-10}
  \end{subfigure}  
 \end{tabular} 
 
  \caption{\small Latency vs.\ throughput  in Omid and \sysll. \inred{Yoni: please create the graphs with only Omid and Omid-LL, thanks!} }
  \vspace{-0.3cm}
  \label{fig:throughput-latency}
\end{figure*}



