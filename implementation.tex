	\subsection{\sys\ algorithm} 
\label{ssec:ll}

\begin{figure}[!t]
  \centering
  
  \begin{subfigure}[t]{\columnwidth}
      \centering
    \begin{tabular}{|c|c|c|c|c|}
      \hline
      key & value & version & commit& leader\\
      %\hline
      \hline
      k1 & a & 3 & &\\
      %\hline
      k2 & b & 3 & &k1\\
      %\hline
      k2 & c & 3 & &k1\\
      \hline
    \end{tabular}
	\caption[]{Tentative transaction}
    \label{fig:model:tentative}
  \end{subfigure}
  
  \begin{subfigure}[t]{\columnwidth}
    \centering
    \begin{tabular}{|c|c|c|c|c|}
      \hline
      key & value & version & commit& leader\\
      \hline
      %\hline
      k1 & a & 3 & 7&\\
      %\hline
      k2 & b & 3 & &k1\\
      %\hline
      k2 & c & 3 & &k1\\
      \hline
    \end{tabular}
	\caption[]{Committed transaction}
    \label{fig:model:committed}
  \end{subfigure}


  \begin{subfigure}[t]{\columnwidth}
    \centering
    \begin{tabular}{|c|c|c|c|c|}
      \hline
      key & value & version & commit& leader\\
      \hline
      %\hline
      k1 & a & 3 & 7&\\
      %\hline
      k2 & b & 3 & 7&\\
      %\hline
      k2 & c & 3 & 7&\\
      \hline
    \end{tabular}
	\caption[]{Post-committed transaction}
    \label{fig:model:postcommit}
  \end{subfigure}

  
  \caption{Evolution of metadata during a transaction. The transaction receives $ts_r=3$ in the begin stage, 
  and the $ts_c$ it receives when committing is $7$; its leader is k1.}
  \label{fig:model}
\end{figure}


\begin{algorithm}[t]
  \begin{algorithmic}
    \begin{small}
      \Procedure{read}{key} 
      \For{rec $\leftarrow$ ds.get(\emph{key}, versions down from $ts_r$)}
		\If{rec.commit $\not \in\{$nil, abort$\}$} \Comment committed
     			 \If{rec.commit $< ts_r$}  \State return rec.value \EndIf
      		\ElsIf{rec.commit = nil}  
      			\State leader $\leftarrow$ rec.leader.commit
      			 \If{leader $\not \in\{$nil, abort$\}$} \Comment committed
     				 \If{leader $< ts_r$}  \State return leader \EndIf
			\Else \Comment abort conflicting transaction
				\State $ts_c$ $\leftarrow$ checkAndMutate(leader, nil, abort)
      				\If{$ts_c$ $\not=$ {abort}} \Comment  committed
     					 \State rec.commit $\leftarrow ts_c$ \Comment helping 
      				  	\If{$ts_c$ $< ts_r$}  return rec.value \EndIf
				\EndIf
			\EndIf
		\EndIf
      \EndFor
      \State  return nil
      \EndProcedure

\Statex

\Procedure{commit}{}
      
      	\State $ts_c \leftarrow$ TM.commit($ts_r$, write-set) \Comment may return abort
      	\State  res $\leftarrow$ {\sc checkAndMutate}(leader.commit, nil,  $ts_c$)
	\State \Comment post-commit
	\ForAll{versions ver written by this transaction}
			\If{res = abort} remove ver      	
			\Else\ ver.commit $\leftarrow ts_c$ 
			\EndIf
	\EndFor
\EndProcedure
      
       \end{small}
  \end{algorithmic}
  \caption{\sys's read and commit operations.} 
  %for transaction with read timestamp $ts_r$.}
  \label{fig:get-pseudocode}
\end{algorithm} 

The meta-data used by \sys\ for transaction management is stored in two additional columns associated with each object version in the data store:
(1) \emph{commit} holds the status of the transaction that wrote this version -- \emph{nil} if it is pending, its  $ts_c$ if committed, and 
otherwise \emph{abort}; 
and
(2) \emph{leader} is a pointer to the writing transaction's commit entry, namely, the \emph{commit} column of the first key written by the same
transaction. 

Since the transaction commit needs to be an atomic step, the first key in the transaction's write-set is chosen to be the leader, and the transaction commits via a single atomic write of $ts_c$ to the leader's commit column. Transactional read operations, in turn, refer to the leader to find out whether the value they are attempting to read has been committed or not.
Figure~\ref{fig:model} shows the metadata of a transaction during its stages of execution. 

\sys's TM is based on Omid's TM, and has the same functionality except that it does not write commit entries.
In particular, it manages a monotonically increasing global clock for allocating read timestamps on begin and commit timestamps on commit,
and it performs validation (i.e., conflict detection) upon commit using an in-memory hash table. 
Since the TM algorithm has been reported elsewhere~\cite{omid-fast}, and the \sys\ client need not be aware of 
its implementation details, we avoid repeating these details here. 

Algorithm~\ref{fig:get-pseudocode} describes \sys's implementation of read and commit operations.
%Begin and write operations follow the template of Algorithm~\ref{alg:schema}. 
The client's operations proceed as follows:

\paragraph{Write.}
The client adds the tentative record \emph{(key, value, $ts_r$, nil , leader)} to the data store, where
the leader is the first key the client updates during the transaction. It also tracks key in its local write-set.
\remove{
To ensure monotonicity of written versions as seen in Section~\ref{ssec:regular-client-alg}, instead of using the standard \emph{put} operation the data store exposes, we must use checkAndMutate which will atomically put the record only if the latest version of the key is smaller than $ts_r$.
}

\paragraph{Read.}
The data store's get operation is referred to as ds.get(key,version). The algorithm traverses records pertaining
to key with versions smaller than $ts_r$, latest to earliest, and returns the first key that is committed. Upon
encountering a tentative record (with commit=nil), the algorithm gets the leader's record from the data store,
reads its $ts_c$, and again returns the key if the $ts_c$ is smaller than $ts_r$. 

If thecommit entry is still nil, read cannot return without determing the final outcome
of the pending transaction. \Yoni{example here?}. 
It  therefore forcefully aborts it. This is done using an atomic 
 {checkAndMutate} function that atomically reads and sets the commit entry to abort if its value is nil.
%When a transaction commits, it atomically checks it hasn't been invalidated before inserting the $ts_c$ to the commit .


\paragraph{Commit.}
The client sends a commit request to the TM, which assigns it a commit timestamp $ts_c$ and checks for conflicts. 
If the TM does not return abort, this means that the validation has been successful, and the client writes $ts_c$ in the leader's {commit} column. 
However, the transaction can commit only if no read has invalidated it. Therefore, it uses 
the data store's checkAndMutate function to write $ts_c$ only if the commit status is still nil.
To avoid an extra read of the leader on every read, once a transaction is committed, the post-commit stage writes 
the $ts_c$ to the {commit} columns of all keys in  the transaction's write-set. 
%Following a successful commit, the client adds $ts_c$ to all data items it wrote to.

\remove{
\paragraph{Singleton Read}
\Yoni{This is the same as \ref{ssec:local-client-alg}}
\paragraph{Singleton Write}


\Yoni{what about WC and BRWC?}

\subsection{HBase region server}
\label{ssec:hbase}
\Yoni{is there something special to say here?}
}
