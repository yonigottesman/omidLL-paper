%=========================================================================
%  LaTeX definitions
%=========================================================================

%%%     definining margins
%\newcommand{\CLASSINPUToutersidemargin}{0.64in} %redefining margins; from about 0.68 to 0.64 > 0.625; see also package "changepage"

%%%     more margins:
%\newcommand{\CLASSINPUTtoptextmargin}{1in}
%\newcommand{\CLASSINPUTbottomtextmargin}{0.75in}
%\newcommand{\CLASSINPUTinnersidemargin}{0.75in}
%\newcommand{\CLASSINPUToutersidemargin}{0.75in}

%========================
%  Tech. Report  vs. Conference
%========================

%the definitions help switch between the two: \Conf{Due to space limits...} \TR{Here is the full proof}

% to produce the conference paper comment the \conffalse uncomment the \conftrue
\newif\ifconf
%\conffalse
\conftrue

% to include comments in the generated pdf comment the \commfalse and uncomment \commtrue
\newif\ifcomm
\commtrue
%\commfalse

% to produce the blind paper version comment the \conffalse uncomment the \conftrue
\newif\ifblind
%\blindfalse
\blindtrue

\ifconf
    %%Conference version?
    \newcommand{\Conf}[1]{#1}
    \newcommand{\TR}[1]{}
    \newcommand{\Journal}[1]{}  %enable for journal only
    \newcommand{\OnlyTR}[1]{}   %enable for TR only, disable for journal
\else
    %%%%%TR/Journal version?
    \newcommand{\Conf}[1]{}
    \newcommand{\TR}[1]{#1}
    \newcommand{\Journal}[1]{}  %enable for journal only
    \newcommand{\OnlyTR}[1]{#1}   %enable for TR only, disable for journal
\fi

%========================
%   Paper Format
%========================
%%%% IEEE
%\Conf{
%%\documentclass[letterpaper, 10pt, conference]{ieeeconf}
%    \documentclass[conference]{IEEEtran}
%}
%\TR{
%    \documentclass[journal]{IEEEtran}
%}

%%% ACM
\Conf{
\documentclass[singlecolumn]{vldb} 
}
\TR{
\documentclass[letterpaper,conference]{vldb}
}


%%% INFOCOM addition: http://www.ieee-infocom.org/2009/paper-layout.htm
%\makeatletter
%\def\ps@headings{%
%\def\@oddhead{\mbox{}\scriptsize\rightmark \hfil \thepage}%
%\def\@evenhead{\scriptsize\thepage \hfil\leftmark\mbox{}}%
%\def\@oddfoot{}%
%\def\@evenfoot{}}
%\makeatother
%\pagestyle{headings}

%========================
%  Packages
%========================

\usepackage{graphicx}
    %\usepackage[dvips]{graphicx}
    %%\usepackage[pdftex]{graphicx} %if pdftex
%\usepackage{amsmath}
%\usepackage[cmex10]{amsmath} %ACM conflict with next package
\usepackage{amssymb}
%\usepackage{amsthm}  %<---- for a different "look" in theorems (theorem word in bold, etc.) %ACM conflict with proof definition?
%\usepackage{subfigure}
\usepackage[tight,footnotesize]{subfigure}
\usepackage{cite}
\usepackage{url}
%\usepackage{cases}
%\usepackage{color}
%\usepackage[ruled,vlined,linesnumbered]{algorithm2e}
%\usepackage{algorithm}
%\usepackage{algpseudocode}
\usepackage{times}
%\usepackage{showkeys} %[notref]{showkeys}
%\usepackage{epsfig}
%\usepackage{graphics}
%\usepackage{float}
%\usepackage{paralist}
%\usepackage{verbatim}
\usepackage{todonotes}
%\usepackage{authblk}
\usepackage[normalem]{ulem} %strikethrough: \sout{Hello World}
\usepackage{lastpage} %for number of pages
\usepackage{xspace}
\usepackage{multirow}
\usepackage{balance}
%========================
%  Defining theorem types
%========================

\newtheorem{theorem}{Theorem}%[section]    -> example: \begin{theorem}   ... \end{theorem}
\newtheorem{theorem*}{Theorem}
\newtheorem{definition}{Definition}%[section] -> example: \begin{definition} [$B_{h}$ Sequence] \label{def:B} ..... \end{definition}
\newtheorem{proposition}{Proposition}%[section]
\newtheorem{property}{Property} %\newtheorem{property}[theorem]{Property}
\newtheorem{lemma}{Lemma}%[section]
\newtheorem{example}{Example}%[section]         -> \begin{example} Consider the query from Example~\ref{exp:query}. Then... \end{example}
\newtheorem{Obs}{Observation}%[section]
\newtheorem{corollary}{Corollary}
\newtheorem{conjecture}{Conjecture}

%%\newtheorem{theor2}{Theorem \cite[Theorem 1.4, Exp. 1.7]{bigqueues}}[theor1]
%%\newenvironment{theor2}{\emph{}}{\addtocounter{\thetheor1}}
%%\newenvironment{theor2}{\refstepcounter{theor1}\emph{Theorem \arabic{theor1}~\cite[Theorem 1.4, Exp. 1.7]{bigqueues}:}}{}

%%% Proofs -- if not defined:
%\newcommand{\qedbox}{\vbox{\hrule\hbox{\vrule\squarebox{.667em}\vrule}\hrule}}
%\newcommand{\qed}{\nopagebreak\mbox{}\hfill\qedbox\smallskip}
%\newcommand{\sq}{\hbox{\rlap{$\sqcap$}$\sqcup$}}
%\newcommand{\qed}{\hspace*{\fill}\sq}  %empty square
%\newcommand{\qed}{\hfill \mbox{\raggedright \rule{.07in}{.1in}}}
%\newenvironment{proof}{\noindent {\bf Proof.}\ }{\qed\par\vskip 4mm\par}
%\newenvironment{proofof}[1]{\bigskip \noindent {\bf Proof of #1:}\quad }{\qed\par\vskip 4mm\par}
%\newenvironment{proofsketch}[1]{\smallskip {\em \quad Proof sketch: \quad}}{\qed\par\vskip 4mm\par}
%\newenvironment{proof-outline}{\noindent{\bf Proof Outline}\hspace*{1em}}{\qed\bigskip\\}

%======================================
%  Defining code related macros
%======================================
\newcommand{\code}[1]{\texttt{#1}}

%========================
%  Defining comments
%========================

\ifcomm
    % (1) comments in bold
    \newcommand{\A}[1]{\textbf{[#1]}}       %Comments -> in bold

    % (2) comments in footnote
    \newcommand{\F}[1]{\footnote {\color{red} [#1]}}

    % (3) comments in rectangle
    \newcounter{commentNumberAB}
    \setcounter{commentNumberAB}{0}
    \newcommand{\AB}[1]{\addtocounter{commentNumberAB}{1} \todo[inline]{\textbf{(AB.\arabic{commentNumberAB})} #1}}



\else
    \newcommand{\AB}[1]{}
\fi

% (3) change marks
\usepackage[normalem]{ulem}
\newcommand{\add}[1]{\textcolor{red}{EZ: \textbf{#1}}}
\newcommand{\del}[1]{\textcolor{red}{EZ: \textbf{\sout{#1}}}}

% (4) remove (hide) stuff
\newcommand{\remove}[1]{}


%========================
%  Defining operators
%========================

%%% Shortcuts: quoting equations etc.
%\newcommand{\eqref}[1]{Equation~(\ref{#1})}        % example: "We use the
% result from \eq{eq:iterated} to obtain..."
\newcommand{\thmref}[1]{Theorem~\ref{#1}}
\newcommand{\lemref}[1]{Lemma~\ref{#1}}
\newcommand{\deref}[1]{Definition~\ref{#1}}
\newcommand{\exref}[1]{Example~\ref{#1}}
\newcommand{\secref}[1]{Sect.~\ref{#1}}
\newcommand{\figref}[1]{Fig.~\ref{#1}}
\newcommand{\tabref}[1]{Table~\ref{#1}}
\newcommand{\appref}[1]{Appendix~\ref{#1}}

%%%% Shortcut to start a proof
%\newcommand{\bp}{\begin{IEEEproof}}     %IEEE formats: \begin{IEEEproof} ; \begin{IEEEproof}[Proof of Theorem \ref{thm:my}]; else: \begin{proof}
%\newcommand{\bpo}{ \begin{IEEEproof}[Proof Outline] }
%\newcommand{\ep}{\end{IEEEproof}}       %\newcommand{\ep}{\QED \end{proof}}

%%% Shortcut to start an equation
\newcommand{\be}{\begin{equation}}
\newcommand{\ee}{\end{equation}}

%%% Definining scheme names & text
\newcommand{\OPT}{\text{\textsc{opt}}}
\newcommand{\scheme}{{\langle a,d,c,h\rangle}}
\newcommand{\fpr}{{false positive rate }}   % be careful: "the \fpr of the scheme" works fine, but "it has a good \fpr, and" doesn't (bad space before comma); so we use the next line; check also "\fpr\"
\newcommand{\fprNS}{{false positive rate}} %see above: "it has a good \fprNS, and"

%%% Variables for figures
\newcommand{\tinygraphsize}{0.22} %alternating
\newcommand{\graphsizeA}{0.25}  %3 parallel subfigures <=0.31
\newcommand{\graphsizeB}{0.31}  %3 parallel subfigures <=0.31
\newcommand{\smallergraphsize}{0.24}  %4 parallel subfigures
\newcommand{\smallgraphsize}{0.35}
\newcommand{\figurewidth}{\columnwidth}

\newcommand\MyIncludeGraphics[2][]{% needs packages: \usepackage{graphicx}\usepackage{todonotes}
    \IfFileExists{#2}{%
        \includegraphics[#1]{#2}%
    }{%
        \missingfigure[figwidth=2.0cm]{}%{{\small Missing #2}}%
    }%
}%

\newcommand{\T}[1]{\noindent\textbf{#1}} %paragraph title
\newcommand{\U}[1]{\noindent\textit{#1}} %paragraph sub-title

%========================
%  Math
%========================

\newcommand{\para}[1]{\left( #1 \right)}        %Shortcuts in equations for parentheses
\newcommand{\brac}[1]{\left\{ #1 \right\}}
\newcommand{\set}[1]{\left\{#1\right\}}         %same as \brac
\newcommand{\sbrac}[1]{\left[ #1 \right]}
\newcommand{\floor}[1]{\left\lfloor #1 \right\rfloor}%\newcommand{\floor}[1]{\lfloor #1 \rfloor}
\newcommand{\ceil}[1]{\left\lceil #1 \right\rceil} %\newcommand{\ceil}[1]{\lceil #1 \rceil}
\newcommand{\norm}[1]{\left\Vert#1\right\Vert}
\newcommand{\abs}[1]{\left\vert#1\right\vert}
\newcommand{\reals}{\mathbb{R}}
\newcommand{\rationals}{\mathbb{Q}}
\newcommand{\integers}{\mathbb{Z}}
\newcommand{\naturals}{\mathbb{N}}
\newcommand{\eps}{\varepsilon}
\newcommand{\To}{\longrightarrow}
\newcommand{\partialderiv}[2]{\frac{\partial #1}{\partial #2}}
\newcommand{\calS}{{\mathcal{S}}} %defining a set
\newcommand{\tuple}[1]{\ensuremath{\langle \mbox{#1} \rangle}}

%========================
%  New Variable Macros
%========================
\newcommand{\newVar}[2]{\newcommand{#1}{\ensuremath{#2}\xspace}}

%========================
%  Define Variables
%========================
\newVar{\rkeys}{\mbox{r\textunderscore keys}}
\newVar{\rkey}{\mbox{r\textunderscore key}}
\newVar{\rversion}{\mbox{r\textunderscore version}}
\newVar{\wkey}{\mbox{w\textunderscore key}}
\newVar{\wversion}{\mbox{w\textunderscore version}}
\newVar{\robjs}{\mbox{r\textunderscore objs}}

%%% Defining argmax, Var, etc.
\DeclareMathOperator*{\argmax}{arg\,max}  % \argmax_x f(x) %also: \newcommand{\argmax}{\operatornamewithlimits{argmax}}
\DeclareMathOperator{\Var}{Var}     % Variance
\newcommand{\E}{\mathbb{E}}  %\newcommand{\E}{\text{E}} , \newcommand{\E}{\text{\textbf{E}}}       % Expected value
\newcommand{\eqtri}{\stackrel{\triangle}{=}}    %definition equality
\newcommand{\p}[1]{\Pr \para{#1}}  % Probability
%\renewcommand{\choose}[2]{\genfrac{(}{)}{0pt}{}{#1}{#2}}

\newcommand{\DL}{D_L}
\newcommand{\alg}{Alg}
\newcommand{\opt}{OPT
\newcommand{\w}{{\tilde{w}}}}
\DeclareMathOperator{\onoff}{ON-OFF}

\newcommand*{\Scale}[2][4]{\scalebox{#1}{$#2$}}%
\newcommand{\sm}[1]{\Scale[0.9] {#1}}  %reduce math size




%%% Addition to play with the bib file (1)
%%for citations; et al., etc.
%\makeatletter
%\def\bstctlcite{\@ifnextchar[{\@bstctlcite}{\@bstctlcite[@auxout]}}
%\def\@bstctlcite[#1]#2{\@bsphack
%\@for\@citeb:=#2\do{%
%\edef\@citeb{\expandafter\@firstofone\@citeb}%
%\if@filesw\immediate\write\csname
%#1\endcsname{\string\citation{\@citeb}}\fi}%
%\@esphack}
%\makeatother
%%

\begin{document}

\title{Speeding up Transaction Processing System using Local Transactions}
\numberofauthors{3}
\author{
\alignauthor
Idit Keidar\\
	\affaddr{Technion -- I.I.T}\\
	\email{idish@ee.technion.ac.il}
\alignauthor
Edward Bortnikov\\
	\affaddr{Yahoo Research}\\
	\email{ebortnik@yahoo-inc.com}
\alignauthor
Eshcar Hilel\\
	\affaddr{Yahoo Research}\\
	\email{eshcar@yahoo-inc.com}
\and %
\alignauthor
Ohad Shacham\\
	\affaddr{Yahoo Research}\\
 	\email{ohads@yahoo-inc.com}
 \alignauthor
 Aran Bergman\\
 	\affaddr{Technion -- I.I.T}\\
	\email{aranb@technion.ac.il}
}


%%% author names and affiliations
%%% use a multiple column layout for up to three different
%%% affiliations
%%\author{\IEEEauthorblockN{%\large
%%Great Student and Isaac Keslassy}
%%%\IEEEauthorblockA{Dept. of Electrical Engineering\\
%%%Technion\\% - Israel Institute of Technology\\
%%%Haifa 32000, Israel\\
%%\IEEEauthorblockA{Technion\\
%%\{gs@tx,isaac@ee\}.technion.ac.il }} %{\tt isaac@ee.techion.ac.il}
%
%\Conf{
%\author{
%\IEEEauthorblockN{\large Grad Student and Isaac Keslassy}
%\IEEEauthorblockA{
%Technion\\
%{\small \tt \{grad@tx, isaac@ee\}.technion.ac.il}}
%%\IEEEauthorblockN{Grad Student}
%%\IEEEauthorblockA{
%%Technion\\
%%{\small \tt grad@tx.techion.ac.il}}
%%\and
%%\IEEEauthorblockN{Isaac Keslassy}
%%\IEEEauthorblockA{
%%Technion\\
%%{\small \tt isaac@ee.techion.ac.il}}
%}
%}
%\TR{\author{\IEEEauthorblockN{\large Grad Student and Isaac Keslassy}}}

%%% This option goes with package authblk
%\author[1]{Grad Student}
%\author[1]{Isaac Keslassy}
%\author[2]{Company Engineer}
%\affil[1]{Technion, {\small \tt \{grad@tx, isaac@ee\}.technion.ac.il}}
%\affil[2]{Company, {\small \tt \{engineer\}@company.com}}


% For TR, remove for journal:
\OnlyTR{\markboth{Technical Report TR??-??, Technion, Israel}{}}

% make the title area
\maketitle

%For ACM formats
\sloppy

%=========================================================================
%  Abstract
%=========================================================================

\begin{abstract}
TBD
\end{abstract}

%\MyIncludeGraphics{fake.png}

\section{Introduction} \label{sec:intro}
Popular production systems such as Percolator, Omid, and Tephra, support
transactions over sharded data using centralized �timestamp oracles�, also
called \emph{transaction managers} (TMs) . The TMs provide a \emph{global
version clock} (GVC) [?] and sometimes additional functionalities such as conflict detection and
tracking of committed transactions[?]. The main motivation behind systems that
use a single TM is providing support for \emph{global transactions}, i.e., ones
that span multiple regions (also called nodes, shards, domains, or partitions).
Hence, central coordination is an inherent part of their design. This is in
contrast with federated systems, where transactions are by default local, i.e.,
access a single region, and protocols like two phase commit are used to ensure
atomicity of global transactions.

Using a centralized TM introduces a tradeoff: while transactions spanning
multiple regions are greatly facilitated and expedited using this design, local
transactions incur a performance penalty. This penalty is particularly
significant for short transactions, where the overhead of accessing the TM once
(to read the GVC) or twice (also in order to commit) per transaction is not
amortized across many operations. Typical Web workloads include both long
multi-region transactions and short local transactions, including ones
consisting of a single object read, write or read-modify-write.

In this work we mitigate the above tradeoff by offering a �fast path� for short
local transactions, without a significant impact on multi-region ones while
maintaining correctness. We consider a single-TM system that supports global
transactions, and enhance it with optimized support for specific types of local
transactions. Our main goal is to expedite short single-object transactions,
which are popular in production Web workloads [?].

We focus here on systems supporting \emph{snapshot isolation} (SI) [?], which is
popular in real-world systems [?] and amenable to scalable implementations [?].
The transaction semantics for local transactions are the same as for global
ones, and in particular, a local transaction can be performed using the usual
transaction mechanism, (which accesses the TM), with no change in semantics.

We implement our solution in Omid [?], an open source transaction processing
engine for key-value stores. Omid is database-neutral, while its open-sourced
implementation uses HBase. Our solution consists of two main parts - extensions
to the underlying key-value store, which we implement in HBase, and client-side
extensions to the transaction library, which we implement in Omid�s client
library. In addition, our implementation slightly modifies Omid�s TM to be
compatible with the changes to HBase.
 
Section 2 presents the context of this work, by outlining the data model,
transactional service, and design of existing single-TM systems we improve on.
Section 3 describes our support for local transactions in such systems. Section
4 presents our implementation and its evaluation. We review related work in
Section 5 and conclude with Section 6.

\section{Background, Motivation and Applicability}% and resulting benefits}
\label{sec:background}
We consider a database system that spans multiple regions and supports global
(multi-region) transactions with SI semantics using a single TM. Data is stored
in an underlying NOSQL key-value store, such as HBase, BigTable,  MongoDB, or
RocksDB. Clients access the data store directly, and partake in transaction
coordination using the assistance of the TM.

\section{Model and Notations}
We now describe the data model and API of the underlying data store.

\subsection{Objects}
\begin{enumerate}
\item We assume \emph{object} granularity; with HBase and BigTable this refers
to a row.
Later we discuss cell granularity.
Objects are associated with \emph{keys}.
\item Object values are associated with externally-provided monotonically
increasing \emph{version numbers}. Multiple versions associated with the same
key may co-exist.
\item Thus, an object is a tuple \tuple{key,\tuple{version,value}+}, where value
can be structured to consist of multiple columns.
\end{enumerate}

The data is \emph{partitioned} (sharded). Each object belongs to one region
(node, partition, shard, domain). \emph{Local transactions} are ones that access
a single region.

\subsection{API}
\begin{itemize}
\item \code{\tuple{version,value} read(key)} atomically returns the value with
the highest version associated with key along with the version.
\item The API further allows traversing (reading) earlier versions of the same
key in descending order.
\item \code{write(key,value,version)} atomically creates or updates the version:
\sout{If the provided version number is smaller than the key�s latest stored
version, the update fails.} if the version already exists, its value is updated;
otherwise, a new version is added. Garbage collection of obsolete versions is a separate
process.
\item Additionally, data stores often provide means to \emph{atomically read and
update} an object; (e.g., HBase exports CheckAndMutate operations, which is
internally implemented using a RW lock, whereas Bigtable supports row transactions). We
will extend this capability below in order to implement certain atomic
operations at the data store level.
\end{itemize}

\subsection{Transactions}
 Transactions may span multiple regions, and atomically commit or abort. The
 system supports transactions with SI semantics. Intuitively, SI ensures that
 the information a transaction reads from the data store does not mix old and
 new values. For example, if a task updates the values of two objects, then no
 concurrent transaction may observe the old value of one of these objects and
 the new value of the other. More precisely, SI enforces a total order on all
 committed transactions so that (1) transactions are ordered according to their
 commit times, (2) each transaction sees a consistent snapshot of the database
 reflecting exactly those transactions that committed prior to its start time;
 and (3) a transaction commits only if no updates it has made conflict with any
 concurrent updates made since that snapshot.

\section{Arcitecture and Design}
The TM provides a monotonically increasing GVC, which is used to produce
versions. An artifact of this is that object version numbers are monotonically
increasing system-wide.

Transactions indicate their intention to write to an object; this is done using
a dedicated column (part of the value in our data model), which keeps logical
(application-level) locks in some implementations (Percolator) and tentative
updates in others (Omid). While write intentions differ across implementations,
we assume that the following generic functions are provided:

\code{boolean writeIntent(key)} returns true if the latest version associated
with key has a write indication.

\code{boolean writeIntent(key, version)} returns true if version has a write
indication.

Note: in certain implementations (like a virtual lock per key used in
Percolator), the write-intent of all versions associated with the same key is
the same, whereas in other cases (like Omid, which uses tentative versions), a
new version may have a write intent while an older one does not.

A transaction goes through the following phases:
\begin{enumerate}
  \item{Begin} -- For SI, a transaction obtains a global read-version number
  when it starts. It also obtains a unique monotonically increasing transaction id.
  The two can be combined (i.e., the read-version can be the same as the
  transaction id).
  \item{Collect reads and indicate write intention} -- (either jointly or
  collect first and then write).\\ \emph{Collect} means reading object values
  and versions; each object atomically. Only versions committed with timestamps
  smaller or equal to the transaction�s read-version are read. Encountered write
  indications by uncommitted transactions are dealt with differently in
  different systems, e.g., Percolator waits for the indication to be lifted,
  Omid ignores uncommitted versions and checks for conflicts this may induce at
  commit time, and Distributed-Commit Omid forces the transaction with the write
  indication to abort. Collect occurs during the transaction (encounter time).\\
  \emph{Indicating write intention} changes the objects� write intention columns
  to indicate a transaction attempting to write them is under way. The value the
  transaction intends to write is added with a new version. In Omid, this
  version is the transaction�s id. Each object is updated atomically. This can
  occur either at encounter time or at commit time, in which case it occurs as
  the first phase of the commit. \\
  \emph{Validation/conflict detection} - For SI, need
  to check for write-write conflicts only. This step checks write intentions as well as version numbers.
  \item{Commit} --  in one atomic step to a designated \emph{commit entry}. May
  fail and abort instead. (Omid - commit table entry, Percolator - lock of first written key).
  \item{Clean-up} -- changes the write intentions of the transaction to
  persistent writes in case of commit, and removes them in case of abort. This
  phase occurs after the transaction is persistently committed or aborted, in
  order to reduce the overhead of future transactions and garbage collect
  obsolete information. Note that whenever a transaction encounters a write
  indication in the collect phase it must access the commit entry in order to
  check the transaction�s commit status. Once the clean-up phase is over, future
  transactions no longer incur this overhead for keys updated by the terminated
  transaction.
\end{enumerate}

\section{Adding Local Transactions}
We now explain our support for local transactions with SI semantics in the
context of a system as described in Section 2 above. We first define the API and
semantics of local transactions (Section 3.1) and then continue to describe our
solution, which consists of two parts. First, in Section 3.2, we enhance the
underlying data store with support for per-region local version clocks and an
API to manipulate this clock jointly with objects stored at that region. This
entails a minor modification to version management in the TM. Second, we add
client-side support for local transactions, as explained in Section 3.3.

\subsection{Local Transactions API and Semantics}
There are two distinctive characteristics of local transactions. First, they
must access a single region. Second, in contrast to regular transactions, they
cannot dynamically evolve, but rather must be provided in their entirety with a
single API-call. Local transactions APIs are prefixed with LTX. The simplest
examples are singleton transactions, i.e., transactions that perform a single
read operation via LTXread(key) or a single update via LTXwrite(key, value). A
more elaborate example is LTXMRSW(\wkey, \rkeys, f), which atomically reads
values associated  with a list of \rkeys and updates the value associated with
\wkey according to some function f of the read values. Note that this API
applies only in case \wkey and all \rkeys reside in the same region. In case
they do not, the call fails, and the transaction may be restarted as a regular
transaction.

The semantics for ordering local transactions relative to regular ones are
weaker than SI in that they do not guarantee real-time order over all regular
and local transactions together. Specifically, a regular transaction overlapping
two local transactions that access different regions may observe the updates of
the second and miss an update by the first. For example, assume objects x and y
are managed in two different regions, then real-time order can be violated as
follows:
{\small
\begin{verbatim}
x=0; y=0

  read(x)->0                        read(y)->1
|---------------------------------------------|
            write(x,1)      write (y,1) 
         |-------------|  |-------------|
\end{verbatim}}

The system still enforces a total order on all committed transactions, so that
(1) regular transactions (though not local ones) are ordered according to their
commit times; (2) each transaction sees a consistent snapshot of the database
reflecting a sequence of transactions that includes all those committed prior to
its start time plus any number of concurrent local transactions; and (3) a
transaction commits only if no updates it has made conflict with any concurrent
updates made since that snapshot.

\subsection{Data Store Extensions}
We add some functionality to a region, to be used exclusively by our client-side
extensions given in Section 3.3 below. The key mechanism used to support local
transactions is a local version clock (LVC) per region. Like the GVC, LVCs are
also monotonically increasing. Each region (domain, partition, shard) has its
own LVC. In Section 3.2.1 we describe this local clock�s API and its impact on
the GVC. We then proceed to present the extension of the data store API for
accessing objects together with the LVC in Section 3.2.2.

\subsubsection{Local version clock}
Each region�s LVC is loosely synchronized with the GVC. The idea is to use the
region�s LVC for ordering local transactions in any given region, and allow
local transactions to progress in different regions independently. Note that it
is safe to do so because no global order needs to be enforced among transactions
that access disjoint sets of objects.

Multi-region transactions, in turn, continue to obtain their versions from the
GVC. Therefore, whenever a multi-region transaction t accesses a given region,
we have to synchronize that region�s LVC with the GVC in order to ensure that
the version obtained by t exceeds those obtained by earlier completed
transactions within the region, and that later transaction within that region
will obtain higher versions than t�s.

To support such loose synchronization, the GVC now advances at a coarse
granularity of epochs. This can be implemented, for example, by choosing some
epoch size $2^l$, and keeping the l least significant bits of the GVC padded
with zeros. In other words, every increment of the GVC increases its value by
$2^l$.
The LVC obtains the epoch (n-l most-significant bits for an n-bit GVC) from the
GVC, and proceeds to assign timestamps within the designated epoch (by
incrementing the least significant bits).

The LVC has a single component, LVC.current, and it supports the following API:

\code{LVC.skip(epoch)} - atomically set LVC.current to max(epoch, LVC.current) .

\code{LVC.fetchAndIncrement()} - atomically increment LVC.current and return new
value.
For simplicity, we assume that the LVC is used so it does not overrun the epoch
and does not wrap around within the epoch. That is, fetchAndIncrement is called
less than $2^l$ times in each epoch.

\code{LVC.get()} - return LVC.current.

By incrementing the GVC, a multi-region transaction essentially initiates a new
epoch, and obtains a timestamp exceeding all those of older local transactions.
In addition, multi-region transactions enforce the synchronization of the LVC
with respect to the GVC using the skip operation. Specifically, whenever a
transaction in a new epoch accesses (for either read or write intention
indication) an object in a region whose LVC is still in an older epoch, it
invokes that region�s LVC.skip so it will not lag behind the transaction�s
read-timestamp obtained from the GVC. Thus, new local transactions that will
begin later in the region will have higher timestamps, as needed.

Note that transaction commits do not alter the LVC; the LVC only reflects
transactions� read-timestamps obtained when they  begin.

Note also that as long as a running transaction does not access a given region,
further updates can occur by local transactions in that region, and these
updates can be reflected in the transaction�s snapshot in case it later reads
these objects. Thus, unlike with regular transactions, which satisfy real-time
order, a transaction�s snapshot may reflect changes that occur after it
commences.  This may also lead to violation of real-time order in the commit
order of singletons, as in the example below:
{\scriptsize
\begin{verbatim}
Region 1, LVC=0     LVC=10
Region 2, LVC=0                      LVC=10
GVC = 10                                             GVC=12

    read-ts=10 read(x)->0            read(y)->1  commit
               x.skip(10)            y.skip(10)
|--------------------------------------------------------|

                      write(x,1)     write (y,1) 
                    |-----------|  |------------|

\end{verbatim}}

Here, x is written with LVC=10 and later y is written with its LVC=0. The
concurrent transaction�s snapshot time is 10, which includes the update of x and
not that of y. The same scenario does not occur with two local transactions
accessing the same region, since once the LVC is incremented, not further
updates in the same region can occur with older LVC values.

\subsubsection{Accessing data and the LVC together}
In order to allow local transactions to execute with a single data store access,
we extend the data store to support functions that access data objects and the
LVC together. Thus, a local transaction can increment LVC, obtain a version, and
write with this version, all in one round-trip to the local data store.

We further enforce atomic access to the data and the LVC. This is important in
order to avoid races between obtaining a version from the LVC and updating the
data. For example, if the updates of the LVC and the data were separate, the
following scenario could have arisen:
\begin{itemize}
  \item Local transaction l1 plans to update object A and obtains LVC value 2
  \item Multi-region transaction t1 obtains read version 2
  \item Local transaction l2 obtains LVC value 3 and updates object B
  \item Multi-region transaction t2 obtains read version 3
  \item t2 reads the old version of A (since it had not been written by l1 yet)
  and the new version of  B
  \item l1 writes A with 2 and completes
  \item t2 reads the new version of A and the old version of  B
\end{itemize}
Here, SI is violated.

In addition, the new functions must take care not to breach the the atomicity of
concurrent transactions. To this end, they rely on the write intention indications.

The new functions are:
\begin{enumerate}
  \item \code{boolean mutate(key, value)} atomically increments the region�s LVC
  and creates a new version for key associated with the new LVC value. The
  operation fails if the latest version of the object has a write intention
  indication. The pseudo-code for this operation is as follows:
\begin{verbatim}
atomically {
  if (writeIntent(key)) then return false
  version <- LVC.fetchAndIncrement()
  add <version, value> to key
  return true
}
\end{verbatim}
  \item \code{boolean validateAndMutate(\wkey, \wversion, value, [<\rkey,
  \rversion>*])}, where \robjs is an optional parameter.
Atomically validates that the provided versions of \wkey and \rkeys (if
applicable)  are equal to the highest ones in the data store and mutates the
object associated with \wkey and the LVC as mutate does. Returns true if the
validation is successful, otherwise returns false and does not perform the
mutation. The pseudo-code for this operation is as follows:
{\tiny
\begin{verbatim}
atomically {
  if (writeIntent(key)) then return false
  if (latest version of w_key != w_version) then return false
  forall (r_key, r_version) {
    if (latest version of r_key != r_version) then return false
  }
  version <- LVC.fetchAndIncrement()
  add <version, value> to w_key
  return true
}
\end{verbatim}} 
Note that the version numbers provided with the object parameters correspond to
old versions; they are used for validation, and are not stored with the new
version. The new version is produced using the LVC.
\item \code{< array of <version,value>, ts> collect(keys)}.   
Returns a consistent snapshot of keys. In particular, it selects a ts, which is the region�s LVC at some point during its execution, and returns the version with the highest version that does not have a write intention indication associated with each key up to this ts. In pseudo-code:
{\tiny
\begin{verbatim}
ts <- LVC.get()
foreach key in keys {
S = {<key, version, value> in data store | !writeIntent(key, version, value) ^
version <= ts} add argmax_S (version) to snapshot
}
return snapshot
\end{verbatim}}
\end{enumerate}

\subsection{Client-Side Support for Local Transactions}
We support a restricted set of local transactions, focusing on short ones that
frequently occur in production. Recall that short transactions benefit most from
our optimizations since for them, the overhead of accessing the TM once or twice
per transaction is significant.

Local transactions forgo the standard API including begin and commit operations,
and instead invoke a single function for executing the entire transaction. The
new API functions for local transactions are prefixed with \verb|LTX_|.

The most basic type of short local transaction we support is a singleton, i.e.,
a transaction consisting of a single read or write operation. Singletons are
implemented directly using the new functions in the underlying data store, as
follows:

\begin{verbatim}
LTX_read(key) 
  return collect(key).array

LTX_write(key, value)
  return mutate(key, value)
\end{verbatim}
The read operation can be trivially extended to a scan:

\begin{verbatim}
LTX_scan(keys)
  return collect(keys).array
\end{verbatim}

The next type of local transaction is a single-key read-and-write, SRW. It is
parametrized by some compute function f that generates the new value from the
old one.
{\small
\begin{verbatim}
LTX_SRW(key, f)
<version, value> <- read(key)
  new_val <- f(value)	// client side
  return validateAndMutate(key, version, new_val) 
\end{verbatim}}

Because \verb|LTX_SRW| does not perform the update to the data store atomically
with the read, it needs to validate the mutate operation to ensure that the object
has not been updated since its latest version was read at the beginning of the
operation. If the validation fails, the transaction can be retried, either the
same way or using a regular transaction.

We next  extend the above operation to read from multiple objects and update one:
{\scriptsize
\begin{verbatim}
LTX_MRSW(w_key, r_keys, f)  // keys is a non-empty list 
  <r_snapshot, version> <- collect(keys)
  new_val <- f(r_snapshot)		// client side
  return validateAndMutate(w_key, version, new_val, r_snapshot)
\end{verbatim}}

\section{Evaluation}


\section{Related Work}


\section{Conclusion}


\ifblind
\else
\section{Acknowledgments}

\fi

\bibliographystyle{abbrv}
\bibliography{mybib}


%%%%%%%%%%%%%%%%%%%%%%%%%%%%%%%%%%%%%%%%%%%%%%%%%%%%%%%%%%%%%%%%%%%%%%%%%%%%%%%%%%%%%%%%%%%%%%%%%%%%%%%%%%%%%%%%%%%%%%%%%%%%%%%%%%%%
% Appendix
%%%%%%%%%%%%%%%%%%%%%%%%%%%%%%%%%%%%%%%%%%%%%%%%%%%%%%%%%%%%%%%%%%%%%%%%%%%%%%%%%%%%%%%%%%%%%%%%%%%%%%%%%%%%%%%%%%%%%%%%%%%%%%%%%%%%

%\newpage

\begin{appendix}
	
	\section{Notations and definitions} \label{sec:app1}


\end{appendix}
%%%%%%%%%%%%%%%%%%%%%%%%%%%%%%%%%%%%%%%%%%%%%%%%%%%%%%%%%%%%%%%%%%%%%%%%%%%%%%%%%%%%%%%%%%%%%%%%%%%%%%%%%%%%%%%%%%%%%%%%%%%%%%%%%%%%


\end{document}
