\Yoni{make txid tsc nice like in the omid paper}
\begin{figure}[!t]
  \centering
  
  \begin{subfigure}[t]{\columnwidth}
      \centering
    \begin{tabular}{|c|c|c|c|c|}
      \hline
      key & value & version & commit& leader\\
      \hline
      \hline
      k1 & a & 3 & &\\
      \hline
      k2 & b & 3 & &k1\\
      \hline
      k2 & c & 3 & &k1\\
      \hline
    \end{tabular}
	\caption[]{Tentative transaction}
    \label{fig:model:tentative}
  \end{subfigure}
  
  \begin{subfigure}[t]{\columnwidth}
    \centering
    \begin{tabular}{|c|c|c|c|c|}
      \hline
      key & value & version & commit& leader\\
      \hline
      \hline
      k1 & a & 3 & 7&\\
      \hline
      k2 & b & 3 & &k1\\
      \hline
      k2 & c & 3 & &k1\\
      \hline
    \end{tabular}
	\caption[]{Committed transaction}
    \label{fig:model:committed}
  \end{subfigure}


  \begin{subfigure}[t]{\columnwidth}
    \centering
    \begin{tabular}{|c|c|c|c|c|}
      \hline
      key & value & version & commit& leader\\
      \hline
      \hline
      k1 & a & 3 & 7&\\
      \hline
      k2 & b & 3 & 7&\\
      \hline
      k2 & c & 3 & 7&\\
      \hline
    \end{tabular}
	\caption[]{Post-committed transaction}
    \label{fig:model:postcommit}
  \end{subfigure}

  
  \caption{Different stages of metadata during a transaction. The txid the transaction received from the begin stage is 3, and the tsc it received when committing is 7. The leader chosen for this transaction is k1 }
  \label{fig:model}
\end{figure}


In this section we describe \sys\ implementation to details.

\subsection{Transaction metadata}
\label{ssec:tso}
Figure~\ref{fig:model} shows the metadata of a transaction during its stages of execution. The \emph{version} field  is the txid of the transaction which is returned at the begin stage. The \emph{commit} field indicates whether the data is committed, and if it is, its commit timestamp (tsc). Since transaction commit needs to be an atomic step, one key from the transaction's write set is chosen to be the \emph{leader} of the transaction, and when the transaction is committed a single atomic write is initiated to the \emph{leader's} commit field with the tsc. Transactions' read operations refer to the keys \emph{leaders} to find out if the value has been committed or not, and if it is committed, they use the tsc to determine whether the value is in their snapshot.

To avoid an extra RPC to the leader for every read, after the transaction is committed, a post-commit stage writes the tsc to the \emph{commit} field of every key in the write set of the transaction. Before initiating an extra RPC to the leader, a read operation first reads the \emph{commit} field field of the key and only if it doesn't have any value the operation will read from the leader.

\subsection{TSO}
\label{ssec:tso}
Not sure if we need this. we already talk about it in ~\ref{ssec:ll-txns}
\paragraph{Begin}

\paragraph{Commit}

\subsection{Client API}
\label{ssec:api}

\paragraph{Begin}
When the client calls the begin API, an RPC is initiated towards the TSO to get a \emph{ts}. This ts is used later while reading/writing/committing.

\paragraph{Read}
\Yoni{remove this text and do this in an algorithm box}
As mentioned in Algorithm~\ref{alg:schema}, a read must return the highest committed version that is lower than the ts obtained during the begin stage. When a read is called, \sys\ will read the N latest versions smaller then ts of the key together with each versions shadow cell and leader cell. \sys\ will then iterate through the versions from newest to oldest and return the newest version that is not tentative. To check whether a key is committed or not, \sys\ will first check if its shadow cell has a value, if not, it will initiate a new read to the key's leader shadow cell. If the leader shadow cell has a value, the key is committed. \Yoni{talk about invalidating the leader}

\paragraph{Write}
A write 


\paragraph{Commit}

\paragraph{Post-commit}

\paragraph{Singleton Read}

\paragraph{Singleton Write}

--chart of steps while reading or writing and commiting
\Yoni{example of reading + jumping to shadow cell.post commit, checkandput HA}

\subsection{HBase region server}
\label{ssec:hbase}
