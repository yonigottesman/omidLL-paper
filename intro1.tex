% Transactions in big-data platforms
In recent years, transaction processing technologies have paved their way into big data 
platforms that scale to many petabytes of data~\cite{Spanner2012,Percolator2010,Omid2017}. 
In some cases, they are built into the storage system itself~\cite{Spanner2012} whereas in the 
others they are standalone services~\cite{Omid2017}. Transaction management complements 
the underlying key-value storage with {\em atomicity}, {\em consistency}, {\em isolation\/} and 
{\em durability} (ACID) semantics~\cite{Gray:1992:TPC:573304} that enable programmers perform 
complex data manipulation without over-complicating their applications. The need for transaction awareness 
in web-scale applications started from specific use cases like real-time content indexing~\cite{Percolator2010,
Omid2017} but quickly expanded to generic high-level abstractions, e.g., full-scale SQL OLTP and 
online analytics~\cite{Phoenix, F1-2013}.

Similarly to many technologies, the adoption of transactions took the "functionality-first" approach. 
For example, the developers of Google Spanner~\cite{Spanner2012} claim: ``We believe it
is better to have application programmers deal with performance problems due to overuse 
of transactions as bottlenecks arise, rather than always coding around the lack of transactions''. 
However, the expectation for high performance is picking up rapidly. For instance, the early-days  
transactions-enabled systems were throughput-oriented~\cite{Percolator2010, Omid2017}. 
With the thrust into new domains like messaging~\cite{Borthakur:2011} and algorithmic 
trading~\cite{opentsdb} that require interactive behavior, latency becomes king. 

Consider the platform powering Yahoo! Mail -- a service with hundreds of millions of users. 
The Mail backend ingests billions of messages daily. Every new message undergoes machine 
classification (e.g., spam filters, thread detection and ``smart view'' tagging\footnote{\footnotesize{\url{
https://yahoohelpcommunity.tumblr.com/post/118485031125/getting-to-know-smart-views-in-yahoo-mail}}}). 
The users can further manipulate their messages -- star, tag, and move between folders.    
They can browse and search for content; the volume of metadata and search requests 
is many billions per day.  Mail users expect a consistent experience -- messages never 
disappear, starred content gets prioritized in search, folder counters are reliable, etc.

The Yahoo! Mail metadata and search platform is built on top of Apache HBase~\cite{hbase}, 
which provides reliable and scalable key-value storage. The system's first generation was not 
transaction-enabled. This exposed the developers to very complex programming scenarios 
to achieve ACID behavior -- e.g., to guarantee consistent folder listings, or to provide 
atomic updates of multiple counters. Using a transaction API, e.g.,  
Apache Incubator Omid~\cite{omid}, makes their life much easier, however doing so 
must preserve the real-time latency SLA's for interactive user experience. For example, 
simple updates, point queries and small scans must complete within single-digit milliseconds. 
The original Omid implementation, which is designed for high-throughput data pipelines~\cite{Omid2017}, 
fails to match these requirements. % -- especially under high loads. 

Motivated by this example, we present {\sys\/} -- a low-latency {\em and\/} high-throughput 
transaction processing system. {\sys\/} dissipates the bottlenecks in the existing transaction 
managers to reduce the overhead of begin and commit operations. 
It especially favors short single-key transactions (read, write, and read-compute-write) -- 
a key use case in multiple latency-sensitive applications. {\sys}'s basic version is compatible 
with standard key-value store API's. The advanced version introduces a standalone API 
for short transactions that eliminates the begin/commit overhead entirely. This  
capability requires minor extensions of the underlying data store. 
 
Similarly to multiple modern transaction managers~\cite{cockroach,Spanner2012,Percolator2010,Omid2017},
{\sys\/} implements the {\em snapshot isolation} (SI) consistency model~\cite{DBLP:conf/sigmod/BerensonBGMOO95}
that scales better than the traditional lock-based serializability implementations. 

% \Idit{The roadmap below is not essential; maybe replace with summary of contributions.}
The remainder of this paper is organized as follows:
In Section~\ref{sec:api} we define the  API and semantics of a TPS. 
Section~\ref{sec:ll} presents \sys, without the fast path. 
Section~\ref{sec:alg} then describes our support for fast-path  transactions in \sys.  
Section~\ref{sec:eval} presents an empirical evaluation.
In Section~\ref{sec:context} we generalize the local transactions algorithm, and explain how it could be implemented in 
other TPSs. 
 We review related work in Section ~\ref{sec:related} and conclude with Section~\ref{sec:conclusions}.
