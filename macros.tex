%========================
%  Defining theorem types
%========================

\newtheorem{theorem}{Theorem}%[section]    -> example: \begin{theorem}   ... \end{theorem}
\newtheorem{theorem*}{Theorem}
\newtheorem{definition}{Definition}%[section] -> example: \begin{definition} [$B_{h}$ Sequence] \label{def:B} ..... \end{definition}
\newtheorem{proposition}{Proposition}%[section]
\newtheorem{property}{Property} %\newtheorem{property}[theorem]{Property}
\newtheorem{lemma}{Lemma}%[section]
\newtheorem{example}{Example}%[section]         -> \begin{example} Consider the query from Example~\ref{exp:query}. Then... \end{example}
\newtheorem{Obs}{Observation}%[section]
\newtheorem{corollary}{Corollary}
\newtheorem{conjecture}{Conjecture}


%======================================
%  Defining code related macros
%======================================
\newcommand{\code}[1]{\textsf{\fontsize{9}{11}\selectfont #1}}
%\newcommand{\code}[1]{\textsf{#1}}

%========================
%  Defining comments
%========================

% change marks
\usepackage[normalem]{ulem}
\newcommand{\add}[1]{\textcolor{red}{EZ: \textbf{#1}}}
\newcommand{\del}[1]{\textcolor{red}{EZ: \textbf{\sout{#1}}}}

% remove (hide) stuff, comments
\newcommand{\inred}[1]{{\color{red}{#1}}}
\newcommand{\remove}[1]{}
\newcommand{\idit}[1]{[[\inred{Idit: #1}]]}
\newcommand{\Idit}[1]{[[\inred{Idit: #1}]]}
\newcommand{\sys}{Lorra}

%========================
%  Defining operators
%========================



%%% Definining scheme names & text
\newcommand{\OPT}{\text{\textsc{opt}}}
\newcommand{\scheme}{{\langle a,d,c,h\rangle}}
\newcommand{\fpr}{{false positive rate }}   % be careful: "the \fpr of the scheme" works fine, but "it has a good \fpr, and" doesn't (bad space before comma); so we use the next line; check also "\fpr\"
\newcommand{\fprNS}{{false positive rate}} %see above: "it has a good \fprNS, and"

%%% Variables for figures
\newcommand{\tinygraphsize}{0.22} %alternating
\newcommand{\graphsizeA}{0.25}  %3 parallel subfigures <=0.31
\newcommand{\graphsizeB}{0.31}  %3 parallel subfigures <=0.31
\newcommand{\smallergraphsize}{0.24}  %4 parallel subfigures
\newcommand{\smallgraphsize}{0.35}
\newcommand{\figurewidth}{\columnwidth}

\newcommand{\T}[1]{\noindent\textbf{#1}} %paragraph title
\newcommand{\U}[1]{\noindent\textit{#1}} %paragraph sub-title

%========================
%  Math
%========================

\newcommand{\para}[1]{\left( #1 \right)}        %Shortcuts in equations for parentheses
\newcommand{\brac}[1]{\left\{ #1 \right\}}
\newcommand{\set}[1]{\left\{#1\right\}}         %same as \brac
\newcommand{\sbrac}[1]{\left[ #1 \right]}
\newcommand{\floor}[1]{\left\lfloor #1 \right\rfloor}%\newcommand{\floor}[1]{\lfloor #1 \rfloor}
\newcommand{\ceil}[1]{\left\lceil #1 \right\rceil} %\newcommand{\ceil}[1]{\lceil #1 \rceil}
\newcommand{\norm}[1]{\left\Vert#1\right\Vert}
\newcommand{\abs}[1]{\left\vert#1\right\vert}
\newcommand{\reals}{\mathbb{R}}
\newcommand{\rationals}{\mathbb{Q}}
\newcommand{\integers}{\mathbb{Z}}
\newcommand{\naturals}{\mathbb{N}}
\newcommand{\eps}{\varepsilon}
\newcommand{\To}{\longrightarrow}
\newcommand{\partialderiv}[2]{\frac{\partial #1}{\partial #2}}
\newcommand{\calS}{{\mathcal{S}}} %defining a set
\newcommand{\tuple}[1]{\ensuremath{\langle \mbox{#1} \rangle}}

%========================
%  New Variable Macros
%========================
\newcommand{\newVar}[2]{\newcommand{#1}{\ensuremath{#2}\xspace}}

%========================
%  Define Variables
%========================
\newVar{\rkeys}{\mbox{r\textunderscore keys}}
\newVar{\rkey}{\mbox{r\textunderscore key}}
\newVar{\rversion}{\mbox{r\textunderscore version}}
\newVar{\wkey}{\mbox{w\textunderscore key}}
\newVar{\wversion}{\mbox{w\textunderscore version}}
\newVar{\robjs}{\mbox{r\textunderscore objs}}

%%% Defining argmax, Var, etc.
%\DeclareMathOperator*{\argmax}{arg\,max}  % \argmax_x f(x) %also: \newcommand{\argmax}{\operatornamewithlimits{argmax}}
%\DeclareMathOperator{\Var}{Var}     % Variance
\newcommand{\E}{\mathbb{E}}  %\newcommand{\E}{\text{E}} , \newcommand{\E}{\text{\textbf{E}}}       % Expected value
\newcommand{\eqtri}{\stackrel{\triangle}{=}}    %definition equality
\newcommand{\p}[1]{\Pr \para{#1}}  % Probability
%\renewcommand{\choose}[2]{\genfrac{(}{)}{0pt}{}{#1}{#2}}

\newcommand{\DL}{D_L}
\newcommand{\alg}{Alg}
\newcommand{\opt}{OPT
\newcommand{\w}{{\tilde{w}}}}
%\DeclareMathOperator{\onoff}{ON-OFF}
