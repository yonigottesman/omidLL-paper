

The past few years have seen a growing interest in distributed 
transaction management~\cite{PattersonENAA12,Cowling2012,Aguilera2015,Balakrishnan2013,Thomson2012,eyal2013ordering,Warp}.
Recently, many efforts have been dedicated to improving performance using advanced 
hardware trends like RDMA and HTM~\cite{Wei2015,Dragojevic2014,Dragojevic2015}.  
These efforts are, by and large, orthogonal to ours.

Our work follows the line of industrial production systems, such as 
Google�s Spanner~\cite{Spanner2012}, Megastore~\cite{Megastore}, and Percolator~\cite{Percolator2010}, 
Yahoo�s Omid1~\cite{OmidICDE2014} and Omid~\cite{Omid2017}, 
Cask�s Tephra\cite{tephra}, and more~\cite{cockroach}.
These systems develop transaction processing engines on top of existing persistent 
highly-available data stores; for example, Megastore is layered on top of
Bigtable~\cite{Chang2008}, Warp~\cite{Warp} uses HyperDex~\cite{Escriva2012}, 
and CockroachDB~\cite{cockroach} uses RocksDB~\cite{rocksdb}.
Like  Omid~\cite{Omid2017}, we layer \sys\ atop Apache HBase~\cite{hbase}.

As discussed in Section~\ref{sec:ll}, a number of these systems follow a common paradigm
with different design choices, and \sys\ chooses a new operation point in 
this design space. In particular, \sys\ eliminates the bottleneck of Omid by
distributing the commit entry, makes commits and begins faster than Omid's by 
allowing reads to abort pending transactions, but unlike Percolator and CockroadDB, 
uses centralized conflict detection. This eliminates the need to detect client 
failures as in Percolator or have transactional writes perform conflict detection 
as in CockroachDB. 

As a separate contribution we developed a fast path for single-key transactions,
which is applicable to any of the aforementioned systems.
A similar mechanism was previously developed in Mediator~\cite{mediator}.
\inred{How Mediator differs from \sys\ FP.}

\remove{

Omid most closely resembles Tephra [6] and
Omid1 [25], which also run on top of a distributed keyvalue
store and leverage a centralized TM (sometimes
called oracle) for timestamp allocation and conflict resolution.
However, Omid1 and Tephra store all the information
about committed and aborted transactions in the
TM�s RAM, and proactively duplicate it to every client
that begins a transaction (in order to allow the client
to determine locally which non-committed data should
be excluded from its reads). This approach is not scalable,
as the information sent to clients can consist of
many megabytes. Omid avoids such bandwidth overhead
by storing pertinent information in a metadata table
that clients can access as needed. Our performance
measurements in Section 7 below show that Omid significantly
out-performs Omid1, whose design is very close
to Tephra�s. For high availability, Tephra and Omid1
use a write-ahead log, which entails long recovery times
for replaying the log; Omid, instead, reuses the inherent
availability of the underlying key-value store, and hence
recovers quickly from failures.
Percolator also uses a centralized �oracle� for timestamp
allocation but resolves conflicts via two-phase commit,
whereby clients lock database records rendering
them inaccessible to other transactions; the Percolator
paper does not discuss high availability. Other systems
like Spanner and CockroachDB allot globally increasing
timestamps using a (somewhat) synchronized clock
service. Spanner also uses two-phase commit whereas
CockroachDB uses distributed conflict resolution where
read-only transactions can cause concurrent update transactions
to abort. In contrast, Omid never locks (or prevents
access to) a database record, and never aborts due
to conflicts with read-only transactions.
The use cases production systems serve allow them
to provide SI [31, 25, 6, 5], at least for read-only transactions
[17]. It is nevertheless straightforward to extend
Omid to provide serializability, similarly to a serializable
extension of Omid1 [35] and Spanner [17]; it is merely
a matter of extending the conflict analysis to cover readsets
[24, 14], which may degrade performance.
A number of other recent efforts avoid the complexity
of two-phase commit [26] by serializing transactions using
a global serialization service such as highly-available
log [11, 23, 13] or totally-ordered multicast [15]. Omid
is unique in utilizing a single transaction manager to resolve
conflicts in a scalable way.

}