% Transactions in big-data platforms
In recent years, transaction processing technologies have paved their way into multi-petabyte big data platforms~\cite{Percolator2010,Spanner2012,Omid2017}. 
%In some cases, they are built into the storage system itself~\cite{Spanner2012} whereas in the others they are standalone services~\cite{Omid2017}. 
Modern industrial \emph{transaction processing systems (TPSs)}~\cite{Percolator2010, Omid2017, tephra, cockroach} complement 
existing underlying NoSQL key-value storage with {\em atomicity}, {\em consistency}, {\em isolation\/} and {\em durability} (ACID) semantics that enable programmers to perform complex data manipulation without over-complicating their applications. 
%Google's Percolator~\cite{Percolator2010} pioneered a transaction API atop the Bigtable storage. Apache 
%Incubator projects Tephra~\cite{tephra} and Omid~\cite{Omid2017} followed suit with Apache HBase~\cite{hbase}. 

Such technologies must  evolve  in  light of the recent paradigm shift towards massive deployment of  
big data services in public clouds; 
for example,  the AWS cloud can run HBase as part of  Elastic MapReduce, 
and SQL engines like Apache Phoenix increasingly  target public cloud deployment.
Yet adapting TPS  technology for cloud use is not without challenges, as we now highlight.  

\subsection{Challenges}

{\bf Diverse functionality.\ }
Transaction support was initially motivated by specific use cases like content indexing for 
web search~\cite{Percolator2010, Omid2017} but  rapidly evolved into a wealth of OLTP and  analytics 
applications (e.g.,~\cite{F1-2013}). Today,  users expect to manage diverse workloads within 
a single data platform, to avoid lengthy and error-prone extract-transform-load processes.
A Forrester report~\cite{Forrester2017} coins the notion of {\em translytics}  as ``a unified and integrated data platform that supports multi-workloads such as transactional, operational, and analytical simultaneously in real-time, ... and ensures full transactional integrity and data consistency''. 

As use cases become more complex, application developers tend to prefer high-level SQL abstractions to 
crude NoSQL data access methods. Indeed, scalable data management platforms (e.g., Google Spanner~\cite{Spanner2012}, Apache Phoenix, and CockroachDB~\cite{cockroach}) now provide full-fledged SQL interfaces  allowing complex query semantics  with strong data guarantees. SQL APIs raise new requirements for transaction management, e.g., in the context of maintaining secondary indexes for accelerated analytics. 

{\bf Scale.\ } 
Public clouds  are built to  serve a multitude of applications 
with elastic resource demands, and their efficiency is achieved through  scale. 
Thus, cloud-first data platforms are  designed to scale well beyond the limits of a single application. 
For example, Phoenix is designed to scale to 10K query processing 
nodes in one instance, and is expected to   process hundreds
 of thousands or even millions of transactions per second ({\em tps}). 
 
{\bf Latency.\ }
Whereas early applications of big data transaction processing systems were 
mostly throughput-sensitive \cite{Percolator2010, Omid2017}, 
with the thrust into new interactive domains like social networks, %~\cite{chatter},  
messaging %~\cite{Borthakur:2011}
 and algorithmic trading %~\cite{opentsdb} 
latency becomes essential.  
SLAs for interactive user experience mandate that simple updates and point queries  complete within 
single-digit milliseconds. The early experience with Phoenix shows that programmers  often
refrain from using transaction semantics altogether for fear of high latency, 
and instead  adopt solutions that may compromise consistency.

{\bf Multi-tenancy.\ } Data privacy is key in systems that host data owned by multiple applications.
Maintaining  access rights is therefore an important design consideration for TPSs, 
e.g., in maintaining shared persistent state. 

% Drum roll
\subsection{Evolving Omid for public clouds}

This paper describes recent functionality extensions and performance improvements 
we made in Omid -- an open source transaction manager for HBase -- to power Phoenix\footnote{\url{https://phoenix.apache.org}},  
a Hadoop SQL-compliant  data analytics platform designed 
for cloud deployment. 

In contrast to earlier SQL  engines for Hadoop (e.g., 
Hive and Impala), which focused on immutable data, Phoenix targets converged data ingestion and analytics~\cite{PhoenixUseCases},
which necessitates transactions, and its  
applications are both latency and throughput sensitive. Phoenix uses HBase 
as its key-value storage layer, and Omid for transaction processing\footnote{{Tephra 
is supported as alternative transaction manager, though its scalability and reliability are inferior to Omid's.}}.

This paper describes two concrete contributions to Omid and Phoenix. 
First, Section~\ref{sec:ll} describes the 
protocol's re-design for low-latency.  
The new protocol,    \emph{Omid Low Latency (\sysll)}, 
dissipates Omid's major architectural bottleneck. 
It reduces the latency of short transactions by 5x under light load, and 
by 10x--100x under heavy load. % compared to the legacy Omid. 
It also scales 
the overall system throughput to 550K tps while remaining 
within real-time latency SLAs. In contrast to previously published protocols 
(e.g.,~\cite{Percolator2010}), our solution is amenable to multi-tenancy.
The development of this new feature is reported in \url{https://issues.apache.org/jira/browse/OMID-90}.
%~\cite{OMID-90}.

Second, Section~\ref{sec:sql} discusses our {\em SQL compliance}. 
We add support for creating secondary indexes on-demand, 
without impeding concurrent database operations or sacrificing consistency. 
We further extend the traditional SI model, which provides single-read-point-single-write-point semantics, with multiple read and write points. This functionality is used to avoid recursive read-your-own-writes scenarios in complex queries. 
The development of these new features in Omid and Phoenix is reported in
\url{https://issues.apache.org/jira/browse/OMID-82} and 
\url{https://issues.apache.org/jira/browse/PHOENIX-3623},
%~\cite{OMID-82} and~\cite{PHOENIX-3623}, 
respectively.


% Finally, Section~\ref{sec:related} reviews related work, and  Section~\ref{sec:conclusions} concludes. % the paper.


\remove{
\begin{figure}
  \centerline{
        \includegraphics[width=0.75\columnwidth]{figs/OmidVersions}	
  }
\caption{
%Our extensions of Apache Incubator Omid. 
              We extend the  Apache Incubator version of Omid to support APIs  for SQL transactions in Phoenix. 
              In a complementary branch,
              we develop a low latency version \sysll, and extend it to support fast-path transactions in \sys.
              We also implement a variant based on two phase commit, \syspc, for comparison.}
\label{fig:evolution}
\end{figure}
}

\remove{
The performance  improvement  is achieved in two steps, as depicted in the middle row in  Figure~\ref{fig:evolution}.  
%by two design changes.
First, in Section~\ref{sec:ll}, we create a new low latency version of Omid, \sysll, which
dissipates Omid's principal bottleneck. It  
distributes persistent writes of transaction commit records, which is a centralized bottleneck in Omid. 
\sysll\ uses Omid's centralized transaction manager for timestamp allocation and conflict detection, 
but relieves it of performing I/O. 
%
For comparison, we implement a variant of Omid that uses Percolator's \emph{two-phase commit (2PC)} approach to 
conflict detection~\cite{Percolator2010}, which we call \syspc. 
\inred{
However, the Percolator approach  assumes that all transactions have permission 
to access all data tables, which is common in single-application settings, but is often not the case 
in multi-tenancy scenarios. \sysll\  instead, follows Omid's approach of storing conflict information in a common \emph{commit table}, which does  not  hold application data and is exclusively accessed by the transaction management layer.
}

Our second performance improvement is motivated by the observation that, in many production settings, single-key transactions are prevalent. \inred{For example, an in-house analysis of transactions processed by a major e-commerce company revealed that $70\%$ of transactions affected a single database key.} 
In Section~\ref{sec:alg} we introduce a novel \emph{fast path} algorithm for short single-key transactions, which 
executes short transactions without the begin/commit overhead,  almost as fast as native HBase operations. 
%This entails minor extensions to the underlying data store. 
We call the resulting system \sys. 
Note that the fast path is orthogonal to other protocol aspects, and can be supported in other 
transaction processing services. 
%, to enable local commits directly within the storage layer and verify the correctness of general 
%transactions in presence of such commits. 
\inred{We discussed the fact that it uses coprocessor and maybe say it is experimental, but neither seems essential to elaborate at the introduction level.}
}

\remove{
%We have implemented {\sys\/} %\footnote{\url{https://github.com/yonigottesman/incubator-omid/tree/localTransactions}} 
Our implementations are 
based on the open source Omid code\footnote{\url{https://omid.incubator.apache.org}}.
To support \syspc\ and \sys, we further extended  HBase  to enable locking and the fast path, resp. 
%\footnote{\url{https://github.com/yonigottesman/hbase_local_transactions/tree/0.98-add-rmw}}. 
Our code is publicly available on github\footnote{\small{Omitted for blind review.}}
and since \sysll\ does not require HBase extensions.
\inred{We hope to contribute it back to the Omid open-source project.}
%and we are currently productionalizing \sysll, (which does not require HBase extensions),  
%and contributing the changes back to the Omid open-source project.
 }
 
 \remove{
 \inred{ 
 During our integration of Omid in Apache Phoenix, we came across the need to extend Omid's API in order
 to support certain SQL features, such as index building and  snapshots (Ohad, what was the name for that?). 
Section TBD describes the functionality extensions added in order to integrate Omid in Apache Phoenix. These extensions are currently implemented in a separate branch that has been integrated in Phoenix. They are complementary to the performance improvements of \sysll\ and \sys, which we also plan to integrate into the Phoenix branch in the future.
}
 
 Section~\ref{sec:eval} presents our experiments on mid-range hardware, which show substantial performance improvements.
Under low load, \sysll\ transactions are 4x to 5x faster than Omid's, 
and \sys\ further reduces the latency of short transactions by another $55\%$ on average.
As system load increases, Omid's latency surges (at  $\sim\!\!\!150$K tps in our tests),  
whereas \sys's remains stable. \syspc\ performs similarly to \sysll, except on long transactions
where it is slower; e.g., on $10$-key transactions \syspc\ is $30\%$ slower than \sysll.
% until we generate a higher load ($\sim\!\!500$K tps)
%, and increases four-fold at 500K tps.  
Fast path transactions incur no scalability
bottlenecks, and execute at the low latency of native HBase operations regardless of system load
(thanks to HBase's near perfect horizontal scalability).
This comes at the cost of a minor (15-25\%) negative impact on longer transactions. 
%{\inred{The system scales beyond 1M transactions per second on medium-end hardware,
%which surpasses Omid at least 4x.}} 
Additionally, \sys\ has negligible impact on  transaction abort rates.

\remove{
% \Idit{The roadmap below is not essential; maybe replace with summary of contributions.}
%The remainder of this paper is organized as follows:
In Section~\ref{sec:api} we define the  API and semantics of a transaction processing service. 
Section~\ref{sec:ll} describes \sysll, and 
%Section~\ref{sec:ha} discusses its high availability mechanism. 
Section~\ref{sec:alg} presents \sys.  Our evaluation is reported in 
Section~\ref{sec:eval}.
%In Section~\ref{sec:context} we generalize the fast path algorithm, and explain how it could be implemented in other systems. 
}

We review related work in Section~\ref{sec:related} and conclude with Section~\ref{sec:conclusions}.
In summary, our contributions are as follows:
\begin{enumerate}
    \setlength{\itemsep}{1pt}
    \setlength{\parskip}{1pt}
    \setlength{\parsep}{1pt}  
\item We extend Omid with functionality that facilitates implementing a SQL API and integrate it in Apache Phoenix.
\item We implement \sysll, a low-latency distributed transaction engine \inred{amenable to multi-tenancy}. 
%we plan to contribute it  to the Apache Omid project.
%which improves latency by 5x and doubles throughput; 
%we are working to incorporate in  production;
%\item We explore the design space of transaction processing engines for NoSQL data stores, 
%highlighting the tradeoffs between various design choices.  
\item We design a novel fast path for short transactions and implement it in \sys.
\item We extensively evaluate the new algorithms.
\end{enumerate}
}
