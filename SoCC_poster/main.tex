\documentclass[letterpaper,twocolumn,10pt]{article}
\usepackage{times}

% do not change these values
\baselineskip 12pt
\textheight 9in
\textwidth 6.5in
\oddsidemargin 0in
\topmargin 0in
\headheight 0in
\headsep 0in


\usepackage{epsfig,endnotes}

%========================
%  Packages
%========================

\usepackage{graphicx,url,color}
\usepackage{amsmath}
\usepackage{amssymb}
%\usepackage{amsthm}  %<---- for a different "look" in theorems (theorem word in bold, etc.) %ACM conflict with proof definition?
%\usepackage{subfigure}
%\usepackage[tight,footnotesize]{subfigure}
\usepackage{algorithm}
\usepackage[noend]{algpseudocode}
\usepackage{times}

%\usepackage{todonotes}
%\usepackage[normalem]{ulem} %strikethrough: \sout{Hello World}
%\usepackage{lastpage} %for number of pages
%\usepackage{xspace}
%\usepackage{multirow}
%\usepackage{balance}

\usepackage[colorlinks=true,allcolors=blue,breaklinks]{hyperref}   % hyperlinks, including DOIs and URLs in bibliography
\usepackage{subcaption}
\usepackage{graphicx}
%\usepackage{caption}
\usepackage{float}

%========================
%  Macros
%========================

\newcommand{\code}[1]{\textsf{\fontsize{9}{11}\selectfont #1}}

\newcommand{\inred}[1]{{\color{red}{#1}}}
\newcommand{\remove}[1]{}
\newcommand{\Idit}[1]{[[\inred{Idit: #1}]]}
\newcommand{\Yoni}[1]{[\inred{Yoni: #1}]}
\newcommand{\tb}{\hspace{5mm}}

\newcommand{\sys}{Fragola}


\newcommand{\speedup}[1]{#1$\times$}
\newcommand{\tuple}[1]{\ensuremath{\langle \mbox{#1} \rangle}}

%========================
\begin{document}

\date{}

\title{\sys: Low-Latency Transactions in Distributed Data Stores}

\author{
{\rm Yonatan Gottesman\footnotemark[1]\tb Aran Bergman\footnotemark[2]\tb   Edward Bortnikov\footnotemark[1] }\\ 
{\rm Eshcar Hillel\footnotemark[1]\tb Idit Keidar\footnotemark[1] \footnotemark[2]\tb Ohad Shacham\footnotemark[1]}\\
	\footnotemark[1] Yahoo Research\ \ \footnotemark[2] Technion  \\ [2mm]
} % end author


\maketitle



%=========================================================================
%  Abstract
%=========================================================================

\remove{
\begin{abstract}
We present \sys, a highly scalable transaction processing engine 
for Apache HBase. 
The \sys\ implementation is based on Apache Incubator Omid, but 
improves its protocol to reduce latency 
while at the same time improving throughput.
Under light load, \sys\ is 4x to 5x faster than Omid, and under high loads, it is 
more than an order of magnitude faster. 
 \sys\ further implements a \emph{fast path} for single-key transactions,
processing them almost as fast as native HBase operations, while preserving
transactional semantics relative to longer transactions.

\end{abstract}
}
%========================

As transaction processing services begin to be used in new application domains, low 
transaction latency becomes an important consideration. 
Motivated by such use cases we developed \sys, a highly scalable 
low-latency  {\em and\/} high-throughput  transaction processing engine for Apache HBase. 
Similarly to other modern transaction 
managers, 
{\sys\/} provides a variant of \emph{generalized snapshot isolation (SI)},
which scales better than traditional serializability implementations. 

{\sys\/} is based on Apache Omid~\cite{Omid2017}, 
but dissipates the principal bottleneck present therein.
%in the prior implementations the overhead of begin and commit operations. 
Its advantage is maximized for short  transactions, which are prevalent in latency-sensitive applications.
\sys\ processes  such transactions in a handful of milliseconds. The new protocol  also doubles the system 
throughput.
%Moreover, it is amenable to a simpler high availability solution than the original design. 
%, which is \inred{an order of magnitude higher than} previously achievable limits. 
%Different aspects of our protocol are inspired by other systems, while their combination is novel.
%, to the best of our knowledge. 

As a separate contribution, we introduce a novel {\em fast path\/} algorithm for short single-key transactions 
that eliminates the begin/commit overhead entirely, and executes short transactions 
 almost as fast as native HBase operations. This entails minor extensions to the underlying 
data store. The fast path is orthogonal to other protocol aspects, and can be  supported in other 
transaction processing services. 
%, to enable local commits directly within the storage layer and verify the correctness of general 
%transactions in presence of such commits. 

We have implemented {\sys\/}\ %\footnote{\url{https://github.com/yonigottesman/incubator-omid/tree/localTransactions}} 
based on the open source Omid code\footnote{\url{https://omid.incubator.apache.org}}, 
and extended the HBase code to enable fast path 
transactions\footnote{\url{https://github.com/yonigottesman/hbase_local_transactions/tree/0.98-add-rmw}}. 
Our experiments on mid-range hardware show substantial performance improvements.
For example, Figure~\ref{fig:tl-1} shows the performance of single-operation transactions using 
Omid and two variants of \sys: Vanilla \sys, which does not include the fast path support,
and FP \sys, which includes the fast path. We can see that under low load (left side of the graph),
even without the fast path, \sys\ transactions are 4x to 5x faster than Omid's, 
and the fast path further reduces the latency of short transactions by another $55\%$ on average.
As system load increases, Omid's latency surges (at  $\sim\!\!\!150$K tps in our tests),  
whereas \sys's remains stable until we generate a higher load ($\sim\!\!250$K tps), and 
increases four-fold at 500K tps.
%; and the fast path further improves latency by 2x-5x. 
Overall fast path transactions incur no scalability
bottlenecks, and continue to execute at the low latency of native HBase operations regardless of system load
(thanks to HBase's near perfect horizontal scalability).
This comes at the cost of a minor (less than 15\%) negative impact on longer transactions. 
%{\inred{The system scales beyond 1M transactions per second on medium-end hardware,
%which surpasses Omid at least 4x.}} 
Additionally, \sys\ has negligible impact on  transaction abort rates.


\begin{figure}[htb]
	\centering
      	\includegraphics[width=0.48\textwidth]{figs/throughputlatency1.pdf}
	    \caption{Throughput vs.\ latency, transaction size = 1. Vanilla \sys\ does not include the fast path.}
        \label{fig:tl-1}      
\end{figure}

The \sys\ code is publicly available\footnote{\url{https://github.com/yonigottesman/incubator-omid/tree/localTransactions}}, and 
we are working to contribute it back to Omid. 
We believe our improvements will become instrumental for modern cloud-based, large-scale, 
latency-sensitive services, e.g., the Phoenix OLTP system, which is designed 
to harness up to ten thousand nodes.

%\subsection*{Acknowledgments}

\bibliographystyle{abbrv}
\bibliography{refs}

\end{document}
