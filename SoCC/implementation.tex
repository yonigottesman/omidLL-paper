\subsection{\sys\ algorithm} 
\label{ssec:ll}

\begin{figure}[!t]
  \centering
  
  \begin{subfigure}[tb]{\columnwidth}
      \centering\small
    \begin{tabular}{|c|c|c|c|c|}
      \hline
      key & value & version & commit& leader\\
      %\hline
      \hline
      k1 & a & 3 & nil &\tuple{k1,3}\\
      %\hline
      k2 & b & 3 & nil &\tuple{k1,3}\\
      %\hline
      k3 & c & 3 & nil &\tuple{k1,3}\\
      \hline
    \end{tabular}
	\caption[]{Pending transaction}
    \label{fig:model:tentative}
  \end{subfigure}
  
  \begin{subfigure}[t]{\columnwidth}
    \centering\small
    \begin{tabular}{|c|c|c|c|c|}
      \hline
      key & value & version & commit& leader\\
      \hline
      %\hline
      k1 & a & 3 & 7&\tuple{k1,3}\\
      %\hline
      k2 & b & 3 & nil &\tuple{k1,3}\\
      %\hline
      k3 & c & 3 & nil &\tuple{k1,3}\\
      \hline
    \end{tabular}
	\caption[]{Committed transaction}
    \label{fig:model:committed}
  \end{subfigure}


  \begin{subfigure}[tb]{\columnwidth}
    \centering\small
    \begin{tabular}{|c|c|c|c|c|}
      \hline
      key & value & version & commit& leader\\
      \hline
      %\hline
      k1 & a & 3 & 7&\tuple{k1,3}\\
      %\hline
      k2 & b & 3 & 7&\tuple{k1,3}\\
      %\hline
      k3 & c & 3 & 7&\tuple{k1,3}\\
      \hline
    \end{tabular}
	\caption[]{Post-committed transaction}
    \label{fig:model:postcommit}
  \end{subfigure}

  
  \caption{Evolution of \sys\ metadata during a transaction. The transaction receives $ts_r=3$ in the begin stage, 
  and $ts_c=7$ when committing; its leader is \tuple{k1,3}.}
  \label{fig:model}
\end{figure}



\begin{algorithm}[htb]
\begin{algorithmic}[1]

\Procedure{begin}{}
\State checkRenew()
\State return Clock.fetchAndIncrement()
\EndProcedure
\Statex
\Procedure{commit}{$txid$, write-set}
\State checkRenew()
\State $ts_c \leftarrow$ Clock.fetchAndIncrement()
\If{conflictDetect($txid$, write-set, $ts_c$) } 
 \State return $ts_c$
 \Else
 \State return {\sc abort}
\EndIf
\EndProcedure
\Statex
%
\Procedure{checkRenew}{} \Comment HA support
\If{lease $<$ now + $\delta'$} \label{l:lease-start} \Comment extend lease
 \State renew lease for $\delta$ time \Comment atomic operation
 \If{failed} halt \EndIf 
\EndIf  \label{l:lease-end}
\If{Clock $=$ epoch}  \label{l:epoch-start} \Comment extend epoch
\State epoch $\leftarrow$ Clock +  range
 
\If{$\neg$CAS(maxTS, Clock, epoch) } halt \EndIf 
\EndIf  \label{l:epoch-end}
\EndProcedure

\end{algorithmic}
\caption{\sys's TM algorithm with HA support.}
\label{alg:ha}
\end{algorithm}

\begin{algorithm}[t]
  \begin{algorithmic}
    \small
      \Procedure{read}{key} 
      \For{rec $\leftarrow$ ds.get(key, versions down from $ts_r$)}
		\If{rec.commit $\not \in\{$nil, abort$\}$} \Comment committed
     			 \If{rec.commit $< ts_r$}  \State return rec.value \EndIf
      		\ElsIf{rec.commit = nil}  
 			\State $ts_c \leftarrow$ {\sc checkLeader}(rec.leader)
   			\State rec.commit $\leftarrow ts_c$ \Comment helping 
 			 \If{$ts_c \not=$ abort  $\wedge\ ts_c <  ts_r$}  \State return  rec.value \EndIf
      		\EndIf
      		


      \EndFor
      \State  return nil
      \EndProcedure

\Statex

%      		\Procedure{checkLeader}{\tuple{k, ts}} 
%     			\State leader $\leftarrow$ ds.get(\tuple{k, ts})
%    			 \If{leader $=$ nil} return abort \Comment{leader removed} \EndIf 
%		     	 \If{ leader.commit $\not=$nil} \Comment transaction is complete
%		     	 	\State return leader.commit 
%		     	 \EndIf
%		     	 \Comment try to abort pending transaction 
%			\State ok $\leftarrow$ ds.check\&mutate(leader.commit, nil, abort)
%			\If{$\neg$ok} \Comment leader status has changed -- recheck  
%				\State return  {\sc checkLeader}({leaderPtr}) \EndIf
%			\State return abort
%	   	\EndProcedure

      		\Procedure{checkLeader}{\tuple{k, ts}} 
		     	 \While{true} 
      				\State leader $\leftarrow$ ds.get(\tuple{k, ts})
     				 \If{leader $=$ nil} return abort \Comment{leader removed} \EndIf 		     	 	
     				 \If{leader.commit $=$ nil}  \Comment{abort transaction} 
					\State ds.check\&mutate(leader.commit, nil, abort)
				\Else\ return leader.commit \Comment transaction is complete 
%					\State return leader.commit \Comment transaction is complete
				\EndIf 		     	 	
		     	 \EndWhile
	   	\EndProcedure
\Statex

\Procedure{commit}{$ts_r$, write-set}
      
      	\State $ts_c \leftarrow$ TM.commit($ts_r$, write-set) \Comment may return abort
      	\If{$ts_c \not=$ abort}
		\State leader  $\leftarrow$ getLeader(write-set)
	      	 \State  ok $\leftarrow$ ds.check\&mutate(leader.commit, nil,  $ts_c$)
   		\If{$\neg$ok} 
   		 $ts_c \leftarrow$ abort \EndIf
	\EndIf
	\State \Comment post-commit
	\ForAll{keys k $\in$ write-set}
			\If{$ts_c =$ abort} ds.remove(\tuple{k, $ts_r$})  	
%			\Else\ update  commit field of \tuple{k, $ts_r$} to $ts_c$  in ds%  $\leftarrow ts_c$ 
			\Else\ ds.update(\tuple{\tuple{k, $ts_r$}, commit}, $ts_c$)
			\EndIf
	\EndFor
\EndProcedure
      
  \end{algorithmic}
  \caption{\sys's read and commit operations.} 
  %for transaction with read timestamp $ts_r$.}
  \label{fig:get-pseudocode}
\end{algorithm} 



The metadata used by \sys\ for transaction management is stored in two additional columns associated with each object version in the data store:
(1) \emph{commit} holds the status of the transaction that wrote this version -- \emph{nil} if it is pending, its  $ts_c$ if committed, and 
otherwise \emph{abort}; 
and
(2) \emph{leader} is a pointer to the writing transaction's commit entry, namely, the \emph{commit} column of the first key written by the same
transaction. 

Since the commit needs to be an atomic step, the first key in the transaction's write-set is chosen to be the leader, and the transaction commits via a single atomic write of $ts_c$ to the leader's commit column. Transactional read operations that find a nil commit entry 
refer to the leader to find out whether the value they are attempting to read has been committed.
Figure~\ref{fig:model} shows the metadata of a transaction during its stages of execution. 

\sys's TM is a simplified version of Omid's TM, and appears  in Algorithm~\ref{alg:ha}. It has two roles:
First, it allocates begin and commit timestamps by fetching-and-incrementing a monotonically increasing global clock.
Second, it performs validation (i.e., conflict detection) upon commit using an in-memory hash table, 
and updates the table with the write-set of the new transaction and its $ts_c$. 
Since the conflict detection algorithm has been reported elsewhere~\cite{Omid2017}, and the \sys\ client need not be aware of 
its implementation details, we refrain from repeating the details of the \emph{conflictDetect} function here. 
The  \emph{checkRenew} procedure supports the TM's  high availability, and is explained in the next section.

Algorithm~\ref{fig:get-pseudocode} describes \sys's implementation of read and commit operations.
%Begin and write operations follow the template of Algorithm~\ref{alg:schema}. 
The client's operations proceed as follows:

\paragraph{Begin.}
The client sends a begin request to the TM, which assigns it a read timestamp $ts_r$.

\paragraph{Write.}
The client adds a tentative record to the data store, with the written key and value, version number $ts_r$, 
nil in the commit column, and a pointer to the first key written during the transaction in the leader column.
It also tracks key in its local write-set.


\paragraph{Read.}
The algorithm traverses data  records (using the data store's ds.get API) pertaining
to the requested key with a version that does not exceed $ts_r$, latest to earliest, and returns the first value that is committed
with a  version smaller than or equal to $ts_r$. Upon
encountering a tentative record (with commit=nil), the algorithm calls the {\sc checkLeader} function, which
gets the leader's $ts_c$ from the data store, and again returns the key if that $ts_c$ is smaller than $ts_r$. 

If the leader's commit entry is still nil, read cannot return without determining the final commit timestamp
of the pending transaction. 
%\Yoni{example here?}. 
To this end, {\sc checkLeader} attempts to forcefully abort the pending transaction. This is done using the
 {check\&mutate} operation, which atomically reads and sets the commit entry to abort in case its value is still nil.
If check\&mutate fails, this means that the transaction is no longer pending, and the next iteration in the loop 
%{\sc checkLeader} is recursively called to
re-reads the leader's entry. Note that during the second iteration, the record can no longer be nil, and so it
always returns without attempting another check\&mutate.
 
%When a transaction commits, it atomically checks it hasn't been invalidated before inserting the $ts_c$ to the commit .


\paragraph{Commit.}
The client sends a commit request to the TM, which assigns it a commit timestamp $ts_c$ and checks for conflicts. 
If the TM does not return abort, this means that the validation has been successful, and the client 
attempts to commit the transaction. 
Note that the transaction can commit only if no read has invalidated it. Therefore, the client uses 
the data store's check\&mutate function to write $ts_c$ to the leader's commit column 
only if the commit status is still nil.

To avoid an extra read of the leader on every transactional read, once a transaction is committed, the post-commit stage writes 
the transaction's $ts_c$ to the commit columns of all keys in  the transaction's write-set. 
%Following a successful commit, the client adds $ts_c$ to all data items it wrote to.

\paragraph{High availability.} The TM is implemented as a primary-backup process pair to ensure its high availability. 
The backup detects the primary's failure using timeouts. When it detects a failure, it immediately begins serving new
transactions, without any recovery procedure. Because the backup may falsely suspect the primary because of
unexpected network or processing delays (e.g., garbage collection stalls), we take precautions to avoid correctness
violations in (rare) cases when both primaries are active at the same time.

To this end, we maintain two shared objects, managed in Apache Zookeeper and accessed infrequently.
The first is \emph{maxTS}, which is the maximum timestamp an active TM is allowed to return to its clients.
An active (primary) TM periodically allocates a new \emph{epoch} of timestamps that it may return to clients by atomically
increasing maxTS (using a compare-and-swap (CAS) operation).  
Following failover, the new primary  allocates a new epoch for itself, and thus all the timestamps it returns to clients
exceed those returned by previous primary. 
The second is a locally-checkable \emph{lease}, which is essentially a  limited-time lock. 
As with locks, at most one TM may hold the lease at a given time (this requires the TMs' clocks to advance roughly at the same rate). 
This ensures that no client will be able to commit a transaction in an old epoch after the new TM has started 
using a new one.



