\documentclass[sigconf,10pt, nonacm]{acmart}


\setcopyright{rightsretained}

% DOI
\acmDOI{10.475/123_4}

% ISBN
\acmISBN{123-4567-24-567/08/06}

%Conference
\acmConference[SYSTOR'19]{ACM International Systems and Storage Conference}{June 2019}{Haifa, Israel}
\acmYear{2019}
\copyrightyear{2019}

%========================
%  Macros
%========================

\newcommand{\code}[1]{\textsf{\fontsize{9}{11}\selectfont #1}}

%\newcommand{\inred}[1]{{\color{red}{#1}}}
\newcommand{\inred}[1]{#1}
\newcommand{\remove}[1]{}
\newcommand{\Idit}[1]{[[\inred{Idit: #1}]]}
\newcommand{\Yoni}[1]{[\inred{Yoni: #1}]}
%\newcommand{\Ohad}[1]{{[\color{red}{Ohad: #1}]}}
\newcommand{\tb}{\hspace{10mm}}

\newcommand{\sys}{Omid~FP}
\newcommand{\sysll}{Omid~LL}
\newcommand{\syspc}{Omid~2PC}

\newcommand{\speedup}[1]{#1$\times$}
\newcommand{\tuple}[1]{\ensuremath{\langle \mbox{#1} \rangle}}

\newcommand{\mypara}[1]{ \vspace{8pt} \noindent{\bf #1}\hspace{6pt}}

%========================
\begin{document}






\title{Taking Omid to the Clouds: \\ 
Fast,  Scalable Transactions for Real-Time Cloud Analytics }

%\numberofauthors{6} 
\author{Ohad Shacham}
\affiliation{\institution{Yahoo Research}}
\email{ohads@verizonmedia.com}

\author{Yonatan Gottesman} %\footnotemark[1]
       \affiliation{\institution{Yahoo Research}}
       \email{yonatang@verizonmedia.com}

\author{Aran Bergman}
       \affiliation{\institution{Technion}}
       \email{aranb@campus.technion.ac.il}

\author{Edward Bortnikov}
       \affiliation{\institution{Yahoo Research}}
       \email{ebortnik@verizonmedia.com}

\author{Eshcar Hillel}
       \affiliation{\institution{Yahoo Research}}
       \email{eshcar@verizonmedia.com}

\author{Idit Keidar}
       \affiliation{\institution{Technion and Yahoo Research}}
       \email{idish@ee.technion.ac.il}
       
\renewcommand{\shortauthors}{O. Shacham et al.}




%=========================================================================
%  Abstract
%=========================================================================

\begin{abstract}


We describe how we evolve Omid, a transaction processing system for Apache HBase, 
to power Apache Phoenix, a cloud-grade real-time SQL analytics engine.  

Omid was originally designed for data processing pipelines at Yahoo, which are, by and large, 
%huge-scale 
throughput-oriented monolithic NoSQL applications. 
%Evolving Omid into 
Providing a platform 
to support 
%a variety of 
converged real-time transaction processing and analytics applications --
dubbed {\em translytics} --  
introduces new functional and performance requirements. For example, SQL 
support is key for developer productivity, multi-tenancy is essential for cloud deployment, 
and latency is cardinal for just-in-time data ingestion and analytics insights.
% in a variety of applications. 

We discuss our efforts to adapt Omid to these new domains, 
as part of 
%which emerged in 
the process of integrating it into Phoenix as the transaction processing backend. A central piece
of our work is latency reduction in Omid's protocol, which also improves  scalability.  
Under light load, the new protocol's latency is 4x to 5x smaller than the legacy Omid's, whereas 
under increased loads it is an order of magnitude faster. We further describe a \emph{fast path} 
protocol for single-key transactions, which enables processing them almost as fast 
as native HBase operations.

\end{abstract}
\maketitle

\section{Bibliographic Reference} 



\emph{
Ohad Shacham, Yonatan Gottesman, Aran Bergman, 
Edward Bortnikov, Eshcar Hillel, and Idit Keidar:}
{\bf Taking omid to the clouds: fast, scalable transactions for real-time cloud analytics}. 
Published in 
Proceedings of the VLDB Endowment,  
Volume 11 Issue 12, August 2018, Pages 1795-1808. 
\noindent
\url{https://dl.acm.org/citation.cfm?id=3275548}

\newpage

\section{Key Aspects of the Solution}

\begin{itemize}
\item 
Optimizing Omid for low latency.
\item 
Redesigning Omid to eliminate its principal bottleneck.
\item
Novel fast-path API for single-key transactions.
\item
Integrating Omid as the transactional layer in Phoenix SQL engine.
\end{itemize} 

\section{Why a Highlight Presentation?}


\begin{itemize}
\item 
Real-world solution: used in production in a popular platform, as part of an Apache incubator project. 
\item 
Pushes the state-of-the-practice in transaction processing systems. 
\item
Addresses challenges that arise in modern use cases. 
\end{itemize} 





\end{document}
