% Transactions in big-data platforms
In recent years, transaction processing technologies have paved their way into big data 
platforms that scale to many petabytes of data~\cite{Spanner2012,Percolator2010,Omid2017}. 
In some cases, they are built into the storage system itself~\cite{Spanner2012} whereas in the 
others they are standalone services~\cite{Omid2017}. Transaction management complements 
the underlying key-value storage with {\em atomicity}, {\em consistency}, {\em isolation\/} and 
{\em durability} (ACID) semantics~\cite{Gray:1992:TPC:573304} that enable programmers perform 
complex data manipulation without over-complicating their applications. The need for transaction awareness 
in web-scale applications started from specific use cases like real-time content indexing~\cite{Percolator2010,Omid2017} 
but quickly expanded to generic high-level abstractions, e.g., full-scale SQL OLTP and 
online analytics~\cite{Phoenix, F1-2013}.

Similarly to many technologies, the adoption of transactions took the "functionality-first" approach. 
For example, the developers of Google Spanner~\cite{Spanner2012} claim: ``We believe it
is better to have application programmers deal with performance problems due to overuse 
of transactions as bottlenecks arise, rather than always coding around the lack of transactions''. 
However, the expectation for high performance is picking up rapidly. For instance, the early-days  
transactions-enabled systems were throughput-oriented~\cite{Percolator2010,Omid2017}. 
With the thrust into new domains like messaging~\cite{Borthakur:2011} and algorithmic 
trading~\cite{opentsdb} that require interactive behavior, latency becomes king. 

Consider the design of a major email service provider backend. TBD. 

% \Idit{The roadmap below is not essential; maybe replace with summary of contributions.}
The remainder of this paper is organized as follows:
In Section~\ref{sec:api} we define the  API and semantics of a TPS. 
Section~\ref{sec:ll} presents \sys, without the fast path. 
Section~\ref{sec:alg} then describes our support for fast-path  transactions in \sys.  
Section~\ref{sec:eval} presents an empirical evaluation.
In Section~\ref{sec:context} we generalize the local transactions algorithm, and explain how it could be implemented in 
other TPSs. 
 We review related work in Section ~\ref{sec:related} and conclude with Section~\ref{sec:conclusions}.
